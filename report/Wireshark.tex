\documentclass[a4paper, 11pt]{report}
\usepackage[T1]{fontenc}
\usepackage[utf8]{inputenc}
\usepackage[english]{babel}
\usepackage{graphicx} % support graphics
\usepackage{hyperref} % links in the document
\usepackage{float} % position of figures
\usepackage{paralist} % inline lists

%\setcounter{tocdepth}{1} % Depth of table of contents

% Configure links in pdfs
\hypersetup{
    bookmarksopen=false, % Hide bookmarks menu
    colorlinks=true, % Don't wrap links in colored boxes
}

\title{Wireshark:\\ Automated generation of protocol dissectors\\
		Project Requirements}
\author{by Erik Bergersen, Sondre Johan Mannsverk,\\ Terje Snarby,
		Even Wiik Thomassen, Lars Solvoll Tønder,\\ Sigurd Wien
		and Jaroslav Fibichr}
\date{\today}

\begin{document}

\section{Wireshark}
Wireshark (previously Ethereal) is a free, open source network protocol analyzer. It lets you capture and browse traffic running through a computer network. Wireshark is currently being developed by the Wireshark team, a group of networking experts spanning the globe. Because of its rich set of features and ease of use, Wireshark is often used as a de facto standard in many different industries and educational institutions. Wireshark is able to display and dissect and display data from a plethora of different protocols, but one of its strengths lies in the ease of which developers can add their own dissectors, post-dissectors and taps. Developers can add their own dissectors either by using the programming language C or the LUA-scripting language. In general, most dissectors are written in C and compiled with Wireshark as C is several times faster than LUA. LUA-scripts on the other hand are mostly used as prototypes or to process non time crucial data. Our customer used Wireshark not only to browse through and filter regular networking traffic, but also for monitoring interprocess-traffic where it was important to have a tool which could easily be extended to dissect and display structures and data types unique to the organization.
See http://www.wireshark.org for more information on Wireshark.



\end{document}
