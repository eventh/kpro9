%=================
\chapter{Planning}
%=================
This chapter explains the administrative part of the project.


%---------------------
\section{Project Plan}
%---------------------
The project plan includes the specified plan on which tool we are using in this project, measurement of project effects, limitations and the concrete project work plan.

\subsection{Measurement of Project Effects}
%------------------------------------------
Automatic generation of Lua scripts from C header files would bring considerable resource savings in the customer's usual work process. Time and therefore financial resources, will be saved by delivery of the solution every time they need to know the contents of inter-process communication messages that include C structs.

The biggest impact on savings will be caused by enabling filtering of the messages by the specific attribute in the C struct in Wireshark. Once this is made possible, a lot of unnecessary searching can be omitted.

Before the project start, C structs were investigated in two ways. The first, manual method, which means counting individual bytes of the binary file that includes data in C structs. This is possible only for small-sized messages. For bigger messages, this method is inapplicable, since the message can consist of several thousands bytes. The second method consists of writing the dissector in Lua manually for the specific C header. Also, this method cannot be used for more complex C structs, i.e. those using nested structs. So far, there has been written about 10 Lua scripts manually.

According to the customer, there are approximately 3000 messages that include C structs and those need to be dissected by the solution. Time spent to write a dissector for a message manually depends on the struct’s complexity. For simple structs, this takes 15 minutes. It took about 1 hour for the most complicated dissectors that the customer had developed so far.

If 1 hour is the average time for creating a dissector for 1 message, our solution will save about 3000 hours of work. Due to everyday workload of the customer’s development team, this amount of time could never be used to accomplish such a task.

None of the workaround methods mentioned above is capable of processing messages with C structs that are big and complex. Time savings in these cases are not easily estimated.

Also, sometimes a representative of Thales Norway AS has to physically move to the their customer’s site to solve a problem. With the delivered solution, in some cases, this will be no longer necessary and the problem will be solved remotely by sending capture files to Thales and solving it in-house. Savings in this case are not only time-based but it will also directly cut the transportation costs and it will increase the satisfaction for the client.

\subsection{Limitations}
%-----------------------
As in all other projects, the project members have to deal with various limitations and constraints given either by the customer or simply by the fact that they are students.

\subsubsection{Technical Limitations}
\begin{description}
	\item[C preprocessor] To fulfill all the requirements we might need to
		either modify an existing preprocessor or write our own, which can be
		a huge undertaking.
	\item[Platforms] SPARC platform that are required for the program are not
		available to the project group.
\end{description}

\subsubsection{Non-technical Limitations}
\begin{description}
	\item[Experience] None in the group have experience with Lua-scripts,
		running a project with a larger team, or has planned a project before.
	\item[Time] The project group has a limited time of 12 weeks and a project
		deadline that cannot be changed. Also, the team consists of 7 members
		that have different schedules so finding a time when everyone is
		available for a meeting might be difficult. These limitations might
		lead to considerable delays in the project progress.
	\item[Language] In this project the team will have to write and speak in
		English, which is a second language for all team members. This may
		lead to misunderstandings and will negatively affect the time it
		takes to write the report.
\end{description}

\subsection{Tool Selection}
%--------------------------
To support collaboration and project management the team has considered and
selected the listed tools for use in this project.

\subsubsection{Git \& Github}
\begin{wrapfigure}{r}{3.5cm}
	\vspace{-20pt}
	\includegraphics[width=3.5cm]{./planning/img/git_logo}
	\vspace{-20pt}
\end{wrapfigure}
The team has selected Git as the version control system, with git repository
hosting provided by Github\footnote{\url{http://github.com/}}.

\begin{wrapfigure}{r}{3cm}
	\vspace{-20pt}
	\includegraphics[width=3cm]{./planning/img/github_logo}
	\vspace{-20pt}
\end{wrapfigure}
We had experience with CVS, SVN, Git and Mercurial, and although everyone 
knew SVN and only two knew Git, we selected Git for this project.
The main reason we didn't choose SVN was because of its lack of hosting capabilities,
and the other reason was that, unlike Git and Mercurial, SVN does not have any of the
advantageous features of a modern version control system like branches and a distributed
repository model.

We evaluated free hosting sites of version control systems, which could also 
provide us with other collaborative features that we wanted. Github, 
Bitbucket\footnote{\url{http://bitbucket.org/}} and
SourceForge\footnote{\url{http://sourceforge.net/}} all provided wiki and
issue tracker in addition to free version control system hosting. We eliminated 
SourceForge because their focus is divided between software users and 
developers, while the other two sites are fully focused on developers. The 
two remaining sites provides almost identical features, where one focuses on 
Git and the other on Mercurial.

Github with Git version control system was selected because more team members 
had experience with Git than Mercurial. Since we use different platforms,
we will also use different git clients, but for Windows most of the group has
selected tortoisegit.

\subsubsection{Skype}
\begin{wrapfigure}{r}{3cm}
	\vspace{-20pt}
	\includegraphics[width=3cm]{./planning/img/skype_logo}
	\vspace{-20pt}
\end{wrapfigure}
Skype\footnote{\url{http://www.skype.com}} is an application which allows the
user to make video and voice chats over the Internet, including conference
calls and chatting. The team will use Skype to communicate and collaborate when
we are not physically at the same location at the same time.

\subsubsection{Google Docs}
\begin{wrapfigure}{r}{2cm}
	\vspace{-20pt}
	\includegraphics[width=2cm]{./planning/img/google_docs_logo}
	\vspace{-20pt}
\end{wrapfigure}
Google Docs\footnote{\url{http://docs.google.com/}} is a free web site offering
functionality for creating documents, spreadsheets and presentations.
The benefits of using Google Docs are that it is easy to share documents with other users, and
it is possible to collaborate in real-time. For this project we are going to use Google Docs
to collaborate on document drafts, and to share documents within the team and
with the team's advisor.

\subsubsection{\LaTeX}
\LaTeX \,is a document markup language used to
create reports, articles and books. It was chosen by the group for its 
high quality typesetting which produces professional looking documents, and 
because it is suitable for larger scientific reports. 
Writing documents in \LaTeX \ is very different from writing them in, for example,
Microsoft Word, as most of the visual presentation is handled by the \LaTeX \ system 
and not by the user itself. Because the writer does not have to spend time
thinking about how the document looks, he can focus entirely on the content.
\LaTeX \,also provides automatic numbering of chapters and sections,
automatic generation of table of contents, cross-referencing and bibliography
management. Since \LaTeX \,files are plain text files they are suitable for versioning 
with a version control system like Git. We will use \LaTeX \,to write the final project 
report, and we have created a few templates for test plans and minutes.

\subsubsection{Mailing List}
\begin{wrapfigure}{r}{4cm}
	\vspace{-20pt}
	\includegraphics[width=4cm]{./planning/img/ntnu_logo}
	\vspace{-20pt}
\end{wrapfigure}
For asynchronous communication the team uses an electronic mailing list
provided by IDI, NTNU.

\subsubsection{Google Calendar}
\begin{wrapfigure}{r}{3cm}
	\vspace{-20pt}
	\includegraphics[width=3cm]{./planning/img/google_calendar_logo}
	\vspace{-20pt}
\end{wrapfigure}
Since all team members have google accounts, we have created a group calendar
in Google Calendar\footnote{\url{http://calendar.google.com/}}
to help schedule and keep track of meetings. A single calendar which all
members can include in their own, prevents misunderstandings and duplication
of work.

\subsubsection{text2pcap}
In this project text2pcap is used to generate capture files from ASCII
hexdumps. Since the customer cannot provide us with capture files, 
we have to create them ourselves. The capture files are important for testing
the generated Lua-scripts. The text2pcap tool is included with Wireshark. 
The input to the tool is a ASCII hexdump as a text-file, and the output will
be a pcap-file. When running text2pcap, it is necessary to specifiy the protocol
type. In this project the team uses ''-l 147'' as an option to text2pcap,
which is a user defined link-layer type.

\subsubsection{Hex Editor}
The team is using a hex editor to create input for text2pcap. To make it
easier to write ASCII hexdumps, it is necessary to write them in a hex editor.
HxD\footnote{\url{http://mh-nexus.de/en/hxd/}} is the recommended hex editor
for this project, since it is free and has the necessary functionality. HxD is
only for Windows.

\subsubsection{Violet}
\begin{wrapfigure}{r}{3cm}
	\vspace{-20pt}
	\includegraphics[width=3cm]{./planning/img/violet_logo}
	\vspace{-20pt}
\end{wrapfigure}
Violet\footnote{\url{http://violet.sourceforge.net/}} is a free
and easy to use modeling software for making UML-diagrams.
It is also a cross platform solution, which means that all team members can use
the same application. If we had to use different UML applications we would have problems
editing the diagrams due to incompatible file formats.
The architectural and design team will be using violet to create diagrams
to illustrate the workings of the utility. As very advanced diagrams were not needed for
this project, violet seemed like a fitting tool.

\subsection{Based on the Planned Effort}
%---------------------------------------
Calculations done by the course staff suggests that each student should use 325
person-hours distributed over 13 weeks. Our group, consisting of seven
students, will have a total of 2275 person-hours to spend on the project.

\subsection{Schedule of Results}
%-------------------------------
This project will have two deliveries, pre-delivery and final delivery.
The milestones and sprints are listed below.

\subsubsection{Milestones}
\begin{tabular}{l l}
	30. August & Project start \\
	6. October & Pre-delivery of project report \\
	24. November & Final delivery of project report \\
	24. November & Presentation and project demo \\
\end{tabular}

\subsubsection{Sprints}
\begin{tabular}{l l}
	Sprint I & 14. September - 27. September \\
	Sprint II & 5. October - 18. October \\
	Sprint III & 19. October - 1. November  \\
	Sprint IV & 2. November - 15. November \\
\end{tabular}

\subsection{Concrete Project Work Plan}
%--------------------------------------
The two first weeks of the project will be used on planning and pre study.
The project will be divided into 4 sprints, that will each last for two weeks.
The first sprint will have an estimated 200 person-hours, while the last three
sprints will have 250 person-hours. The last 1.5 weeks will be used to finish
the final report and prepare for the presentation. \autoref{tab:wbs} shows the
work breakdown structure, and the project timeline is in \autoref{fig:gantt}.

\begin{table}[!htb] \footnotesize \center
\caption{Work Breakdown Structure\label{tab:wbs}}
\begin{tabular}{l l l c c}
	\toprule
	& & & \multicolumn{2}{c}{Effort} \\
	\cmidrule(r){4-5}
	Task & From date & To date & Estimated & Actual  \\
	\midrule
	Misc & 30.08.2011 & 24.11.2011 & 875 & 494  \\
	\midrule
	Project Management & 30.08.2011 & 24.11.2011 & 275 &161 \\
	Lectures & 02.09.2011 & 18.10.2011 & 100 & 60 \\
	Self Study & 30.08.2011 & 04.10.2011 & 150 & 44 \\
	Planning & 05.09.2011 & 12.09.2011 & 150 & 122 \\
	Pre-study & 05.09.2011 & 12.09.2011 & 100 & 49 \\
	Requirement Specification & 05.09.2011 & 12.09.2011 & 100 & 58 \\
	\midrule
	Sprint I & 14.09.2011 & 27.09.2011 & 200 & 157 \\
	\midrule
	Sprint I Planning & 14.09.2011 & 14.09.2011 & 30 & 29 \\
	Sprint I Work & 15.09.2011 & 26.09.2011 & 150 & 98 \\
	Sprint I Review & 27.09.2011 & 27.09.2011 & 20 & 30 \\
	\midrule
	Sprint II & 05.10.2011 & 18.10.2011 & 250 & 0 \\
	\midrule
	Sprint II Planning & 05.10.2011 & 05.10.2011 & 30 & 0 \\
	Sprint II Work & 06.10.2011 & 17.10.2011 & 200 & 0 \\
	Sprint II Review & 18.10.2011 & 18.10.2011 & 20 & 0 \\
	\midrule
	Sprint III & 19.10.2011 & 01.11.2011 & 250 & 0 \\
	\midrule
	Sprint III Planning & 19.10.2011 & 19.10.2011 & 30 & 0 \\
	Sprint III Work & 20.10.2011 & 31.10.2011 & 200 & 0 \\
	Sprint III Review & 01.11.2011 & 01.11.2011 & 20 & 0 \\
	\midrule
	Sprint IV & 02.11.2011 & 15.11.2011 & 250 & 0 \\
	\midrule
	Sprint IV Planning & 02.11.2011 & 02.11.2011 & 30 & 0 \\
	Sprint IV Work & 03.11.2011 & 14.11.2011 & 200 & 0 \\
	Sprint IV Review & 15.11.2011 & 15.11.2011 & 20 & 0 \\
	\midrule
	Report \& Presentation & 30.08.2011 & 24.11.2011 & 450 & 59 \\
	\midrule
	Write Report & 20.08.2011 & 24.11.2011 & 375 & 0 \\
	Presentation & 22.11.2011 & 24.11.2011 & 75 & 0 \\
	\midrule
	Total & 30.08.2011 & 24.11.2011 & 2275 & 710 \\
	\bottomrule
\end{tabular}
\end{table}

\begin{figure}[!htb]
	\noindent\makebox[\textwidth]{%
	\includegraphics[scale=0.48]{./planning/img/gantt}}
	\caption{Gantt diagram\label{fig:gantt}}
\end{figure}


%-----------------------------
\section{Project Organization}
%-----------------------------
This section describes how the team is organized, which roles the developers are divided into and the partners of the projet.

\subsection{Project Organization}
%--------------------------------
Our project organization has a flat structure and the organization chart can be seen in figure \ref{fig:orgchart}, the roles listed in the organization chart are described in table \ref{tab:roles}.

\begin{figure}[htb]
	\noindent\makebox[\textwidth]{%
	\includegraphics[scale=0.45]{./planning/img/organization}}
	\caption{Project Organization\label{fig:orgchart}}
\end{figure}

\begin{table}[!htb] \footnotesize \center
\caption{Project Roles\label{tab:roles}}
\noindent\makebox[\textwidth]{%
\begin{tabularx}{\textwidth}{l X}
	\toprule
	Role name & Main responsibilities  \\
	\midrule
	Project manager & Responsible for having an overview of the project, delegating jobs and resolving conflicts. \\ 
	Adviser contact & Responsible for distributing information between the group and the advisor. \\
	Organizer & Responsible for setting up and informing the group about the meeting schedule. \\ 
	Document master & Responsible for document quality and quantity. \\ 
	System architect & The lead designer of the system. \\ 
	Lead programmer & Makes sure everyone follows the agreed code standards and ensures the quality of the code. \\ 
	Customer contact & Responsible for distributing information between the group and the customer. \\ 
	Technically responsible & Finds good technical solutions and makes sure that the essential tools are operative. \\ 
	Technology evangelist & Brings in ideas about new technologies and tools. \\ 
	Scrum master & Responsible for scrum meetings. \\ 
	Lead tester & Responsible for good test coverage, both unit and end to end, and to ensure the quality of those tests. \\ 
	Secretary & Takes note from each meetings and stores it in the cloud. Responsible for preparing minutes for advisor/customer. \\
	\bottomrule
\end{tabularx}}
\end{table}

\subsection{Partners}
%--------------------
This subsection lists the partners of this project. The customer of this
project is Thales Norway AS, which are located at Lerkendal Stadium,
Strindveien 1, 7030 Trondheim. The customer contacts are listed in
\autoref{tab:plan:customer}. The development team consist of seven student
from NTNU, and are listed in \autoref{tab:plan:devs}. The team is assigned two
advisors from the Department of Computer and Information Science at NTNU,
listed in \autoref{tab:plan:advisors}. 

\begin{table}[!htb] \footnotesize \center
\caption{Customers\label{tab:plan:customer}}
\begin{tabular}{l l l}
	\toprule
	Name & Mobile & E-mail \\ 
	\midrule
	Christian Tellefsen & 959 98 765 & christian.telefsen@thalesgroup.com \\ 
	Stig Bjørlykke & 982 29 806 & stig.bjorlykke@thalesgroup.com \\ 
	\bottomrule
\end{tabular}
\end{table}

\begin{table}[!htb] \footnotesize \center
\caption{Developers\label{tab:plan:devs}}
\begin{tabular}{l l l}
	\toprule
	Name & Mobile & E-mail  \\ 
	\midrule
	Terje Snarby & 915 27 390 & snarby@stud.ntnu.no \\ 
	Even Wiik Thomassen & 991 61 929 & evenwiik@stud.ntnu.no \\ 
	Sondre Johan Mannsverk & 948 15 506 & sondrejo@stud.ntnu.no \\ 
	Erik Bergersen & 917 48 305 & eribe@stud.ntnu.no \\ 
	Lars Solvoll Tønder & 976 00 317 & larssot@stud.ntnu.no \\ 
	Sigrud Wien & 472 54 625 & sigurdw@stud.ntnu.no \\ 
	Jaroslav Fibichr & 451 26 314 & jaroslaf@stud.ntnu.no \\ 
	\bottomrule
\end{tabular}
\end{table}

\begin{table}[!htb] \footnotesize \center
\caption{Advisors\label{tab:plan:advisors}}
\begin{tabular}{l l l}
	\toprule
	Name & Mobile & E-mail \\ 
	\midrule
	Daniela Soares Cruzes & 942 49 891 & dcruzes@idi.ntnu.no \\ 
	Maria Carolina Mello Passos & & mariacm@idi.ntnu.no \\ 
	\bottomrule
\end{tabular}
\end{table}


%--------------------------
\section{Quality Assurance}
%--------------------------
The following section contains internal processes and routines the team will use in the project. This includes procesures for meetings, document templates and standards and internal reports.

\subsection{Routines for Ensuring Quality Internally}
%----------------------------------------------------
We will organize in pairs when producing items, where the pair reviews each others work. This is done in an effort to enhance the quality of the project, as we will be able to find more errors, and 
also get a broader perspective on style and solutions.

We also has assigned quality assurance responsibilities for three subjects: documents, code and tests. The respective team members will try to have a birds eye overview in their area to catch further errors.

We have agreed to have three weekly meetings to accommodate these routines.
\begin{itemize}
	\item Monday 12-14
	\item Wednesday 12-17
	\item Friday 10-13
\end{itemize}

\subsection{Phase Result Approval}
%---------------------------------
To ensure the quality of the sprint deliverables, the group members responsible for quality assurance for a given subject will either read through, or delegate that task to another free group member.

We will also present the results for the customer and advisor. They will then have an opportunity to point out problems and misunderstandings, and suggest solutions. This is a result of the Scrum methodology: deliveries and deadlines throughout the project, making the progress very visible to the customer and advisers. We will not be able to fail big in the end of the project because we have been guided by the feedback during the process of making the utility and report. If a problem appears, we will try to correct the it and then reiterate the quality assurance.

\subsection{Procedures for Customer Meetings}
%--------------------------------------------
All customer meetings should be called with time, place, agenda specified. All background documents relevant to the meetings should also be supplied. This is to ensure efficient and effective meetings.

All customer meetings should be summarised in a document (minute). This document should include:
\begin{itemize}
	\item Time of meeting
	\item Place
	\item Version
	\item Meeting responsible
	\item Names of the attendees
	\item Decisions
	\item Actions
	\item Clarifications
	\item The above should be in sequence according to time
\end{itemize}

This summary should be done and sent to the customer by 12:00 the day after the meeting. If the customer does not approve the minutes, the minutes should again be corrected by and sent 12:00 the following day. The customer contact is responsible for the above tasks.

The customer has committed to respond to our interactions within 2 working days.

\subsection{Procedure for Advisor Meeting}
%-----------------------------------------
The weekly advisor meeting will be at 10:30 every Friday unless otherwise stated.

\subsubsection{Agenda for Meeting}
A meeting with the advisor should be called before 14:00 the day before the meeting, and this calling should include:
\begin{itemize}
	\item Time
	\item Place
	\item Agenda
	\item Status report
	\item Table of reported working hours
	\item Minutes for the last meeting
	\item Other relevant documents
\end{itemize}

\subsubsection{Minutes from Meeting}
The minutes should be written and sent to the advisor for approval before 12:00 the day after the meeting. If the advisor should reject the minutes, they should be corrected and resent 12:00 the day following the rejection. The minutes should include:
\begin{itemize}
	\item Time of meeting
	\item Place
	\item Version
	\item Names of the attendees
	\item Decisions
	\item Actions
	\item Clarifications
	\item A rough timeline of the above
\end{itemize}

\subsection{Document Templates and Standards}
%--------------------------------------------
The team has specified procedures for templates and file organization.

\subsubsection{Templates the Group Has Created}
All templates are stored under docs/ on Github, the team has templates for:

\begin{itemize}
	\item Meeting agenda
	\item Status report
	\item Meeting minutes
\end{itemize}

\subsubsection{Standard for Organizing Files}
We use Github and Google Docs to store the files included in this project. The
location of a file is dependent on what type the file is. 

\begin{itemize}
	\item All source code is to be saved in the Github repository under
		CSjark/. This ensures that all the team members have the current
		version of the code. The structure for this folder:
		\begin{verbatim}
CSjark/
    csjark/  -- todays source/, for source code
        test/  -- for unit tests
        etc/  -- for configuration files
        headers/  -- header files used to test the program
    bin/  -- file for executing our program
    docs/  -- for CSjark-specific documentation
    utils/  -- for cpp.exe and fake header files
		\end{verbatim}
	\item All textual documents that are completed will be put in the
		docs/ folder.
	\item All LaTeX documents are stored in the Github repository
		under report/. The structure for this folder:
		\begin{verbatim}
report/
    planning/  --   Planning & Requirements part
        img/  --  Images for this part
    sprints/  --  Sprints part
        img/  --  Images for this part
    evaluation/  --  Evaluation part
    appendices/  --   Appendices part
		\end{verbatim}
	\item Examples of header-files and Lua-scripts, packet capture-files and information from customer is stored under wireshark/.
\end{itemize}

\subsubsection{File Name Standard}
The file name should consist of the name of the document (meeting minutes,
agenda, phasedoc, e.g.) and the date, if applicable.

\subsubsection{Coding Style Standard}
The programming language used to implement the utility specified by the
customer requirements is Python. The coding style the team has agreed upon is
the Python Standard Styling Guide as defined by
PEP8\footnote{Style Guide for Python Code: \url{http://www.python.org/dev/peps/pep-0008/}}.
In addition the design should attempt to be pythonic, as detailed by
PEP20\footnote{The Zen of Python \url{http://www.python.org/dev/peps/pep-0020/}}.

\subsection{Version Control Procedures}
%--------------------------------------
Every relevant digital item should be pushed to our repository at github, and be checked out by other participants. Those who work on a given item should commit and push their changes often, so that others can be as up to date as possible. All digital items should be labeled with a version number, starting at version 1. If an item goes under review and is deemed insufficient by the customer, the version number should also be incremented by one for each revision of the document

Relevant digital items includes source code, documents, picture files, binary blobs, etc.

NB: Google docs is not to be used for version control, so every document written there should also be pushed to git hub.

\subsection{Internal Reports}
%----------------------------
Some of the groups' internal activities should be documented. This includes:
\begin{itemize}
	\item Activities, what is done, and what remains
	\item Minutes for internal meetings
	\item Milestones, complete/incomplete
	\item Effort registration shall be done daily by each team member.
\end{itemize}
These documents should follow the templates specified in Templates and Standards (A4) if applicable.


%------------------------
\section{Risk Management}
%------------------------
The following section lists the possible risk scenarios that can occur in the project, and how they should be handled. \autoref{tab:risk} shows how to handle the possible risks in the team. Each risk has a consequence and a probability:  High (H), Medium (M) or Low (L).
Strategy and actions describes what we should do to reduce the consequences of the risk, or prevent the risk from happening altogether. Deadline states when we need to handle the risk.
\begin{description}
\item[R1. Choosing an incompatible technical solution]  The team decides to use a technical solution that is not suited for the given problem, or decides on an implementation that is too time consuming.
\item[R2. Too much focus on report]   We spend too much time working on the report and neglect the implementation. 
\item[R3. Too much focus on implementation]  We spend too much time working on the implementation and neglect writing all the needed documentation for the report.
\item[R4. Illness/Absence]  Members of the team become ill or are otherwise unavailable. 
\item[R5. Even is absent]  Even is the only one in the group that is highly proficient in Python, if he is absent for a long period, the overall development process of the utility will be hindered.
\item[R6. Conflicts within group]   Internal conflicts which hinders the group's ability to work together. 
\item[R7. Lack of technical competence]  The team lack the needed technical ability to solve the given problem. 
\item[R8. Miscommunication within team]  Team members don’t know what to do, or misunderstands the task given to them. 
\item[R9. Miscommunication with customer]  The team misunderstands the requirements given by the customer. 
\item[R10. Lack of experience with Scrum]  The team does not have any experience in doing Scrum projects.
\end{description}

\begin{longtable}{l p{9cm}}
	\caption{Handling Risks}
	\endhead
	\toprule
	Risk ID & R1 \\
	Risk factor & Choosing an incompatible technical solution \\
	Consequences & H: The project will not be completed on time, or at all. \\
	Probability & M \\ 
	Strategy \& actions & Do a good pre-study, consult the customer’s technical expert. \\
	Deadline & During the first sprint.\\
	Responsible & Even and Erik \\
	\midrule
	Risk ID & R2 \\
	Risk factor & Too much focus on report \\
	Consequences & M: The product will not be of a satisfying quality. \\
	Probability & M \\ 
	Strategy \& actions & Plan enough hours to use on the customer product. Look at old reports for inspiration. Ask the advisor for help. \\
	Deadline & Continuous\\
	Responsible & Sondre \\
	\midrule
	Risk ID & R3 \\
	Risk factor & Too much focus on implementation \\
	Consequences & H: The documentation will not be good enough, leads to a bad grade. \\
	Probability & M \\ 
	Strategy \& actions & Plan enough hours to use on the report. Write documentation sooner later than later. Write good requirements that limit the scope of the project. \\
	Deadline & Continuous \\
	Responsible & All \\
	\midrule
	Risk ID & R4 \\
	Risk factor & Illness/Absence \\
	Consequences & L/M/H: Consequences depend on how many members are absent, and how often they are absent. Absence may hinder the progress of the project in different ways.  \\
	Probability & M \\ 
	Strategy \& actions & Make sure several people are proficient in the technical parts of the project. Have backups for the most important roles. \\
	Deadline & Continuous \\
	Responsible & Terje \\
	\midrule
	Risk ID & R5 \\
	Risk factor & Even is absent \\
	Consequences & H \\
	Strategy \& actions & All group members should read through the source code and code documentation written by Even. \\
	Deadline & Continuous \\
	Responsible & All \\
	\midrule
	Risk ID & R6 \\
	Risk factor & Conflicts within group \\
	Consequences & M: May lead to bad morale, which could affect the work of the team. Could also be a waste of time. \\
	Probability & M \\ 
	Strategy \& actions & Not all conflicts are bad. If the conflict is simply a disagreement over technical issues, or the planning of the project, it could benefit the team. All such conflicts should lead to a constructive discussion that the entire team should take part in. Other types of conflicts, that can not positively influence the project should be avoided if possible. The team should agree on specific ground rules. \\
	Deadline & Continuous \\
	Responsible & Terje \\
	\midrule
	Risk ID & R7 \\
	Risk factor & Lack of technical competence \\
	Consequences & H: The team is unable to solve the problem  \\
	Probability & H \\ 
	Strategy \& actions & Make sure the team is proficient in the programming languages and tools that are to be used. Decide on technical solutions that the team already is familiar with. Do a good pre-study of the parts the team is unfamiliar with. Consult with the customer’s technical expert. \\
	Deadline & Early phases of development (Pre-study, first sprint) \\
	Responsible & All \\
	\midrule
	Risk ID & R8 \\
	Risk factor & Miscommunication within team \\
	Consequences & M: Team members waste time doing nothing, or doing something that is irrelevant. \\
	Probability & M \\ 
	Strategy \& actions & Make sure that everyone knows what to do at all times. Ask questions if you are unsure about your specific task. \\
	Deadline & Continuous \\
	Responsible & Terje \\
	\midrule
	Risk ID & R9 \\
	Risk factor & Miscommunication with customer \\
	Consequences & H: The team waste time on functionality the customer did not want. The completed product does not do what the customer wanted. \\
	Probability & M \\ 
	Strategy \& actions & Make sure we and the customer have a common understanding of the requirements. Have frequent meetings with the customers.  \\
	Deadline & Continuous \\
	Responsible & Sigurd \\
	\midrule
	Risk ID & R10 \\
	Risk factor & Lack of experience with Scrum \\
	Consequences & M: The team does not provide the correct documents for the report, which could lead to a bad grade. \\
	Probability & M \\ 
	Strategy \& actions & Learn how to properly do Scrum. Have Scrum meetings as often as possible. Get feedback on documents from advisor. \\
	Deadline & Continuous \\
	Responsible & Jaroslav \\
	\bottomrule
	\label{tab:risk}
\end{longtable}

