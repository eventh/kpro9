%######################################################
% This file generates the whole report for the project
% Run with "pdflatex report.tex" twice to create pdf
%
% Use *_standalone.tex to generate standalone parts
%######################################################
\documentclass[a4paper, 11pt]{report}
\usepackage[T1]{fontenc}
\usepackage[utf8]{inputenc}
\usepackage[english]{babel}
\usepackage{graphicx} % support graphics
\usepackage{hyperref} % links in the document
\usepackage{float} % position of figures
\usepackage{paralist} % inline lists
\usepackage{verbatim} % mulitline comments
\usepackage[table]{xcolor} % table row coloring
\usepackage{booktabs} % Professional tables
\usepackage{tabularx} % Simple column stretching
\usepackage{multirow} % Row spanning
\usepackage{wrapfig} % Wrap text around figures
\usepackage{array}
\usepackage{longtable}

%\setcounter{tocdepth}{1} % Depth of table of contents

% Configure links in pdfs
\hypersetup{
    bookmarksopen=false, % Hide bookmarks menu
    colorlinks=true, % Don't wrap links in colored boxes
}


%############
% Top matter
%############

\title{Wireshark:\\ Automated generation of protocol dissectors}
\author{by\\ Erik Bergersen, Sondre Johan Mannsverk,\\ Terje Snarby,
		Even Wiik Thomassen, Lars Solvoll Tønder,\\ Sigurd Wien
		and Jaroslav Fibichr}
\date{\today}

\begin{document}

%============
% Title page
%============
\maketitle

%==========
% Abstract
%==========
\begin{abstract}
This report documents the work done in a project in the course TDT4290 
Customer Driven Project at NTNU. The project was executed on behalf of
Thales Norway AS.

The project consisted of seven students from the Norwegian University 
of Technology and Science.

The task was to develop a tool for Wireshark that could automatically
dissect C-structs. The utility creates Lua scripts which act as package
dissectors in Wireshark. The tool itself was written in Python. In addition 
to this the tool supports configuration-files written in YAML.

During the development process we used the Scrum methodology for
management. The development process consisted of 4 two-week long
sprints, with new functionality added in each sprint.

The project duration was from the 30th August to the 24th of November.
\end{abstract}

%=========
% Preface
%=========
\pagenumbering{roman}
\chapter*{Preface}
\addcontentsline{toc}{chapter}{Preface}
TODO

%===================
% Table of Contents
%===================
\clearpage
\phantomsection
\addcontentsline{toc}{chapter}{Contents}
\tableofcontents

%=================
% List of Figures
%=================
\clearpage
\phantomsection
\addcontentsline{toc}{chapter}{List of Figures}
\listoffigures

%================
% List of Tables
%================
\clearpage
\phantomsection
\addcontentsline{toc}{chapter}{List of Tables}
\listoftables


%###########################
% Planning and Requirements
%###########################
\clearpage
\pagenumbering{arabic}
\part{Planning \& Requirements}

%==========================
\chapter{Project Directive}
%==========================
This chapter will briefly introduce the project, its background and purpose.


%------------------------
\section{Project Mandate}
%------------------------
The purpose of this project is to develop an \gls{utility} that automatically creates \Gls{lua}-\glspl{dissector} for \Gls{wireshark}, from \Gls{c}-\gls{header} files. This report presents the team’s process from the initial requirement specification to the finished product. 

The title of the project is "\Gls{wireshark} - Automated generation of \gls{protocol} \glspl{dissector}", it was given to us by the customer and describes exactly what we are planning to accomplish.The name chosen for the \gls{utility} is "CSjark". Sjark is the Norwegian name for an iconic type of fishing boat, most commonly used in Northern Norway. The reason why the team picked this name was because of the way the \gls{utility} "fishes" for \Gls{c}-\glspl{struct} in \gls{header} files. The \gls{utility} then creates \glspl{dissector} for these \glspl{struct} so that \Gls{wireshark} can display the \gls{struct} information properly. This reminded the team of what fishermen do to prepare the fish for the market. The word Sjark is also pronounced in a similar way to "shark", which makes our \gls{utility}-name a play on words when comparing it to "\Gls{wireshark}", the program our \gls{utility} is supposed to work with.


%-------------------
\section{The Client}
%-------------------
The client for this project is
Thales Norway AS\footnote{\url{http://www.thales.no/}}. Thales is an
international electronics and systems group, which focuses on defence,
aerospace and security markets worldwide. The Norwegian branch primarily
supplies military communication systems, used by the Norwegian Armed Forces
and other \Gls{nato} countries. Thales Norway AS consists of 170 highly skilled
employees, which offers a wide range of technical competence.


%--------------------------------
\section{Involved parties}
%--------------------------------
The following parties are involved in this project.

The client, described in the section above, is represented by Christian Tellefsen and Stig Bjørlykke. See \autoref{tab:plan:customer} for their contact information.
The project team consists of seven computer engineering students from \Gls{ntnu}, as described in \autoref{tab:plan:devs}.
For feedback and help during the project period our team has been assigned a main supervisor, Daniela Soares Cruzes.
She will be assisted by Maria Carolina Passos. Their contact information can be found in \autoref{tab:plan:advisors}.

%----------------------------
\section{Project Background}
%---------------------------
Thales currently uses \Gls{wireshark} to analyze traffic data between different network nodes, for example, \Gls{ip} \glspl{packet} sent between a client and a server.
They want us to extend the functionality of \Gls{wireshark}, so that they could use it to monitor internal data between \glspl{process}. The extended functionality will be automatic generation of \Gls{lua}-\glspl{dissector} from \Gls{c}-\gls{header} files.

Before, when Thales wanted to debug with \Gls{wireshark}, they had to write the \glspl{dissector} manually. Their hope is therefore that our tool can save them valuable time.

%--------------------------
\section{Project Objective}
%--------------------------
The objective from the customer is to design a \gls{utility} that will be able to generate \Gls{lua} code for dissecting the \gls{binary} representation of \Gls{c}/\Gls{c++} \glspl{struct}, allowing \Gls{wireshark} to show, filter, and search through the data.
 The \gls{utility} needs to support a flexible configuration, as this will make it useable for complex and specific \gls{header} files. 
The code and configuration should be well documented, making it easy for Thales to use and extend the tool as they see fit.

The objective from \Gls{ntnu}'s point of view is that we acquire practical experience in executing all phases of a bigger \Gls{it}-project and learn
how to work in a team.

Our team's goals for this project are to attain experience in working in a real development project, and to create a solution that the customer
is satisfied with.

%-----------------
\section{Duration}
%-----------------
Calculations done by the course staff suggests that each student should conduct 325 person-hours distributed over 13 weeks for the project. Our group, consisting of seven students, will have a total of 2275 person-hours to spend.\\
\begin {itemize}
	\item Project start: August 30th
	\item Project end (Presentation): November 24th
\end{itemize}


\section{Project Plan}


\subsection{Measurement of Project Effects}
Automatic generation of Lua scripts from c-header files would bring considerable resource savings in the customers usual work process. Time (and therefore financial resources) will be  be saved by delivery of the solution every time they need to know the contents of investigated IPC messages that include C struct(s).

The most of the savings will be caused by enabling filtering of the messages by the specific attribute in the C struct in Wireshark. Once this will be possible, a lot of unnecessary searching can be omitted.

Before the project start, C structs were investigated in two ways. The first, manual method, which means counting individual bytes of the binary file that includes data in C structs. This is possible only for small-sized messages. For bigger messages, this method is inapplicable, since the message can consist of several thousands bytes. The second method consists of writing the dissector (as Lua script) manually for the specific C header. Also, this method cannot be used for more complex C structs, i.e. those using nested structs. So far, there has been written about 10 Lua scripts manually.

According to the customer, there are approximately 3000 messages that include C structs and those need to be dissected by the solution. Time spent to write a dissector for a message manually depends on the struct’s complexity. For trivial messages, it takes 15 minutes to create a dissector manually. It took about 1 hour for the most complicated dissectors that were developed by the customer so far.

If 1 hour is the average time for creating a dissector for 1 message, our solution will save about 3000 hours of work. Due to everyday workload of the customer’s development team, this amount of time could never be used to accomplish such a task.

None of the workaround methods mentioned above is capable of processing messages with C structs that are big and complex. Time savings in these cases are not easily estimated.

Also, sometimes a representative of Thales Norway AS has to physically move to the their customer’s site to solve a problem. With the delivered solution, in some cases, this will be no longer necessary and the problem will be solved remotely by sending capture files to Thales and solving it in-house. Savings in this case are not only time-based but it will also directly cut the transportation costs and it will increase the satisfaction for the client.

\subsection{Limitations}
As in all other projects, the project members have to deal with various limitations and constraints given either by the customer or simply by the fact that they are students.

\subsubsection{Technical Limitations}
\begin{itemize}
	\item \textbf{C preprocessor:} To fulfill all the requirements we might need to either modify an existing preprocessor or write our own, which can be a huge undertaking.
	\item \textbf{Platforms:} SPARC platform that are required for the program are not available to the project group. 
\end{itemize}

\subsubsection{Non-technical Limitations}
\begin{itemize}
	\item \textbf{Experience:} None in the group has experience with Lua-scripts, running a project with larger team, or has planned a project before.
	\item \textbf{Time:} The project group has limited time of 12 weeks and a project deadline that cannot be changed. Also, the team consists of 7 members that have different schedules and so finding a time when everyone is available for a meeting might be difficult. These limitations might lead to considerable delays in the project progress.
	\item \textbf{Language:}         Language: In this project the team will have to write and speak in English, which is a second language for all team members. This may lead to misunderstandings and will negatively affect the time it takes to write the report.
\end{itemize}

\subsection{Tool Selection}
To support collaboration and project management the team has considered and selected the listed tools for use in this project.

\begin{itemize}
\item \textbf{Git \& Github:} 

The team has selected Git as the version control system, hosted at Github.com.

We had experience with CVS, SVN, Git and Mercurial, and although everyone knew SVN and only two knew Git, we selected Git for this project. We evaluated free hosting sites of version control systems, which could also provide us with other collaborative features that we wanted. github.com, bitbucket.com and sourceforge.net all provided wiki and issue tracker in addition to free version control system hosting. We eliminated sourceforge.net because their focus is divided between software users and developers, while the other two sites are fully focused on developers. The two remaining sites provides almost identical features, where one focuses on Git and the other on Mercurial.

Github with Git version control system was selected because more team members knew Git than Mercurial. By selecting a newer version control system like Git, we get advantageous features like branches and distributed repository model. Since we use different platforms, we will also use different git clients, but for Windows most of the group has selected tortoisegit.

\item \textbf{Skype:} 

Skype is an application which allows the user to make video and voice chats over the Internet, including conference calls and chatting. The team will use Skype to communicate and collaborate when we are not physically at the same location at the same time.

\item \textbf{Google Docs:} 

Google Docs is web-based office suite, which include documents, spreadsheets and presentations. It makes it possible to collaborate in real-time. For this project we are going to use Google Docs to collaborate on document drafts, and to share documents within the team and with the teams advisor.

\item \textbf{LaTeX}

LaTeX is a document markup language and document preparation system used to create reports, articles and books. LaTeX was chosen by the group for its high quality typesetting which produces professional looking documents, and because it is suitable for larger scientific reports. LaTeX also provides automatic numbering of chapter and sections, automatic generation of table of contents, cross-referencing and bibliography management. Since LaTeX files are plain text files they are suitable for versioning with a version control system like Git. We will use LaTeX to write the final project report, and we have created a few templates for test plans and minutes.

\item \textbf{Mailing list}

For asynchronous communication the team uses an electronic mailing list provided by IDI, NTNU.

\item \textbf{Google Calendar}

Since all team members have google accounts, we have created a google calendar to help schedule and keep track of meetings. A single calendar which all members can include in their own prevents misunderstandings and duplication of work.

\end{itemize}

\subsection{Based on the planned effort}
Calculations done by the course staff suggests that each student should use 325 person-hours distributed over 13 weeks. Our group, consisting of seven students, will have a total of 2275 person-hours to spend on the project. 

\subsection{Schedule of Results}
This project will have two deliveries, pre-delivery and final delivery.  The milestones and sprints are listed below. \newline
\textbf{Milestones}

\begin{tabular}{l  l}
	30. August & Project start \\
	6. October & Pre-delivery of project report \\
	24. November & Final delivery of project report \\
	24. November & Presentation and project demo \\
\end{tabular}
\newline
\textbf{Sprints}

\begin{tabular}{l  l}
	Sprint I & 14. September - 27. September \\
	Sprint II & 5. October - 18. October \\
	Sprint III & 19. October - 1. November  \\
	Sprint IV & 2. November - 15. November \\
\end{tabular}

\subsection{Concrete Project Work Plan}
The two first weeks on the project will be used on planning and pre study.
The project will be divided into 3 sprints, that will last for three weeks. Each sprint will have a total of 525 person-hours.
The last 1.5 week will be used to finish the final report and prepare for the presentation. 

\subsubsection{Work Breakdown Structure}

\includegraphics[scale=0.80]{./planning/img/wbs.png}

\subsubsection{Gantt Daiagram}

\includegraphics[scale=0.48]{./planning/img/gantt.png}

\section{Project Organization}

\subsection{Project Organization}

\includegraphics[scale=0.45]{./planning/img/organization.png}

\begin{tabular}{|l | p{10cm}|}
	\hline
	\textbf{Role name} & \textbf{Main responsibilities}  \\ \hline
	Project manager & Responsible for having an overview of the project, delegating jobs and resolving conflicts. \\ \hline
	Advisor contact & Responsible for distributing information between the group and the advisor. \\ \hline
	Organizer & Responsible for setting up and informing the group about the meeting schedule. \\ \hline
	Document master & Responsible for document quality and quantity. \\ \hline
	System architect & The lead designer of the system. \\ \hline
	Lead programmer & Makes sure everyone follows the agreed code standards and ensures the quality of the code. \\ \hline
	Customer contact & Responsible for distributing information between the group and the customer. \\ \hline
	Technically responsible & Finds good technical solutions and makes sure that the essential tools are operative. \\ \hline
	Technology evangelist & Brings in ideas about new technologies and tools. \\ \hline
	Scrum master & Responsible for scrum meetings. \\ \hline
	Lead tester & Responsible for good test coverage, both unit and end to end, and to ensure the quality of those tests. \\ \hline
	Secretary & Takes note from each meetings and stores it in the cloud. Responsible for preparing minutes for advisor/customer. \\ \hline

\end{tabular}

\subsection{Partners}

\subsubsection{Customers}
The customer of this project is Thales Norway AS, which are located at Lerkendal Stadium, Strindv 1, 7030 Trondheim.
The table below show the customer contacts of this project.

\begin{tabular}{|l |l |l |}
	\hline
	\textbf{Name} & \textbf{Mobile} & \textbf{E-mail}  \\ \hline
	Christian Tellefsen & 959 98 765 & christian.telefsen@thalesgroup.com \\ \hline
	Stig Bjørlykke & 982 29 806 & stig.bjorlykke@thalesgroup.com \\ \hline
\end{tabular}


\subsubsection{Developers}
This group consist of 7 members.

\begin{tabular}{|l |l |l |}
	\hline
	\textbf{Name} & \textbf{Mobile} & \textbf{E-mail}  \\ \hline
	Terje Snarby & 915 27 390 & snarby@stud.ntnu.no \\ \hline
	Even Wiik Thomassen & 991 61 929 & evenwiik@stud.ntnu.no \\ \hline
	Sondre Johan Mannsverk & 948 15 506 & sondrejo@stud.ntnu.no \\ \hline
	Erik Bergersen & 917 48 305 & eribe@stud.ntnu.no \\ \hline
	Lars Solvoll Tønder & 976 00 317 & larssot@stud.ntnu.no \\ \hline
	Sigrud Wien & 472 54 625 & sigurdw@stud.ntnu.no \\ \hline
	Jaroslav Fibichr & 451 26 314 & jaroslaf@stud.ntnu.no \\ \hline
\end{tabular}

\subsubsection{Advisors}
Advisors are listed below.

\begin{tabular}{|l |l |l |}
	\hline
	\textbf{Name} & \textbf{Mobile} & \textbf{E-mail}  \\ \hline
	Daniela Soares Cruzes & 942 49 891 & dcruzes@idi.ntnu.no \\ \hline
	Maria Carolina Mello Passos & & mariacm@idi.ntnu.no \\ \hline
\end{tabular}

\section{Quality Assurance}

\subsection{Time of response}

\subsubsection{Customer Interaction}
MISSING!!!

\subsubsection{Advisor interaction}
All written documents must be provided to Daniela before 14:00 the day before advisor meeting. This includes meeting agenda, status report and table of reported effort data. The group is responsible for booking a meeting room for all the advisor meetings. Minutes from the advisor meeting has to be delivered for approval the day after before 12:00. The weekly advisor meeting is at 10:30 every Friday. 

\subsubsection{Project Meetings}
We have agreed to have three weekly meetings
\begin{itemize}
	\item Monday 12-14
	\item Wednesday 12-17
	\item Friday 10-13
\end{itemize}

\subsection{Routines for ensuring quality internally}
We will organize in pairs when producing items, where the pair reviews each others work. This is to try to find more errors and to get some extra perspective on style and solutions.

We also has assigned quality assurance responsibilities for three subjects: documents, code and tests. The respective team members will try to have a birds eye overview in their area to catch further errors.

\subsection{Phase result approval}
To ensure the quality of deliverable results, we have have, the group members responsible for quality assurance for a given subject will either read through, or delegate that that task to another free group member.

We will also present the results for the customer. They will then have an opportunity to point out problems and misunderstandings, and suggest solutions. We will then try to correct the problems and the reiterate the quality assurance.

\subsection{Procedures for customer meetings}
All customer meetings should be called with time, place, agenda specified. All background documents relevant to the meetings should also be supplied. This is to ensure efficient and effective meetings.

All customer meetings should be summarised in a document (minute). This document should include:
\begin{itemize}
	\item Time of meeting
	\item Place
	\item Meeting responsible
	\item Names of the attendees
	\item Decisions
	\item Actions
	\item Clarifications
	\item A rough timeline of the above
\end{itemize}

This summary should be done and sent to the customer by 12:00 the day after the meeting. If the customer does not approve the minutes, the minutes should again be corrected by and sent 12:00 the following day.

The customer contact is responsible for the above tasks.

\subsection{Procedures for advisor meetings}
A meeting with the advisor should be called before 12:00 the day before the meeting, and this calling should include:
\begin{itemize}
	\item Time
	\item Place
	\item Agenda
	\item Relevant documents
	\item Minutes for the last meeting
\end{itemize}
The meetings should be summarized. The summary should include:
\begin{itemize}
	\item Time of meeting
	\item Place
	\item Names of the attendees
	\item Decisions
	\item Actions
	\item Clarifications
	\item A rough timeline of the above
\end{itemize}
The minutes should be written and sent before to the advisor for approval before 12:00 the day after the meeting. If the advisor should reject the minutes, they should be corrected and resent 12:00 the day following the rejection.

\subsection{Document templates and Standards}

\textbf{The group has created templates for:}
\begin{itemize}
	\item Meeting agenda
	\item Status report
	\item Meeting minutes
\end{itemize}
\textbf{Standard for organizing files:} \newline
We use GitHub and Google Docs to store the files included in this project. The location of a file is dependent on what type the file is. 
\begin{itemize}
	\item All source code is to be saved in the GitHub repository under source/. This ensures that all the team members have the current version of the code.
		\begin{verbatim}
CSjark/
    csjark/  -- todays source/, for source code
        test/  -- for unit tests
        etc/  -- for configuration files
        headers/  -- header files used to test the program
    bin/  -- file for executing our program
    docs/  -- for CSjark-specific documentation
    utils/  -- for cpp.exe and fake header files
		\end{verbatim}
	\item All textual documents that are completed will be put in the docs/ folder.
	\item All LaTeX documents are stored in the Github repository under report/.
\end{itemize}
\textbf{Standard for naming files:} \newline
The file name should consist of the name of the document(meeting minutes, agenda, phasedoc, e.g.) and the date, if applicable. \newline
\textbf{Standard for coding style:} \newline
The programming language used to implement the utility specified by the customer requirements are Python. The coding style the team has agreed upon is the Python Standard Styling Guide as defined by PEP8(LINK) (Python Enhancement Proposals \#8). In addition the design should attempt to be pythonic, as detailed by PEP20(LINK).

\subsection{Version Control Procedures}
Every relevant digital item should be pushed to our repository at github, and checked out by other participants. Those who works on a given item should commit and push their changes often, so that others can be as up to date as possible. All digital items should be labeled with a version number, starting at version 1. If an item goes under review and is deemed insufficient by the customer or  , the version number should also be incremented by one for each revision of the document

Relevant digital items includes source code, documents, picture files, binary blobs, etc.

NB: Google docs is not to be used for version control, so every document written there should also be pushed to git hub.

\subsection{Internal Reports}
Some of the groups internal activities should be documented. This includes:
\begin{itemize}
	\item Activities, what is done, and what remains
	\item Minutes for internal meetings
	\item Milestones, done/not done.
	\item Person-hours
\end{itemize}
These reports should follow the templates specified in Templates and Standards (A4) if applicable.

\section{Risk Management}

\textbf{R1. Choosing an incompatible technical solution} - The team decides to use a technical solution that is not suited for the given problem, or decides on an implementation that is too time consuming. \newline
\textbf{R2. Too much focus on report} -  We spend too much time working on the report and neglect the implementation. \newline
\textbf{R3. Too much focus on implementation} - We spend too much time working on the implementation and neglect writing all the needed documentation for the report. \newline
\textbf{R4. Illness/Absence} - Members of the team become ill or are otherwise unavailable. \newline
\textbf{R5. Conflicts within group} - Internal conflicts which are destructive to the groups ability to work together. \newline
\textbf{R6. Lack of technical competence} - The team lack the needed technical ability to solve the given problem. \newline
\textbf{R7. Miscommunication within team} - Team members don’t know what to do, or misunderstands the task given to them. \newline
\textbf{R8. Miscommunication with customer} - The team misunderstands the requirements given by the customer. \newline
\textbf{R9. Lack of experience with Scrum} - The team does not have any experience in doing Scrum projects. \newline



%=============================
\chapter{Preliminary Study}
%============================
This chapter presents the preliminary study for this project.
In \autoref{sec:pre:similar} we have examined existing solutions, and in
\autoref{sec:pre:method} we provide a description of two popular software
development methodologies.

\Gls{wireshark}, which our \gls{utility} should create \glspl{dissector} for, are described in
\autoref{sec:pre:wireshark}.
\hyperref[sec:pre:langs]{Section \ref*{sec:pre:langs}} contains the different
programming languages we might use, while \autoref{sec:pre:parser}
describes possible solutions for parsing \Gls{c} \gls{header} files.
\hyperref[sec:pre:config]{Section \ref*{sec:pre:config}} outlines possible
configuration libraries, and \autoref{sec:pre:testing} discusses possible unit
testing frameworks. In \autoref{sec:pre:docs} we describe tools for creating
user documentation and \autoref{sec:pre:ide} describes three integrated
development environments.

\hyperref[sec:pre:eval]{Section \ref*{sec:pre:eval}} provides the
justifications for the choices we have made, and in
\autoref{sec:pre:framework} we describe the framework our \gls{utility} will require.
At the end of the chapter, in \autoref{sec:pre:license}, we describe the
license of our \gls{utility}.


%--------------------------
\section{Similar Solutions}
%--------------------------
\label{sec:pre:similar}
We started by searching for existing solutions in the problem space. This
search turned up idl2wrs\footnote{\url{http://wiki.wireshark.org/idl2wrs}}.
The other solution was suggested by our customer,
Asn2wrs\footnote{\url{http://wiki.wireshark.org/Asn2wrs}}.
Both of these solutions are bundled with \Gls{wireshark}.

\subsection{idl2wrs}
%-------------------
A tool for generating \Gls{wireshark} \glspl{dissector} from \gls{idl} files. The tool is written
in \Gls{python}, and generates \glspl{dissector} in \Gls{c} from \gls{idl} specifications. \gls{idl}
 is used as an interface to enable communication between software of different languages, in a 
language-neutral way. It is used for example in \gls{Sun RPC} and \gls{acorba}. 
Since idl2wrs takes input in a different language than our \gls{utility} will, and creates \glspl{dissector}
in a different language than our \gls{utility}, we can not reuse any of its code. Instead we will look
at its architecture and data structures, especially how it generates \glspl{dissector}.

\subsection{Asn2wrs}
%-------------------
Is a tool for generating \Gls{wireshark} \glspl{dissector} from \gls{asnone} \glspl{protocol}. Asn2wrs requires
four input files: an \gls{asnone} \gls{protocol} description, a configuration file and two
template files. Advantages of using Asn2wrs are faster development because of
easier recompilation, and plugins that are easy to distribute. The disadvantage
is that code and \glspl{makefile} are more complex.\cite{asn2wrs} 

Our customer cannot use this solution as it would require them to rewrite 
their \Gls{c} \glspl{struct} to \gls{asnone} descriptions, which would take a 
very long time. But the team can use the asn2wrs code as an example of how to 
create dissectors for \Gls{wireshark}.

%-----------------------------------------
\section{Software Development Methodology}
%-----------------------------------------
\label{sec:pre:method}
In this section we describe two popular software development methodologies,
while \autoref{sec:pre:devchoice} discusses which one we decided to use, and
why.

\subsection{Waterfall}
%---------------------
\label{sec:pre:waterfall}
Waterfall is a software development methodology based on sequential phases.
It consist of the following phases: requirement specification, design,
implementation, integration, testing, deployment, and maintenance. In its pure
form, these phases are non-overlapping and one way only, which means that each
phase must be fully completed before the next can begin. Following the phases are listed in sequentially order.  

\paragraph{Requirement Specification}
Receiving requirements from a customer and then formalising
these into concrete functional and non-functional requirements. These will
again be further broken down into smaller work items that are easy to quantify
in terms of time of use and importance. These metrics may help distinguish
which features are to be prioritised.

\paragraph{Design}
The design is about planning how to implement the features from the
requirement specification. The goal is to make a precise software architecture for the
project that dictates most of the implementation phase. This may include (but
not limited to) making class diagrams, data flow diagrams, state machines, user
interface mock-ups, etc.

\paragraph{Implementation}
Implementing and coding the design made in the design phase on
a component level.

\paragraph{Integration}
Integrating the different components that results from the
implementation phase.

\paragraph{Testing}
Thoroughly test the result of the implementation and
integration. The goal is to find and fix bugs introduced in these
phases.

\paragraph{Deployment}
Delivering the resulting software to the customer. This may include installing the software on their systems. This is also the phase where the customer either accepts or rejects the resulting software.

\paragraph{Maintenance}
Large software projects are almost impossible to make completely bug free, and
therefore a certain amount of maintenance may be required. The obvious tasks
are to either fix or provide viable workarounds for problems that appear during
normal use. Maintenance may also include developing new features that the
customer finds the need for.

\subsection{\Gls{scrum}}
%-----------------
\label{sec:pre:scrum}
\Gls{scrum} is an agile development methodology based on the philosophy that it is
impossible to completely and accurately plan everything in a software project
before you begin. It is therefore more or less based on iterations of the
waterfall phases described in \autoref{sec:pre:waterfall}, but instead of
having these phases being strictly sequential, they are run in a more
'as needed' basis. Each iteration in \Gls{scrum} is called a sprint and typically
lasts between two and four weeks. This time period is fixed for each
project, so the sprint will always end on time. To make this possible, features
that are not completed on time is deferred to a later sprint. Each sprint
should result in a runnable product that potentially could deliver some value
to the customer, even if this requires some redundant work.

\subsubsection{Main \Gls{scrum} Roles}
\begin{description}
	\item[\Gls{scrum} Master] has the responsibility of maintaining the process and
		for removing obstacles for other team members. In short, the \Gls{scrum}
		master tries to keep the other team members focused on their tasks.
	\item[Product Owner] represents and speaks for the customer. Not
		necessarily a part of the customer's organization, but must have the
		stated authorities.
	\item[Team Members] are responsible for creating and delivering the product.
		Should consist of a self organizing team of five to nine persons with
		a cross functional skill set.
\end{description}

\subsubsection{\Gls{scrum} Artifacts}
\begin{description}
	\item[Product Backlog] contains a high level description of all the desired
		features for the project. These should be prioritised based on their
		business value and evolve along with the project.
	\item[Sprint Backlog] contains what the team is committed to complete over
		the next sprint. These commitments are features broken down into work
		items. These items should not be larger than 16 hours of work, and they
		should be described so that everyone in the team could contribute to
		implementing them.
	\item[Burn Down Chart] A daily updated chart consisting of what work
		remains in the sprint. Its purpose is both to show what work to do next
		and to give a visual representation of the work progress.
\end{description}

A sprint begins with the sprint planning meeting, which consists of two stages. In the first, the team and the product owner prioritizes the product backlog. In the second, the team discusses what features they can commit to, based on priority, and break these down into work items, which are added to the sprint backlog. This should include giving each item an estimated completion time.

The sprint itself consists of producing what is required for completing work
items, updating the burn down chart, and daily \Gls{scrum} meetings. In these daily
meetings each team member provides a short update of what they did the day
before, what they plan to do today and what problems might be in their way.
These problems should not be discussed in this meeting, but rather dealt with
separately after the meeting, which is the \Gls{scrum} master's responsibility.

At the end of the sprint cycle, the team should hold a \Gls{scrum} review meeting.
In this meeting the team should discuss what was completed and what was not, and
demonstrate the completed features for the customer.	

After the review meeting, a separate retrospect meeting should be held with all
the team members, where all members share their reflections of how the sprint
went and on how we could improve for the next sprint. This is important for improving the
process.


%------------------
\section{\Gls{wireshark}}
%------------------
\label{sec:pre:wireshark}
\begin{wrapfigure}{r}{3.5cm}
	\vspace{-10pt}
	\includegraphics[width=3.5cm]{./planning/img/wireshark_logo}
	\vspace{-20pt}
\end{wrapfigure}
\Gls{wireshark}\footnote{\url{http://www.wireshark.org/}} is a free, open source
network \gls{protocol} analyzer. It lets you capture and browse traffic running
through a computer network. \Gls{wireshark} is currently being developed by the
\Gls{wireshark} team, a group of networking experts spanning the globe.\cite{WiresharkORG} Because of
its rich set of features and ease of use, \Gls{wireshark} is the de facto standard in
many different industries and the educational community. \Gls{wireshark} is able to
dissect and display data from a plethora of different \glspl{protocol}. One of its
strengths lies in the ease of which developers can add their own \glspl{dissector},
\gls{post-dissector} and taps.

\Glspl{dissector} can be written in either \Gls{c} or \Gls{lua}. Most \glspl{dissector} are written in \Gls{c}
for increased speed. \Gls{lua}-scripts are mostly used as prototypes or to process
non time crucial data as they don't need compilation to be used. Our customer
uses \Gls{wireshark} not only to browse through and filter regular networking
traffic, but also for monitoring \gls{ipc} where it is
important to have a tool which can easily be extended to dissect and display
structures and data types unique to the organization.

Our \gls{utility} should read \Gls{c} \gls{header} files and create \Gls{wireshark} \glspl{dissector} written
in \Gls{lua} for \glspl{struct} found in the \gls{header} files.


%------------------------------
\section{Programming Languages}
%------------------------------
\label{sec:pre:langs}
The \glspl{dissector} we have to generate are written in \Gls{lua}, and we have looked at
both \Gls{java} and \Gls{python} programming language for our \gls{utility}. In this section we
describe these different languages. In \autoref{sec:pre:langchoice} we describe
which language we selected and why.

\subsection{\Gls{lua}}
%---------------
\begin{wrapfigure}{r}{2.5cm}
	\vspace{-10pt}
	\includegraphics[width=2.5cm]{./planning/img/lua_logo}
	\vspace{-20pt}
\end{wrapfigure}
\Gls{lua}\footnote{\url{http://www.lua.org/}} is a multi-paradigm, dynamically typed
programming language which is designed to be lightweight so it can easily be
embedded into applications. \Gls{lua} has only a few basic data structures: boolean,
numbers, strings and table. Still \Gls{lua} implements advanced features such as
first-class functions, garbage collection, closures, coroutines and dynamic
module loading. \Gls{lua} was created in 1993 at the Pontifical Catholic University
of Rio de Janeiro, in Brazil.\cite{LuaORG}

The output of our \gls{utility} will be \Gls{wireshark} \glspl{dissector} written in \Gls{lua}. While
\Gls{wireshark} supports \glspl{dissector} written in both \Gls{c} and \Gls{lua}, \Gls{lua} is preferred
because they can be added without recompiling \Gls{wireshark}. This is important since some of Thales customers do not allow recompiled versions of Wireshark. \Gls{lua} \glspl{dissector}
interface with \Gls{wireshark} through a simple \Gls{api}.

\subsection{\Gls{java}}
%----------------
\label{sec:pre:java}
\begin{wrapfigure}{r}{1.3cm}
	\vspace{-30pt}
	\includegraphics[width=1.3cm]{./planning/img/java_logo}
	\vspace{-30pt}
\end{wrapfigure}
\Gls{java}\footnote{\url{http://java.com/}} is an object-oriented, structured,
imperative, statically typed programming language. It was originally developed
by Sun Microsystems, which is now a subsidiary of Oracle Corporation. \Gls{java} was
released in 1995, and it derived much of its syntax from \Gls{c} and \Gls{c++}, but with
fewer low-level facilities. \Gls{java}’s strength are portability, automatic memory
management, security, good documentation and an extensive standard \gls{library}.\cite{JavaCOM}
\Gls{java} has several tools and \glspl{library} of varying quality for creating \glspl{parser},
for example \gls{antlr} and \gls{javacc}. A detailed description of \gls{antlr} can be found in 
\autoref{sec:pre:antlr}.

\subsection{python}
%------------------
\label{sec:pre:python}
\begin{wrapfigure}{r}{2cm}
	\vspace{-20pt}
	\includegraphics[width=2cm]{./planning/img/python_logo}
	\vspace{-20pt}
\end{wrapfigure}
\Gls{python}\footnote{\url{http://www.python.org/}} is a general-purpose,
multi-paradigm, object-oriented, imperative, dynamically typed programming
language. It was created by Guido van Rossum, and is today developed by \Gls{python}
Software Foundation and the \Gls{python} community. \Gls{python}’s strength include
automatic memory management, large and comprehensive standard \gls{library},
portability, powerful but very clear, concise and simple syntax.\cite{PythonORG} There exists
several pure \Gls{python} \glspl{library} for creating \glspl{lexer} and \glspl{parser}, like \gls{aply},
\gls{pycparser} and cppheaderparser. These are described further in
\autoref{sec:pre:parser}.


%-----------------------------------
\section{Parsers Libraries \& Tools}
%-----------------------------------
\label{sec:pre:parser}
This section contains various tools and \glspl{library} we have looked at for solving
the challenge of parsing \Gls{c} \gls{header} files. They range from language-independent
tools like \gls{agcc} and \Gls{clang} to \Gls{python}-only \glspl{library} like \Gls{aply} and \gls{pycparser}. The
justification for the \glspl{library} we selected can be found in
\autoref{sec:pre:parserchoice}.

\subsection{\gls{antlr}}
%-----------------
\label{sec:pre:antlr}
\begin{wrapfigure}{r}{2.5cm}
	\vspace{-20pt}
	\includegraphics[width=2.5cm]{./planning/img/antlr_logo}
	\vspace{-20pt}
\end{wrapfigure}
\gls{antlr}\footnote{\url{http://www.antlr.org/}}, ANother Tool for Language
Recognition, is a compiler toolkit for creating \glspl{lexer} and \glspl{parser} from grammar
files. It can create these compilers for several different target languages,
including \Gls{java} and \Gls{python}. There exists \gls{antlr} grammar files for the challenges
we are facing: parsing \Gls{c}, \Gls{c} \gls{preprocessor} step and parsing \gls{asnone}. These grammars
configure \gls{antlr} to create \Gls{java} \glspl{lexer} and \glspl{parser} which reads and validates
inputted source code files.

\subsection{PLY}
%---------------
\Gls{aply}\footnote{\url{http://www.dabeaz.com/ply/}} is a \Gls{python} alternative to the
popular \gls{lexer} and \gls{parser} compilers lex and yacc. It also comes with a 95\%
completed \Gls{c} \gls{preprocessor} in case we are required to modify the \gls{preprocessor}
for our \gls{utility}. Other special purpose \glspl{parser} like \gls{pycparser} and
cppheaderparser depends on \Gls{aply}. These are described later in this section.

\subsection{\gls{pycparser}}
%---------------------
\label{sec:pre:pycparser}
There are two \Gls{python} \glspl{library} for parsing \Gls{c} with the same name, but different
capitalization, \gls{pycparser}\footnote{\url{http://code.google.com/p/\gls{pycparser}/}}
and PyCParser\footnote{\url{https://github.com/albertz/PyCParser}}. While they
both aim to solve almost the same problem, the first one appears to have better
documentation, is a more mature project and support more of the \Gls{C99}
specification. \gls{pycparser} requires \Gls{aply} to work.

\subsection{cppheaderparser}
%---------------------------
Cppheaderparser\footnote{\url{http://sourceforge.net/projects/cppheaderparser/}}
is a \gls{parser} for \Gls{c++} \gls{header} files written in \Gls{python}. It is an alternative for
\gls{pycparser} in case we need to parse \Gls{c++} files instead of simple \Gls{c} \gls{header} files.
It also depends on \Gls{aply}.

\subsection{GCC}
%---------------
\label{sec:pre:gcc}
\begin{wrapfigure}{r}{1.5cm}
	\vspace{-20pt}
	\includegraphics[width=1.5cm]{./planning/img/gcc_logo}
	\vspace{-20pt}
\end{wrapfigure}
GNU Compiler Collection\footnote{\url{http://gcc.gnu.org/}} (\Gls{agcc}) is a
compiler system which has front ends which parse \Gls{c} and \Gls{c++} code, and is written
in \Gls{c} and \Gls{c++}. It can be used in our \gls{utility} as an external tool which does the
parsing and then outputs an intermediate language representation which we can
parse/search to find the \Gls{c} \gls{struct} definitions. Its drawbacks are a lack of
flexibility if we need to change its behaviour, and we will still need to write
a custom \gls{parser} or use something like \gls{GCC-XML} and an \gls{axml} \gls{parser}.

\subsection{\Gls{clang}}
%-----------------
\label{sec:pre:clang}
\begin{wrapfigure}{r}{2.5cm}
	\vspace{-20pt}
	\includegraphics[width=2.5cm]{./planning/img/llvm_logo}
	\vspace{-20pt}
\end{wrapfigure}
\Gls{clang}\footnote{\url{http://clang.llvm.org/}} is a compiler front end for \Gls{c}, 
\Gls{c++}, \Gls{Objective-C} and \Gls{Objective-C++}, written in \Gls{c++}. \Gls{clang} differ from \Gls{agcc} as it
behaves as a \gls{library} rather than an external tool, but for \Gls{java} we will have to
use it like \Gls{agcc} because there are no \Gls{java}-\Gls{clang} bindings. It supports
outputting the \gls{AST} as \Gls{axml}, which our \gls{utility} then will need to
parse. \Gls{clang} provides bindings for \Gls{python} so there it can be used as a \gls{library},
but its main drawback is, like \Gls{agcc}, a lack of flexibility. \Gls{clang} is a part of
the LLVM toolkit.


%---------------------------------
\section{Configuration Frameworks}
%---------------------------------
\label{sec:pre:config}
This section looks at different configuration frameworks for \Gls{python}. Which we
selected and why is explained in \autoref{sec:pre:configchoice}.

Our \gls{utility} needs a flexible configuration, as some of the information we
shall display does not exist in the files we parse. For example there are no
clear relationship between enumerated values in messages and their names.
These must be provided through a configuration.

\subsection{\gls{asnone}}
%-----------------
\gls{asn1} (\gls{asnone}) for describing data
structures, the standard is defined by \Gls{iso} 8824. \gls{asnone} is used for
representing, encoding, transferring and decoding of data. It provides a
fundamental tool to be used in application, which make it possible to exchange
information in a independent way. There exist two \Gls{python} \glspl{library} for parsing
of  \gls{asnone}, but neither support the 3.x \gls{branch} of \Gls{python}, which we require.

\subsection{\Gls{yaml}}
%----------------
\label{sec:pre:yaml}
\begin{wrapfigure}{r}{3cm}
	\vspace{-30pt}
	\includegraphics[width=3cm]{./planning/img/pyyaml_logo}
	\vspace{-30pt}
\end{wrapfigure}
\Gls{yaml}\footnote{\url{http://yaml.org/}} (\Gls{yaml} Ain't Markup Language) is a \gls{data serialization} format.
It is designed to be easy to read and write for humans.
\Gls{yaml} syntax was designed to be easily mapped to data types common to most
high-level languages. While most programming languages can use \Gls{yaml} for data
serialization, \Gls{yaml} excels in working with those languages that are
fundamentally built around the three basic primitives. These include the new
wave of agile languages such as \Gls{perl}, \Gls{python}, \GLS{php}, \Gls{Ruby}, and \Gls{Javascript}.

PyYAML\footnote{\url{http://pyyaml.org/}} is a \Gls{yaml} \gls{parser} for the \Gls{python}
programming language, and it is available for both the 2.x and 3.x \gls{branch} of
\Gls{python}. It is licensed under the \Gls{mit} license.

\subsection{configparser}
%------------------------
configparser\footnote{\url{http://docs.python.org/py3k/\gls{library}/configparser.html}}
is a \Gls{python} module to manage user-editable configuration files. The
files are organized into sections, and each section can contain name-value
pairs for configuration data. Value interpolation using \Gls{python} formatting
strings is also supported, to build values that depend on one another.

configparser module is a part of \Gls{python} standard \gls{library}, and therefore does
not require installation or configuration to use.

\subsection{ConfigObj}
%---------------------
ConfigObj\footnote{\url{http://www.voidspace.org.uk/python/configobj.html}} is
a simple but powerful config file reader and writer (originally based on
ConfigParser). Its main feature is that it is very easy to use, with a
straightforward programmer's interface and a simple syntax for config files.
Among others, it has these additional features:
\begin{itemize}
	\item Nested sections (subsections), to any level
	\item List values
	\item Multiple line values
	\item String interpolation (substitution)
	\item Integrated with a powerful validation system
\end{itemize}

\noindent Currently, ConfigObj module exists only for \Gls{python} up to version
2.7. It is under \Gls{bsd} license.


%--------------------------------
\section{Unit Testing Frameworks}
%--------------------------------
\label{sec:pre:testing}
There are many different unit testing frameworks for \Gls{python}. We have evaluated
three of them, to see which best suits our \gls{utility}, which we describe in this
section. In \autoref{sec:pre:testchoice} we describe which we selected and why.

\subsection{py.test}
%-------------------
py.test\footnote{\url{http://pytest.org/latest/}} is a mature, full-featured testing
tool. It runs on \Gls{python} 2.4-3.2, PyPy and Jython-2.5.1 intepreters on both
Windows and Posix platforms. It is well documented and popular in the \Gls{python}
community. The best known project which uses it is PyPy, which has over 16,000
unit tests. py.test discoverers tests automatically by searching for modules,
classes, functions and methods which starts with "test\_". It uses the assert
statement to test variables and values. These implicit behaviours make tests
easier and faster to write, but harder to learn and understand.

\subsection{nose}
%----------------
nose\footnote{\url{http://readthedocs.org/docs/nose/en/latest/}} testing
framework extends \Gls{python}'s unittest \gls{library} to make testing easier.
It provides an alternative test discovery and running process for unittest,
which is intended to mimic the behavior of py.test as much as reasonably
possible without resorting to too much magic. nose support easy-to-write
plugins, and it comes bundled with the most popular ones. It supports both
\Gls{python} 2.x and 3.x \glspl{branch}.

\subsection{Attest}
%------------------
\label{sec:pre:attest}
\begin{wrapfigure}{r}{3.2cm}
	\vspace{-20pt}
	\includegraphics[width=3.2cm]{./planning/img/pocoo_logo}
	\vspace{-30pt}
\end{wrapfigure}
Attest\footnote{\url{http://packages.python.org/Attest/}} is a test automation
framework for \Gls{python}, emphasising modern idioms and conventions. It supports
test collecting using \Gls{python} decorators, introspection of the assert statement,
treating tests as \Gls{python} modules rather than scripts. Attest is a rather young
framework, with limited features and documentation. Attest is a sub-level
project of the Pocoo project.

\subsection{coverage.py}
%-----------------------
coverage.py\footnote{\url{http://nedbatchelder.com/code/coverage/}} is a tool
for measuring code coverage of \Gls{python} programs. It is typically used to measure
the effectiveness of unit tests, by showing which parts of the code are
exercised by tests. coverage.py support \Gls{python} 2.3 to 3.2. It can output
results in plain text, \Gls{html} and \Gls{axml}.


%---------------------------------
\section{User Documentation Tools}
%---------------------------------
\label{sec:pre:docs}
Some of the non-functional requirements for our \gls{utility} is user documentation.
In this section we describe a tool for writing such documentation, and a free
hosting site for our user documentation.

\subsection{Sphinx}
%------------------
\begin{wrapfigure}{r}{2.7cm}
	\vspace{-20pt}
	\includegraphics[width=2.7cm]{./planning/img/sphinx_logo}
	\vspace{-20pt}
\end{wrapfigure}
Sphinx\footnote{\url{http://sphinx.pocoo.org/}} is a \Gls{python} tool for writing
documentation, that makes it easy to create intelligent and beautiful
documentation. It is used for the standard \Gls{python} documentation, and it is
popular in the \Gls{python} community. Sphinx uses reStructuredText as its markup
language, which is a easy-to-read, what-you-see-is-what-you-get plain text
markup syntax and \gls{parser} system. Our use case for sphinx is writing
documentation for our \gls{utility}, how to use it and configure it. Sphinx can
generate output in several different formats, including \Gls{html} and latex/pdf.

\subsection{Read the Docs}
%-------------------------
\begin{wrapfigure}{r}{1.2cm}
	\vspace{-20pt}
	\includegraphics[width=1.2cm]{./planning/img/readthedocs_logo}
	\vspace{-20pt}
\end{wrapfigure}
Read the Docs\footnote{\url{http://readthedocs.org/docs/read-the-docs/}} is a
free hosting of documentation for the open source community. It supports Sphinx
docs written with reStructuredText, and it can automatically pull from Git,
Subversion, Bazaar, and Mercurial repositories. We can configure it so it
automatically pulls and compiles our user documentation from our github
repository whenever we push any changes.


%-------------------------------------------
\section{Integrated Development Environment}
%-------------------------------------------
\label{sec:pre:ide}

\subsection{PyCharm}
%-------------------
\begin{wrapfigure}{r}{3.5cm}
	\vspace{-20pt}
	\includegraphics[width=3.5cm]{./planning/img/pycharm_logo}
	\vspace{-20pt}
\end{wrapfigure}
PyCharm\footnote{\url{http://www.jetbrains.com/pycharm/}}
is a cross platform, proprietary \Gls{ide} for \Gls{python}. It has good support
for text editing, syntax highlighting, auto indentation, code navigation, code
completion and automatic error checking. There is also a decent debugger and
unit test support that can help finding errors and integrated version control
support (including git), which makes it easy to synchronize with a remote
repository. Most mentioned functions is also paired with keyboard shortcuts.

The downside with PyCharm is that it requires a relative expensive license. It
is, however, possible to apply for classroom licenses that are free of charge.
The latter is a requirement to make this \Gls{ide} a viable option.

\subsection{PyScripter}
%----------------------
\begin{wrapfigure}{r}{3cm}
	\vspace{-20pt}
	\includegraphics[width=3cm]{./planning/img/pyscripter_logo}
	\vspace{-20pt}
\end{wrapfigure}
PyScripter\footnote{\url{http://code.google.com/p/pyscripter/}}
is a Windows only, open source \Gls{ide} for \Gls{python}. It has support for
basic text editing functions relevant to programming like syntax highlighting,
auto indentation, code completion, debugger and file management. It also has
some support for navigating the code, for example by offering to find the next
point in the code that references a certain variable or function. The mentioned
function mostly has keyboard shortcuts.

It does not have support for automatic error checking in the program, so it
will not alert the user of spelling and syntax errors. It also lacks integration
with any version control systems like git or svn. The code completion and
code navigation is a little lacking. It will for example not suggest importing
files if you reference a class from another module and it cannot give a
complete list of usages of a function.

\subsection{\gls{vim}}
%---------------
\begin{wrapfigure}{r}{1.5cm}
	\vspace{-20pt}
	\includegraphics[width=1.5cm]{./planning/img/vim_logo}
	\vspace{-20pt}
\end{wrapfigure}
\gls{vim}\footnote{\url{http://www.vim.org/}} is cross-platform, open source text
editor originally created for the Amiga. It is not regarded as an \Gls{ide}, but it
provides all the regular features of text editors, including syntax
highlighting, auto-completion, auto-indentation, searching, multiple undo and
redo. It can be configured to support almost everything modern \Gls{ide}’s support,
and its extensive customizability is considered parts of its strength. But it
is also parts of its weakness, it is very difficult for new \gls{vim} users to learn
how to use it effectively. Therefore we do not suggest any team member which is
not already experienced with \gls{vim} to try it.

\subsection{Summary}
%-------------------
PyCharm is by far the best \Gls{ide} evaluated in terms of functionality, and it is
the one that mirrors \Gls{Eclipse} the most, which is an advantage, since most team
members are best acquainted with \Gls{Eclipse}. It will be the recommended \Gls{ide} for
this project, given that we can acquire classroom licenses.

On the other hand, there is no real reason to dictate the use of \Gls{ide}, since
what determines the productivity of a team member is more how well you know
the specific tool you are using. It will therefore be up to each team member
to choose what \Gls{ide}/text editor they want to use.


%----------------------------------
\section{Evaluation and Conclusion}
%----------------------------------
\label{sec:pre:eval}
In this section we provide a justification for the choices we have made in
regards to process, programming language, and \glspl{library} we will use. Then
in \autoref{sec:pre:framework} we give a brief description of the framework
we will construct for our \gls{utility}.

\subsection{Development Process Choice}
%--------------------------------------
\label{sec:pre:devchoice}
We have chosen \hyperref[sec:pre:scrum]{\Gls{scrum}} for our development strategy.
We do not have a lot of experience with software development either
individually or as a team, so we have little personal knowledge of how much
we are able to produce, and the task may present challenges that we are not
prepared for when the project starts. For these reasons we believe that we
need to take an agile approach to this project. This way, we may both learn as
we go, and adjust later iterations by the result of the previous. We may also
have something to deliver even if we do not have time to implement all the
desired features.

The \Gls{scrum} methodology fits these goals perfectly, and is therefore a natural
choice. The risk factor here is that all team members are mostly unfamiliar
with \Gls{scrum}, while we have at least a little knowledge of waterfall. We do,
however, think that the time and risk of learning will not outweigh
the benefit it will give us over waterfall.

\subsection{Programming Language Choice}
%---------------------------------------
\label{sec:pre:langchoice}
We originally selected \hyperref[sec:pre:java]{\Gls{java}} as our programming
language because it would run on all the platforms required, it offered
automatic memory management so it would be easier to debug, and it was the only
language everyone on the team had experience with.

We looked at \hyperref[sec:pre:antlr]{\gls{antlr}} for generating a
\Gls{c} \gls{lexer} and \gls{parser} in \Gls{java}, which looked very promising. It also provided
grammar files for creating a \Gls{c} \gls{preprocessor} in \Gls{java}. Closer evaluation revealed
that the \Gls{c} \gls{preprocessor} grammar was written in 2006, and had stopped working in
2008 as newer versions of \gls{antlr} was not backwards compatible. Also the
generated \Gls{c} \gls{parser} only validated \Gls{c} code, it did not create an \gls{AST} which we could traverse.
This meant that using \Gls{java} and \gls{antlr} would
require us to modify these grammars to suite our needs, and \gls{antlr}'s lack of
documentation became a significant risk for our project.

These issues and feedback from our customer made us evaluate \Gls{python} for
developing our \gls{utility}. We found several \glspl{library} for parsing \Gls{c} files, and
even one for parsing \Gls{c++} \gls{header} files. These are described in
\autoref{sec:pre:parser}.

We decided to use \hyperref[sec:pre:python]{\Gls{python}} for this project because
the parsing \glspl{library} for \Gls{python} came in working condition with decent
documentation, and because we were able to create a small working prototype
in \Gls{python} in just a few hours. We estimate that it would take at least a week
to achieve the same result in \Gls{java}.

A challenge with our decision is the fact that not all team members have
sufficient experience with \Gls{python}. Most team members must therefore do some
self study before we start the first sprint.

\subsection{Parsers Libraries \& Tools Choice}
%---------------------------------------------
\label{sec:pre:parserchoice}
We outlined three different approaches for parsing \Gls{c} \gls{header} files. The first approach is to
write a custom \gls{parser} ourself, the second is to use a \Gls{c} parsing \gls{library}, and the
third is to use a toolkit \gls{parser} like \hyperref[sec:pre:gcc]{\Gls{agcc}} and
\hyperref[sec:pre:clang]{\Gls{clang}}.

We felt that writing our own \Gls{c} \gls{parser} with \Gls{c} \gls{preprocessor} would possibly take
up a lot, if not all, of the available project time. The third option would add
a large dependency which our customer want to avoid if possible. \Gls{agcc} and \Gls{clang}
can be challenging to install and use on Windows.

Therefore using a \Gls{c} \gls{parser} \gls{library} would be the best solution, and as mentioned
above, \Gls{java} with \gls{antlr} proved challenging. So we evaluated \Gls{python} \gls{parser}
\glspl{library}.

We decided to use \hyperref[sec:pre:pycparser]{\gls{pycparser}}. We favored \gls{pycparser}
over PyCParser and cppheaderparser because it has better documentation, it
seemed to be a more mature project, and it supports the most of the \Gls{C99}
specification. \gls{pycparser} depends on \Gls{aply}, so our \gls{utility} will also depend on it.

For \Gls{c} \gls{preprocessor} we have selected to use a tool for Windows which comes with
\gls{pycparser}, on Mac we will use the one which comes with XCode, and on other
platforms we will either use \Gls{agcc} or tools which comes with the platform. If we
need to modify a \Gls{c} \gls{preprocessor} we might use \Gls{aply}'s incomplete \Gls{c} \gls{preprocessor}.

\subsection{Configuration Framework Choice}
%------------------------------------------
\label{sec:pre:configchoice}
We have listed a summary of some of the advantages and drawbacks of the
different configuration frameworks we looked at in \autoref{tab:pre:config}.
\begin{table}[htbp] \footnotesize \center
\caption{Configuration summary\label{tab:pre:config}}
\noindent\makebox[\textwidth]{%
\begin{tabularx}{1.1\textwidth}{X X X X X}
	\toprule
	& \Gls{yaml} & configparser & ConfigObj & \gls{asnone} \\
	\midrule
	Advantages &
	+Simplicity\newline +Flexibility &
	+Easy to use &
	+Easy to use\newline +Flexibility\newline +Nesting\newline +Type \newline\hspace*{3mm}validation &
	+Customer \hspace*{3mm}wants it \\
	\addlinespace
	Drawbacks &
	-External \gls{library}\newline -No type \newline\hspace*{2mm}validation &
	-Lacks nesting\newline -Lacks lists\newline -No type \newline\hspace*{2mm}validation &
	-External \gls{library}\newline -Lacks lists &
	-External \gls{library}\newline -Too generic \\
	\addlinespace
	Latest version & 3.10 & 3.2 & 4.7.2 & 0.0.13 \\
	\Gls{python} \gls{branch} & 2.7 and 3.3 & 2.7 and 3.3 & 2.7 & 2.7 \\
	License & \Gls{mit} & PSF L & \Gls{bsd}-new & \Gls{bsd} \\
	\bottomrule
\end{tabularx}}
\end{table}

We decided to use \hyperref[sec:pre:yaml]{YAML} for handling configuration
files, as it covered most of our requirements. Because we decided to use the
latest version of \Gls{python}, version 3.2.2, the range of possible configuration
frameworks was reduced. Therefore, although \gls{asnone} and ConfigObj are very
suitable for our task, they were eliminated (those \glspl{parser} are available
only up to version 2.7), which left us 2 main possibilities: \Gls{yaml} and
configparser. configparser turned out to be insufficient for us, mainly because
it lacked lists. These we need for description of hierarchical structures of
the \Gls{c} \glspl{header}. \Gls{yaml} has only two minor disadvantages we should be aware of.
First, there is no type validation mechanism, so we will have to create our
validation manually. Second, it is an external \gls{library}. We find this drawback
minor for now but it can turn out to be a problem in the future. Except for
these issues, \Gls{yaml}, more specifically pyYAML, seems to have a good potential
for creating flexible configuration support for our \gls{utility}.

\subsection{Unit Testing Framework Choice}
%-----------------------------------------
\label{sec:pre:testchoice}
The three frameworks we looked at  are very similar, being modern \Gls{python}
testing frameworks. They differ in maturity and what is often called magic in
the \Gls{python} community.

py.test is the most mature but also the most magic, it uses a lot of
introspection to discover tests and it has no \Gls{api}. nose is heavily influenced
by py.test, but it tries to be more explicit, and provides an \Gls{api}. Attest is
the youngest testing framework, and like nose, has less magic and focuses on
providing a very pythonic \Gls{api}. Being the youngest also means it has the least
documentation, functionality and plugins. Therefore Attest might be the easiest
testing framework to learn. Therefor we decided to use
\hyperref[sec:pre:attest]{Attest} for unit testing of our \gls{utility}.

\subsection{Our Framework}
%-------------------------
\label{sec:pre:framework}
Our \gls{utility} will need to take as input \Gls{c} \gls{header} files, search through them to
find \gls{struct} definitions, and create \Gls{lua} scripts which dissects the \glspl{struct} in
\hyperref[sec:pre:wireshark]{Wireshark}.

To find the \glspl{struct} we will use \gls{pycparser} to parse the input files, create an
\gls{AST}, and to find the \gls{struct} definitions. We will use pyYAML to
read configuration from file, which together with the \gls{struct} definitions will
be placed in some suitable data structures for generating \glspl{dissector}.

The versions of the different tools and \glspl{library} we are using can be found in
\autoref{tab:pre:versions}.
\begin{table}[!h] \footnotesize \center
\vspace{-10pt}
\caption{Versions of tools and \glspl{library}\label{tab:pre:versions}}
\begin{tabular}{l l l}
	\toprule
	Library/Tool & Version & Why \\
	\midrule
	\Gls{python} & CPython 3.2.2 & Latest stable standard \Gls{python} implementation \\
	\gls{pycparser} & 2.05-dev & Development version, for \_Bool support \\
	pyYAML & 3.10 & Latest stable version \\
	\Gls{aply} & 3.4 & Latest stable version \\
	Attest & 0.6-dev & Development version, for \Gls{python} 3.2 support \\
	Sphinx & 1.0.8 & Latest stable version \\
	WireShark & 1.7.0-SVN & Latest nightly build, for \Gls{lua} support \\
	\bottomrule
\end{tabular}
\vspace{-10pt}
\end{table}


%-----------------------------
\section{IP Rights \& License}
%-----------------------------
\label{sec:pre:license}
The customer has explained that they do not intend to distribute our \gls{utility},
and that we are free to license it as open source if we want to, under
whichever license we feel is most suited. They suggested GNU\footnote{\url{http://www.gnu.org/}} \Gls{gpl} as \Gls{wireshark}
is released under it.

We needed to consider the licenses of the \glspl{library} and tools we depend upon
when we decided which license to use. This is summarized by
\autoref{tab:pre:licenses}.
\begin{table}[!h] \footnotesize \center
\vspace{-20pt}
\caption{Licenses\label{tab:pre:licenses}}
\begin{tabular}{l l}
	\toprule
	\Gls{wireshark} & GNU \Gls{gpl} \\
	\Gls{aply} & \Gls{bsd}-new \\
	\gls{pycparser} & \Gls{bsd}-new \\
	pyYAML & \Gls{mit} \\
	\midrule
	Our \gls{utility} & \Gls{bsd}-new \\
	\bottomrule
\end{tabular}
\vspace{-10pt}
\end{table}

\noindent Some of the requirements for our \gls{utility} might require us to modify
the \Gls{c} \gls{preprocessor} in \Gls{aply} and the \gls{pycparser} \gls{library}, which made us consider
the new 2-clause \Gls{bsd} license the most suited for us. Since it also gives us the
option to later move to a more restrictive license like \Gls{gpl}, we decided to use
it.

%=====================
\chapter{Requirements}
%=====================
%\begin{comment}
\label{chap:requirements}
This chapter describes an utility that creates Wireshark dissectors from C
header files. The dissectors must interpret binary representations of C
structs. In \autoref{sec:reqsoverview} we give a high level overview of the
utility and its requirements, \autoref{sec:usecases} provides use cases for the
utility, \autoref{sec:reqspriority} explains how we prioritizes the
requirements, \autoref{sec:reqscompl} explains how we estimate their
complexity, \autoref{sec:reqslist} lists all the functional and non-function
requirements, and \autoref{sec:prodbacklog} contains the complete product
backlog.

\section{Overview}
%-----------------
\label{sec:reqsoverview}
We are to create an utility that allows Wireshark to interpret the binary
representations of C-language structs. While C structs seldom are exchanged
across networks, they are sometimes used in inter-process communication. The
purpose of the utility described here is to provide Wireshark with the
capability of automatically dissecting the binary representation of a C struct,
as long as its definition is known.

The expected work flow for the utility is to read one or more C header files,
which contain struct definitions, and output Wireshark dissectors, implemented
in Lua scripts. A configuration file or source code annotations in the header
files may be used when additional configuration is required.

\subsection{List of requirements}
\autoref{tab:reqsoverview} is an overview of all the requirements. See
\autoref{tab:funcreq} and \autoref{tab:nonfuncreq} for more detailed
description of the requirements.

\begin{table}[H] \small \center
\caption{Requirements overview\label{tab:reqsoverview}}
\begin{tabular}{l l}
	\toprule ID & Description \\ \midrule
	FR1 & Read basic C struct definitions \\
	FR1-A & Support data types: int, float, char and boolean \\
	FR1-B & Support members of type enums \\
	FR1-C & Support members of type structs \\
	FR1-D & Support members of type unions \\
	FR1-E & Support member of type array \\
	\addlinespace
	FR2 & Generate Wireshark dissectors in Lua \\
	FR2-A & Display simple structs \\
	FR2-B & Support display of structs within structs \\
	FR2-C & Support Wireshark filter and search on attributes \\
	FR2-D & Recognize invalid values for a struct member \\
	\addlinespace
	FR3 & Support C preprocessor directives and macros \\
	FR3-A & Support \#include \\
	FR3-B & Support \#define and \#if \\
	FR3-C & Support \verb+WIN32+, \verb+_WIN64+, \verb+__sparc+ etc \\
	\addlinespace
	FR4 & Support user configuration \\
	FR4-A & Support valid ranges for struct members \\
	FR4-B & Support enumerated named values or a bit strings \\
	FR4-C & Custom handling of specific data types \\
	\addlinespace
	FR5 & Structs with headers and/or trailers \\
	\addlinespace
	FR6 & Handle input which size and endian depends on platform \\
	FR6-A & Flags specified for each platform \\
	FR6-B & Flags which signal the platform \\
	\addlinespace
	FR7 & Support parameters from command line \\
	FR7-A & Support parameters for c-header file \\
	FR7-B & Support for configuration file \\
	FR7-C & Support batch mode of c-header and configuration \\
	FR7-D & Don't regenerate dissectors \\
	\addlinespace
	NR1 & Run on latest Windows \& Solaris OS \\
	NR2 & Dissector run on Windows \& Solaris, Intel \& Sparc \\
	NR3 & User interface shall be command line \\
	NR4 & Sufficient documentation for generating Lua-scripts \\
	NR5 & Sufficient documentation for extending functionality \\
	NR6 & Code should follow PEP8 and PEP20 \\
	NR7 & Code should be documented by docstrings \\
	\bottomrule
\end{tabular}
\end{table}

\section{Use Cases}
%------------------
\label{sec:usecases}

\subsection{Actors}
An actor specifies a role played by an external person or thing that interact
with our utility. We have three types of actors to consider. First is the
primary actor which in our case is the user of our utility. He who feeds it a
C file to generate dissectors. A secondary actor is someone who configures our
utility to change the output of it. Finally we have an offstage actor which
does not use our utility himself, but uses the output dissectors in Wireshark.

We have defined two use case actors for our utility. The customer has
specified that the user is the most important actor.
\begin{description}
	\item[User] User of the generated Wireshark dissectors, offstage actor
	\item[Developer] User and configurer of utility, primary and secondary
		actor
\end{description}

\subsection{Use Case Diagrams}
TODO!! Desperately need help for this....

\section{Prioritization}
%-----------------------
\label{sec:reqspriority}
The team has, in cooperation with the customer, prioritized the requirements
in three categories:
\begin{inparaenum}[\itshape a\upshape)]
	\item High,
	\item Medium or
	\item Low.
\end{inparaenum} 

\begin{description}
	\item[High] Core functionality of the utility which must be implemented.
	\item[Medium] Requirements that will improve the value of the utility.
	\item[Low] Requirements that will not add much value to the utility.
\end{description}

\section{Complexity}
%-------------------
\label{sec:reqscompl}
The team has estimated the complexity for each requirement. We use the same
categories as for requirements priority:
\begin{inparaenum}[\itshape a\upshape)]
	\item High,
	\item Medium or
	\item Low.
\end{inparaenum} 

\begin{description}
	\item[High] Functionality which seems difficult and non-trivial to create.
	\item[Medium] Functionality that seems time consuming but straight forward.
	\item[Low] Requirements that are trivial to implement.
\end{description}
%\end{comment}

\section{List of requirements}
%-----------------------------
\label{sec:reqslist}
\autoref{tab:funcreq} lists the functional requirements, while
\autoref{tab:nonfuncreq} lists the non-functional requirements. Each
requirement have a priority (Pri) and a complexity (Cmp): High (H), 
Medium (M) or Low (L).

\begin{table}[ht] \footnotesize \center
\caption{Functional Requirements\label{tab:funcreq}}
\noindent\makebox[\textwidth]{%
\begin{tabularx}{1.2\textwidth}{l X c c}
	\toprule
	ID & Description & Pri. & Cmp. \\
	\midrule
	FR1 & The utility must be able to read basic C language struct definitions from C header files & H & \\
	FR1-A & The utility must support the following basic data types: int, float, char and boolean & H & L \\
	FR1-B & The utility must support members of type enums & H & L \\
	FR1-C & The utility must support members of type structs & H & M \\
	FR1-D & The utility must support members of type unions & M & M \\
	FR1-E & The utility must support member of type array & H & M \\
	\midrule
	FR2 & The utility must be able to generate lua-script for Wireshark dissectors for the binary representation of C struct & H & \\
	FR2-A & The dissector shall be able to display simple structs & H & L \\
	FR2-B & The dissector shall be able to support structs within structs & M & M \\
	FR2-C & The dissector must support Wiresharks built-in filter and search on attributes & H & L \\
	FR2-D & The dissector shall be able to recognize invalid values for a struct member & L & L \\
	\midrule
	FR3 & The utility must support C preprocessor directives and macros & H & \\
	FR3-A & The utility shall support \#include & H & L \\
	FR3-B & The utility shall support \#define and \#if & H & L \\
	FR3-C & The utility shall support \verb+WIN32+, \verb+_WIN32+, \verb+_WIN64+, \verb+__sparc__+, \verb+__sparc+ and \verb+sun+ & M & H \\
	\midrule
	FR4 & The utility must support user configuration & M & \\
	FR4-A & Configuration must support valid ranges for struct members & L & L \\
	FR4-B & Configuration must support integer members which represent enumerated named value or a bit string & M & L \\
	FR4-C & Configuration must support custom handling of specific data types. E.g. a 'time\_t' may be interpreted to contain a unixtime value, and be displayed as a date & L & M \\
	\midrule
	FR5 & A struct may have a header and/or trailer (other registered protocol). The configuration must support the use of integer members to indicate the number of other structs that will follow in the trailer & L & H \\
	\midrule
	FR6 & The dissectors must be able to handle binary input which size and endian depends on originating platform & M & \\
	FR6-A & Flags must be specified for each platform & M & M \\
	FR6-B & Flags within message headers should signal the platform & M & H \\
	\midrule
	FR7 & The utility shall support parameters from command line & H & \\
	FR7-A & Command line shall support parameters for c-header file & H & L \\
	FR7-B & Command line shall support for configuration file & H & L \\
	FR7-C & Command line shall support batch mode of c-header and configuration file & L & M \\
	FR7-D & When running batch mode, dissectors that already are generated, shall not be regenerated, if the source are not modified since last run & L & M \\
	\bottomrule
\end{tabularx}}
\end{table}

\begin{table}[ht] \footnotesize \center
\caption{Non-Functional Requirements\label{tab:nonfuncreq}}
\noindent\makebox[\textwidth]{%
\begin{tabularx}{1.2\textwidth}{l X c c}
	\toprule
	ID & Description & Pri. & Cmp. \\
	\midrule
	NR1 & The utility shall be able to run on latest Windows and Solaris operating system & M & L \\
	\addlinespace
	NR2 & The dissector shall be able to run on Windows x86, Windows x86-64, Solaris x86, Solaris x86-64 and Solaris SPARC & M & M \\
	\addlinespace
	NR3 & The utilities user interface shall be command line. No clicking! & H & L \\
	\addlinespace
	NR4 & The configuration shall have sufficient documentation to allow a person with no previous knowledge of the system to be able to use it to generate LUA-scripts after X hours of reading & M & M \\
	\addlinespace
	NR5 & The configuration should have sufficient documentation to allow a person, already proficient with the system, to understand the code well enough to be able to extend it’s functionality after Y hours of reading & M & M \\
	\addlinespace
	NR6 & The utility code should follow standard python coding convention as specified by PEP8, and try to follow python style guidelines defined by PEP20 & H & L \\
	\addlinespace
	NR7 & The utilities code should be documented by python docstrings which should explain the use of the code. Python modules, classes, functions and methods should have docstrings & M & L \\
	\bottomrule
\end{tabularx}}
\end{table}

\section{User Stories}
The developer in this context is the developers at Thales Norway AS. \newline
The administrator is this context is the users of the dissectors. \newline
\begin{table}[ht] \footnotesize \center
\caption{User stories}
\noindent\makebox[\textwidth]{%
\begin{tabularx}{1.2\textwidth}{l X}
	\toprule
	Req. & User story \\
	\midrule
	FR01: 	&As a developer, I want that the utility can read basic C language struct definitions from C header files, so that I can generate Lua-scripts for wireshark.\\
	FR01A: 	&As a developer,I want to support structs with basic data types(int, float, char, boolean), so that I can generate Lua-scripts.\\
	FR01B: 	&As a developer, I want to support structs with enums, so that I can generate Lua-scripts.\\
	FR01C: 	&As a developer, I want to support structs with other structs, so that I can generate Lua-scripts.\\
	FR01D: 	&As a developer, I want to support structs with unions, so that I can generate Lua-scripts.\\
	FR01E: 	&As a developer, I want to support structs with arrays, so that I can generate Lua-scripts.\\
	\midrule
	FR02:		&As a developer, I want to generate lua-script for Wireshark dissectors for the binary representation of C struct.\\
	FR02A: 	&As a administrator, I want that the dissector is able to display simple C structs.\\
	FR02B: 	&As a administrator, I want that the dissector is able to display C struct, that contains other structs.\\
	FR02C: 	&As a administrator, I want to use Wiresharks built-in filter and search on attributes, on the dissectors.\\
	FR02D:	&As a administrator, I want see when a there is a invalid value for a struct member\\
	\midrule
	FR03:		&As a developer, I want that the utility supports C preprocessor directives and macros\\
	FR03A:	&As a developer, I want support for \#include in the utility\\
	FR03B:	&As a developer, I want support for \#define and \#if in the utility\\
	FR03C:	&As a developer, I want that the utility supports WIN32, \_WIN32, \_WIN64, \_\_sparc\_\_, \_\_sparc and sun macros.\\
	\midrule
	FR04: 	&As a developer, I want that the utility supports user configuration\\
	FR04A: 	&As a developer, I want to specifiy allowed ranges for the value of a struct member\\
	FR04B:	&As a developer, I want support for integer member which represent enumreated named value or bit string, in configuration\\
	FR04C:	&As a developer, I want support for custom handling of specified data types(e.g. time\_t)\\
	\midrule
	FR05:	&As a developer, I want support in configuration to handle structs with header and/or trailer\\
	\midrule
	FR06:	&As a developer, I want dissectors that are able to handle binary input which size and denian depends on originating platform.\\
	FR06A:	&As a developer, I want specified flags for each platform.\\
	FR06B:	&As a developer, I want that flags within message headers should signal the platform\\
	\midrule
	FR07:	&As a developer, I want to specify parameters from command-line\\
	FR07A:	&As a developer, I want to specify parameters for c-header file from command-line\\
	FR07B:	&As a developer, I want to specify parameters for configuration file from command-line\\
	FR07C:	&As a developer, I want start a batch mode of c-header and configuration file from command-line\\
	FR07D:	&As a developer, I do not want to regenerate dissectors that not are modified since last run, when running batch mode\\
	\bottomrule
\end{tabularx}}
\end{table}


\section{Product backlog}
%---------------------------
\label{sec:prodbacklog}
\autoref{tab:prodbacklog} contains a complete product backlog.

\begin{table}[ht] \small \center
\caption{Product Backlog \label{tab:prodbacklog}}
\begin{tabularx}{\textwidth}{l X c c}
	\toprule
	& & \multicolumn{2}{c}{Hours} \\
	\cmidrule(r){3-4}
	Req. & Description & Est. & Act. \\
	\midrule
	FR1 & Read basic C struct definitions & & \\
	FR1-A & Support data types: int, float, char and boolean & 24 & 21 \\
	FR1-B & Support members of type enums & 6 & - \\
	FR1-C & Support members of type structs & 7 & - \\
	FR1-D & Support members of type unions & - & - \\
	FR1-E & Support member of type array & 7 & - \\
	\addlinespace
	FR2 & Generate Wireshark dissectors in Lua & & \\
	FR2-A & Display simple structs & 28 & 25 \\
	FR2-B & Support display of structs within structs & 11 & - \\
	FR2-C & Support Wireshark filter and search on attributes & - & - \\
	FR2-D & Recognize invalid values for a struct member & 22 & 15 \\
	\addlinespace
	FR3 & Support C preprocessor directives and macros & & \\
	FR3-A & Support \#include & 8 & 2 \\
	FR3-B & Support \#define and \#if & 11 & 3 \\
	FR3-C & Support \verb+WIN32+, \verb+_WIN64+, \verb+__sparc+ etc & - & - \\
	\addlinespace
	FR4 & Support user configuration & & \\
	FR4-A & Support valid ranges for struct members & 30 & 15 \\
	FR4-B & Support enumerated named values or a bit strings & 12 & - \\
	FR4-C & Custom handling of specific data types & - & - \\
	FR4-D & Support specifying the ID of dissectors & 4 & - \\
	FR4-E & Support custom Lua files for specific protocols & 18 & - \\
	\addlinespace
	FR5 & Structs with headers and/or trailers & 18 & - \\
	\addlinespace
	FR6 & Handle input which size and endian depends on platform & & \\
	FR6-A & Flags specified for each platform & - & - \\
	FR6-B & Flags which signal the platform & - & - \\
	FR6-C & Dissectors support both little and big endian & 15 & - \\
	FR6-D & Dissectors support different sizes from flags & - & - \\	
	\addlinespace
	FR7 & Support parameters from command line & & \\
	FR7-A & Support parameters for c-header file & 9 & 9 \\
	FR7-B & Support for configuration file & 28 & 8 \\
	FR7-C & Support batch mode of c-header and configuration & - & - \\
	FR7-D & Don't regenerate dissectors & - & - \\
	\addlinespace
	NR1 & Run on latest Windows \& Solaris OS & - & - \\
	NR2 & Dissector run on Windows \& Solaris, Intel \& Sparc & - & - \\
	NR3 & User interface shall be command line & - & - \\
	NR4 & Sufficient documentation for generating Lua-scripts & - & - \\
	NR5 & Sufficient documentation for extending functionality & - & - \\
	NR6 & Code should follow PEP8 and PEP20 & - & - \\
	NR7 & Code should be documented by docstrings & - & - \\
	\bottomrule
\end{tabularx}
\end{table}


%==================
\chapter{Test Plan}
%==================
This chapter presents the test plan for our solution. The test plan is based on
the standards set by the IEEE829-1998 standard for software testing, but with a
few changes to better fit with our project. The purpose of this plan is to have
a structured way of preforming tests, as well as providing the developers with
a list of specific component-behaviors.The tests will be based on functional as
well as non functional requirements, deterring architectural drift and
enforcing our design plans for the system.

\section{Methods for testing}
%--------------------------------
When it comes to software testing, we have 2 different types of tests available, namely Black box and white box tests. This section is dedicated to the discussion of these 2 testing methodologies.

\section{White box testing}
%----------------------------
White box testing is a method of software testing where you test internal structures or modules of an application, as opposed to its functions. White box testing requires the tester to have an internal perspective of the system, as well as sufficent programming skills. As the utility was required to be able to function with a lot of different input, as well as being used as a debugging tool itself, we chose to have every developer on the team write unit tests for their own code, and then have someone else on the team do the testing of their code in order to ensure correctness.

\section{Black box testing}
%----------------------------
Black box testing is a method of software testing where you test the functionality of a system, as opposed to its internal structures. Black box testing does in general not require the tester to have any intimate knowledge about the system or any of the programming logic that went into making it. Black box test cases are built around the specifications and requirements of a system, i.e its functional and in some cases non-functional requirements. The team decided to use black box testing for both the functional and non-functional requirements of the utility, as the customer had already expressed thoughts on extending and understanding the non-functional parts of utility themselves. 


\section{Templates for testing}
%------------------------------
\autoref{tab:testcase} and \autoref{tab:testreport} are templates we'll be
using for testing purposes.

\begin{table}[ht] \small \center
\caption{Test case template \label{tab:testcase}}
\begin{tabular}{l l}
	\toprule
	Portion & Description \\
	\midrule
	Description & Description of requirement \\
	Tester & Team member responsible for the test \\
	Prerequisites & Conditions that needs to be fulfilled before starting the test \\
	Feature & Feature to test \\
	Execution & Steps to be executed in the test \\
	\bottomrule
\end{tabular}
\end{table}

\begin{table}[ht] \small \center
\caption{Test report template \label{tab:testreport}}
\begin{tabular}{l l}
	\toprule
	Portion & Description \\
	\midrule
	Description & Description of requirement \\
	Tester & Team member responsible for the test \\
	Date & The date the testing took place \\
	Result & Success \\
	\bottomrule
\end{tabular}
\end{table}


%==================================
\chapter{Architectural Description}
%==================================
This chapter introduces the architectural documents pertaining to our solution. The team followed the definition of software architecture defined by Len Bass, Paul Clements and Rick Kazman: "The software architecture of a program or computing
system is the structure of structures of the system, which comprise software elements, the externally visible properties of those elements, and the relationships between them."

The purpose of this document is to describe our architecture in a structured way so that it can be used not only by the team, but also as an aid for other stakeholders who are trying to understand the system.

\section{Architectural Drivers}
%------------------------------
This section is dedicated to the discussion of the architectural drivers, requirements 

\subsection{Quality Attribute Requirements}

\subsection{Business Requirements}

\subsection{Stakeholders and concerns}

\section{Architectural Patters}
%------------------------------

\section{Architectural Rationale}
%--------------------------------


\chapter{Temporary Chapter(Overall Design)}
An attempt at giving an overall design for the program, and its different parts.
\section{Design Goals}
\begin{itemize}
	\item Its better with a smart data structure surrounded by dumb code than a dumb data structure and smart code!!
	\item Clear and clean separation of the front-end and the back-end so in the future other parsers can be used to generate dissectors
	\item Try to be pythonic, follow PEP8 and PEP20.
	\item Now is better than never. Don't be afraid to write stupid or ugly code, we can always fix it later.
	\item The first version is never perfect, so don't wait until its perfect before you commit. Commit often!
\end{itemize}

\section{Architecture}
Figure \ref{fig:archdesign}

\begin{figure}[!ht]
\includegraphics[width=\textwidth]{./planning/img/overall_design.png}
\caption{Overall Architecture}
\label{fig:archdesign}
\end{figure}

\section{Overall design}
\begin{itemize}
	\item The program is split into several parts.
	\item The part which the user runs, should accept arguments which specifies which files to parse and config files to use. It should ask the configuration to parse config files, then ask the front-end to parse c files, and finally ask the back-end to generate Wireshark dissectors.
	\item Configuration should parse config files and feed information to the other parts, or they should request informations when they need it.
	\item Front-end C parser should parse C files and look for struct definitions, which they will fill into some data-structures that the back-end will use.
	\item The data structures should store the information the back-end needs to generate Wireshark dissectors.
	\item The back end should use information in the data structures to generate Wireshark dissectors written in Lua.
\end{itemize}

\section{Command line arguments}
\begin{itemize}
	\item  Should probably use argparse module from python standard library.
	\item Needs to parse commands given when the program is started, and supply them to other parts of the program who needs them.
	\item It must fulfill the following requirements:
	\begin{itemize}
		\item FR7-A Command line shall support parameters for c-header file
		\item FR7-B Command line shall support for configuration file
		\item FR7-C Command line shall support batch mode of c-header and configuration file
		\item This simply means it should accept arguments for 0, 1, or more C code files, and/or 0 or 1 configuration file.
	\end{itemize}
	\item It would be useful if it could also support:
	\begin{itemize}
        		\item -v or -verbose: which should print information about AST tree etc.
        		\item -d or -debug: which should print which steps are happening in the process
        		\item -nocpp: option to disable the C preprocessor
        		\item option(s) to specify which folders to include in the C preprocessor step
        		\item option which specifies where the output should be saved
        		\item printing of help/usage information if one gives it no commands or wrong command
	\end{itemize}
\end{itemize}

\section{Configuration}
\begin{itemize}
	\item Parse one (or more?) configuration files which can be used to modify the process of generation dissectors.
	\item Should maybe only modify data structures?
	\item Challenging part is designing how do we support the different scope of configuration: 
	\begin{itemize}
		\item Can refer to a specific named struct
		\item Can refer to a specific C file?
		\item Can refer to all structs
		\item Can refer to a specific type like time\_t
	\end{itemize}
	\item Must fulfill the following requirements: 
	\begin{itemize}
        		\item Must support valid ranges for struct members
        		\item Must support integer members which represent enumerated named value or a bit string
        		\item Must support custom handling of specific data types
        		\item A struct may have a header and/or trailer (other registered protocol). The configuration must support the use of integer members to indicate the number of other structs that will follow in the trailer
	\end{itemize}
\end{itemize}

\section{C parser, front-end}
\begin{itemize}
	\item Use pycparser and PLY libraries for parsing of C files.
	\item Use cpp and fake libc include files, which comes with pycparser, for C preprocessor step as long as possible.
	\item Should accept C header/code files and create an abtract syntax tree, which it then traverses and finds struct defintions and their members.
	\item Should fill in the necessary information into the data structures, so that the dissector generator can create dissectors for the structs.
	\item Must fulfill the following requirements:
	\begin{itemize}
		\item Must be able to read basic C language struct definitions from C header files
		\item Must support the following basic data types: int, float, char and boolean
		\item Must support members of type enums, structs, unions and arrays
    		\item Must support C preprocessor directives and macros: \#include, \#define, \#if, WIN32, \_WIN32, \_WIN64, sparc, \_\_sparc and sun
	\end{itemize}
\end{itemize}

\section{Data structures}
\begin{itemize}
	\item Should be focused on what data is needed to create Wireshark dissectors.
	\item Should support configuration modifying it.
	\item Should be smart, but not magic.
\end{itemize}

\section{Dissector generator, back-end}
\begin{itemize}
	\item Should accept data structures and create Wireshark dissectors written in Lua.
	\item Must fulfill the following requirements: 
	\begin{itemize}
		\item Must be able to generate lua-script for Wireshark dissectors for the binary representation of C structs.
		\item Shall be able to display simple structs and structs within structs
		\item Must support Wiresharks built-in filter and search on attributes
		\item Shall be able to recognize invalid values for a struct member
	\end{itemize}
\end{itemize}

\section{Handle binary input which size and endian depends on platform}
\begin{itemize}
	\item I dont know yet how we will do this
	\item Flags?
\end{itemize}


%#########
% Sprints
%#########
\part{Sprints}
%=================
\chapter{Sprint 1}
%=================

\section{Back log}
%-----------------
Requirements from the project backlog that we intend to fulfill within this sprint.

\begin{table}[ht] \center
\caption{Sprint 1 Backlog}
\begin{tabular}{c p{6cm} c  c}
	Sprint task & Description & Est. Hours & Act. Hours \\
	\hline
	1 & Command line shall support parameters for c-header file & 2 & -\\
	2 & Support basic data types: int, float, char and boolean & 5 & -\\	
	3 & Support basic data types: int, float, char and boolean & 10 & -\\
	4 & Support basic data types: int, float, char and boolean & 1 & -\\
	5 & Support basic data types: int, float, char and boolean & 1 & -\\	
	6 & Support basic data types: int, float, char and boolean & 5 & -\\
	7 & Support basic data types: int, float, char and boolean & 20 & -\\
\end{tabular}
\end{table}


%=================
\chapter{Sprint 2}
%=================


%------------------------
\section{Sprint Planning}
%------------------------
The first sprint resulted in a solid core for the utility. During the next sprint iteration, the core will be extended with more advanced functionality. After this sprint, the utility will have most of the functionalities it need to work in a real environment, and will probably be able to aid Thales in some of their operations.

Not yet understanding the complexity of all the requirements in the sprint backlog, the team ended in an uncertain person-hours estimate for some work objects. Time will show if we understood the complexity and assigned enough hours to implement it. The more complex, but not so critical functionalities will be part of sprint 3 and 4.   



\subsection{Duration}
%-----------------------
According to the work breakdown structure, \autoref{tab:wbs}, the planning meeting of the second sprint should have been conducted the 5th of October. After a request from the customer to see our planning for the second sprint at the weekly customer meeting, which was scheduled to be before our planning meeting the same day, we decided to advance the planning to the 4th of October. This is to maintain the good relationship to the customer and submit to their preference.

The sprint started with the planning meeting the 4th of October and our work started the following day. The sprint duration is 14 days, and will end the 18th of October with a review meeting.  

\subsection{Sprint Goal}
%-----------------------
The second sprint will build on the core created in the first sprint. During the sprint we will extend the functionality with more comprehensive and advanced features. Most of the requirements we intend to fulfill in this sprint had to be done subsequent to the first sprint, because the structure and design of the core had to be in place first. The requirements that are selected for this sprint is a natural advancement on the way to make the utility that the customer wants. 

One of the most crucial functions to work in a real environment, is the support for nested header-files. The handling of the \#include-statement gives the utility this feature. The goal of the sprint is to implement the \#include and mainly to have support for enums, bit streams, endianness and batch mode. 


\subsection{Back Log}
%-----------------------
The second sprint we will implement twelve requirements. Table
\ref{tab:sp2_req1b}-\ref{tab:sp2_req2d}  list each requirement with a time
estimate. \autoref{tab:sprint2req} is the sprint backlog for the first phase.
\autoref{tab:sprint2time} is the time table for the first phase.

\begin{table}[!ht] \small \center
\caption{Sprint 2 Requirement FR1-B\label{tab:sp2_req1b}}
\begin{tabular}{l l c}
	\toprule
	Requirement & Task & Hours \\
	\midrule
	\multirow{4}{5cm}{ Support members of type enum} & Design & 0 \\
	& Implementation & 3 \\
	& Testing & 3 \\
	& Documentation & 0 \\
	\bottomrule
\end{tabular}
\end{table}

\begin{table}[!ht] \small \center
\caption{Sprint 2 Requirement FR1-C\label{tab:sp2_req1c}}
\begin{tabular}{l l c}
	\toprule
	Requirement & Task & Hours \\
	\midrule
	\multirow{4}{5cm}{Support members of type struct} & Design & 0 \\
	& Implementation & 6 \\
	& Testing & 1 \\
	& Documentation & 0 \\
	\bottomrule
\end{tabular}
\end{table}

\begin{table}[!ht] \small \center
\caption{Sprint 2 Requirement FR1-F\label{tab:sp2_req1f}}
\begin{tabular}{l l c}
	\toprule
	Requirement & Task & Hours \\
	\midrule
	\multirow{4}{5cm}{Detect structs with same name} & Design & 0 \\
	& Implementation & 2 \\
	& Testing & 1 \\
	& Documentation & 0 \\
	\bottomrule
\end{tabular}
\end{table}

\begin{table}[!ht] \small \center
\caption{Sprint 2 Requirement FR2-B\label{tab:sp2_req2b}}
\begin{tabular}{l l c}
	\toprule
	Requirement & Task & Hours \\
	\midrule
	\multirow{4}{5cm}{Support display of structs within structs	} & Design & 0\\
	& Implementation & 5 \\
	& Testing & 6 \\
	& Documentation & 0 \\
	\bottomrule
\end{tabular}
\end{table}

\begin{table}[!ht] \small \center
\caption{Sprint 2 Requirement FR4-F\label{tab:sp2_req4f}}
\begin{tabular}{l l c}
	\toprule
	Requirement & Task & Hours \\
	\midrule
	\multirow{4}{5cm}{Support enumerated named values} & Design & 1 \\
	& Implementation & 1 \\
	& Testing & 2 \\
	& Documentation & 1 \\
	\bottomrule
\end{tabular}
\end{table}

\begin{table}[!ht] \small \center
\caption{Sprint 1 Requirement FR4-G\label{tab:sp2_req4g}}
\begin{tabular}{l l c}
	\toprule
	Requirement & Task & Hours \\
	\midrule
	\multirow{4}{5cm}{Support for bit strings} & Design & 2 \\
	& Implementation & 3 \\
	& Testing & 4 \\
	& Documentation & 1 \\
	\bottomrule
\end{tabular}
\end{table}

\begin{table}[!ht] \small \center
\caption{Sprint 2 Requirement FR1-E\label{tab:sp2_req1e}}
\begin{tabular}{l l c}
	\toprule
	Requirement & Task & Hours \\
	\midrule
	\multirow{4}{5cm}{Support members of type array} & Design & 0 \\
	& Implementation & 3 \\
	& Testing & 4 \\
	& Documentation & 0 \\
	\bottomrule
\end{tabular}
\end{table}

\begin{table}[!ht] \small \center
\caption{Sprint 2 Requirement FR4-E\label{tab:sp2_req4e}}
\begin{tabular}{l l c}
	\toprule
	Requirement & Task & Hours \\
	\midrule
	\multirow{4}{5cm}{Structs with various trailers} & Design & 3 \\
	& Implementation & 6 \\
	& Testing & 7 \\
	& Documentation & 2 \\
	\bottomrule
\end{tabular}
\end{table}

\begin{table}[!ht] \small \center
\caption{Sprint 2 Requirement FR4-B\label{tab:sp2_req4b}}
\begin{tabular}{l l c}
	\toprule
	Requirement & Task & Hours \\
	\midrule
	\multirow{4}{5cm}{Custom Lua configuration} & Design & 2 \\
	& Implementation & 5 \\
	& Testing & 9 \\
	& Documentation & 2 \\
	\bottomrule
\end{tabular}
\end{table}

\begin{table}[!ht] \small \center
\caption{Sprint 2 Requirement FR4-D\label{tab:sp2_req4d}}
\begin{tabular}{l l c}
	\toprule
	Requirement & Task & Hours \\
	\midrule
	\multirow{4}{5cm}{Dissector ID} & Design & 0 \\
	& Implementation & 1 \\
	& Testing & 1 \\
	& Documentation & 2 \\
	\bottomrule
\end{tabular}
\end{table}

\begin{table}[!ht] \small \center
\caption{Sprint 2 Requirement FR5-C\label{tab:sp2_req5c}}
\begin{tabular}{l l c}
	\toprule
	Requirement & Task & Hours \\
	\midrule
	\multirow{4}{5cm}{Endian handling} & Design & 0 \\
	& Implementation & 5 \\
	& Testing & 10 \\
	& Documentation & 0 \\
	\bottomrule
\end{tabular}
\end{table}

\begin{table}[!ht] \small \center
\caption{Sprint 2 Requirement FR6-C\label{tab:sp2_req6c}}
\begin{tabular}{l l c}
	\toprule
	Requirement & Task & Hours \\
	\midrule
	\multirow{4}{5cm}{Batch mode, folder support in the CLI} & Design & 0 \\
	& Implementation & 4 \\
	& Testing & 2 \\
	& Documentation & 1 \\
	\bottomrule
\end{tabular}
\end{table}


\begin{table}[!ht] \small \center
\caption{Sprint 2 Requirements\label{tab:sprint2req}}
\begin{tabularx}{\textwidth}{l l X c c}
	\toprule
	& & & \multicolumn{2}{c}{Hours} \\
	\cmidrule(r){4-5}
	\# & Req. & Description & Est. & Act. \\
	\midrule
	1 & FR1-B & Support members of type enums & - & - \\
	\addlinespace
	2 & FR1-C & Support members of type structs & - & - \\
	\addlinespace
	3 & FR2-B & Support display of structs within structs & - & - \\
	\addlinespace
	4 & FR4-B & Support enumerated named values or a bit strings & - & - \\
	\addlinespace
	5 & FR1-E & Array member & - & - \\
	\addlinespace
	6 & FR5 & Config header/trailer int members & - & - \\
	\addlinespace
	73 & FR4-B & Support enumerated named values or a bit strings & - & - \\
	\midrule
	& & Total: & - & - \\
	\bottomrule
\end{tabularx}
\end{table}

\begin{table}[!ht] \small \center
\caption{Sprint 2 Timetable\label{tab:sprint2time}}
\begin{tabularx}{\textwidth}{X c c}
	\toprule
	& \multicolumn{2}{c}{Hours} \\
	\cmidrule(r){2-3}
	Description & Est. & Act. \\
	\midrule
	Design & 30 & -\\
	\addlinespace
	Implementation & 44 & - \\
	\addlinespace
	Testing & 50 & -\\
	\addlinespace
	Documentation & 36 & -\\
	\midrule
	Total: & 160 & - \\
	\bottomrule
\end{tabularx}
\end{table}



%----------------------
\section{System Design}
%----------------------
For sprint 2 the team decided to refactor some of the code in order to make it easier to read and to split the functionalities of the utility in such a way that it reduces coupling within the system.Some new functionality was also added on the parser side in order to get the utility to recognize the datatypes mentioned in the sprint 2 backlog. Other than that, most of the design didn't change from sprint 1

\subsection{Utility}
%--------------------
Figure \ref{fig:sp2_class} shows the changes we made to the design for the utility for sprint 2. The main change was the creation of the wireshark module where the team gathered most of the functionality for creating dissector fields and looking up default C-types and their sizes. The developers also added new functionality to the config and dissector modules that handles the datatypes the utility was slated to support for sprint 2.
\begin{figure}[htb]
	\center
	\includegraphics[width=\textwidth]{./sprints/img/class_diagram_s2}
	\caption{Class Diagram\label{fig:sp2_class}}
\end{figure}




%---------------------------
\section{Implementation}
%---------------------------

The previous sprint had a focus on creating a native implementation of the 
utility. In this sprint the focus has been on implementing data types for the 
C programming language and make it possible to configure more options on how 
the dissector will function. This section will cover the requirements 
implemented, how they are implemented and what the ''output'' look like.

\subsection{Support enum members}
%------------------------------------------
Enum is a type declaration in C, which specifies enumeration constants.  Enum 
is supported because it is a basic datatype in the C language. 
\autoref{code:cenum} shows an example of an enum in a c-header file. The 
wireshark dissector will displayed the named value, because this will make it 
easier to read, an example is shown in \autoref{fig:wscenum}. 


\begin{figure}[ht]
	\center
	\includegraphics[width=\textwidth]{./sprints/img/wireshark_cenum}
	\caption{Class Diagram\label{fig:wscenum}}
\end{figure}


\lstset{language=C,caption={Enum support},label=code:cenum}
\lstinputlisting[language=C]{./sprints/code/cenum_test.h}



%-----------------------
\section{Sprint Testing}
%-----------------------


%--------------------------
\section{Customer Feedback}
%--------------------------


%--------------------------
\section{Sprint Evaluation}
%--------------------------



%=================
\chapter{Sprint 3}
%=================


%------------------------
\section{Sprint planning}
%------------------------

\subsection{Sprint Goal}
%-----------------------

\subsection{Back log}
%--------------------


%----------------------
\section{System Design}
%----------------------


%-----------------------
\section{Implementation}
%-----------------------


%-----------------------
\section{Sprint Testing}
%-----------------------


%--------------------------
\section{Customer Feedback}
%--------------------------


%--------------------------
\section{Sprint Evaluation}
%--------------------------



%=================
\chapter{Sprint 4}
%=================


%------------------------
\section{Sprint Planning}
%------------------------
The fourth sprint will be the last iteration in this project. Testing and bugfixing will take most of the sprint work hours, because it is very important to make the utility work properly on Thales source code. Now the utility fail on most of the structs: Stig tried to generate dissectors for 580 structs, where only 4 of them succeeded. It is probably just small tweaks needed, in order to make it work. 

\subsection{Duration}
%-----------------------

\subsection{Sprint Goal}
%-----------------------

\subsection{Back Log}
%--------------------


%----------------------
\section{System Design}
%----------------------


%-----------------------
\section{Implementation}
%-----------------------


%-----------------------
\section{Sprint Testing}
%-----------------------


%--------------------------
\section{Customer Feedback}
%--------------------------


%--------------------------
\section{Sprint Evaluation}
%--------------------------




\section{Sprint 1 Tests}
\begin{table}[H]
\begin{tabularx}{\textwidth}{l X}
\hline\hline
Portion & Description\\[0.5ex]
\hline
Test identifier & TID01 Supporting parameters for c-header file\\[0.5ex]
Tester & Team member responsible for the test\\[0.5ex]
Prerequisites & The program has to have been compiled on the system\\[0.5ex]
Feature & Test that we are able to feed the solution with a c-header file and have it get dissected\\
Execution & 1.Start the program\newline
		2.Write the name of the c-header file in the command line\newline
		 3.Read the output given by the program\\ 
Expected result & 1.The program should start up without any errors and present the user with a command line interface\newline
3. The user should be presented with some text expressing the success of the LUA-file generation\\[0.5ex]
\hline\hline
\end{tabularx}


\begin{tabularx}{\textwidth}{l X}
\hline\hline
Portion & Description\\[0.5ex]
\hline
Test identifier & TID02 Supporting basic data types\\[0.5ex]
Tester & Team member responsible for the test\\[0.5ex]
Prerequisites & The program has to have been started\\[0.5ex]
Feature & Test that our utility will be able to make a dissectors for C-header files including the following basic data types: int, float, char and boolean \\
Execution & 1. Write the name of a c-header file which includes the aforementioned basic data types.\newline
		2. Read the output given by the program\\ 
Expected result & 2.The program should provide the user with some text expressing the success of the LUA-file generation\\[0.5ex]
\hline\hline
\end{tabularx}


\begin{tabularx}{\textwidth}{l X}
\hline\hline
Portion & Description\\[0.5ex]
\hline
Test identifier & TID03 Displaying simple structs\\[0.5ex]
Tester & Team member responsible for the test\\[0.5ex]
Prerequisites & The utility has already made a dissector\\[0.5ex]
Feature & Test that our utility is able to generate dissectors that displays simple structs \\
Execution & 1. Open wireshark and run the dissector.\newline
		2. Run the dissector on some captured data of simple structs
		3. Read the output \\
Expected result & 1. Wireshark should be able to load the dissector without any errors \newline
			3. Wireshark should display the data inside the structs sent in the capture data \\[0.5ex]
\hline\hline
\end{tabularx}


\begin{tabularx}{\textwidth}{l X}
\hline\hline
Portion & Description\\[0.5ex]
\hline
Test identifier & TID04 supporting \#include\\[0.5ex]
Tester & Team member responsible for the test\\[0.5ex]
Prerequisites & The utility has to be up and running\\[0.5ex]
Feature & Test that our utility supports c-header files with the \#include directive \\
Execution & 1. Write the name of a C-header file with an \#include directive
		2. Read the output\\
Expected result & 2.The program should provide the user with some text expressing the success of the LUA-file\\ generation\\[0.5ex]
\hline\hline
\end{tabularx}

\begin{tabularx}{\textwidth}{l X}
\hline\hline
Portion & Description\\[0.5ex]
\hline
Test identifier & TID05 supporting \#define and \#if\\[0.5ex]
Tester & Team member responsible for the test\\[0.5ex]
Prerequisites & The utility has to be up and running\\[0.5ex]
Feature & Test that our utility supports c-header files with \#define and \#if directives \\
Execution & 1. Write the name of a C-header file with a \#define and \#if directives
		2. Read the output\\
Expected result & 2.The program should provide the user with some text expressing the success of the LUA-file generation\\[0.5ex]
\hline\hline
\end{tabularx}

\begin{tabularx}{\textwidth}{l X}
\hline\hline
Portion & Description\\[0.5ex]
\hline
Test identifier & TID06 supporting configuration files\\[0.5ex]
Tester & Team member responsible for the test\\[0.5ex]
Prerequisites & The utility has to be up and running\\[0.5ex]
Feature & Test that our utility supports reading data from a configuration file \\[0.5ex]
Execution & PLACEHOLDER FOR NOW\\[0.5ex]
Expected result & PLACEHOLDER FOR NOW\\[0.5ex]
\hline\hline
\end{tabularx}

\end{table}

 % ????


%###########################
% Conclusion and Evaluation
%###########################
\part{Conclusion \& Evaluation}
%===================
\chapter{Conclusion}
%===================

%----------------------------------
\section{Final System Architecture}
%----------------------------------

%----------------
\section{Testing}
%----------------

%----------------
\section{Summary}
%----------------


%===========================
\chapter{Project Evaluation}
%===========================
This chapter gives an evaluation of the project and the course, TDT4290 Customer Driven Project. The main focus is to describe the how the team evolved and how work was done during the project.   

%----------------------------------
\section{Team Dynamics}
%----------------------------------
The forming and development of the team are described in this section.
%----------------------------------
\subsection{Goals And Team Building}
%----------------------------------
The project started out with randomly assigned student groups of six to seven people. This was done intentionally to learn the students to work in a realistic setting. Our group consisted of six Norwegian students and one Czech student. 

As one of the team members was a foreign student, all the internal team communication had to be done in english. In addition our advisors were english speaking, so the whole project was done in english. This did not occur as a problem, because all the team members both spoke and wrote fluent english.

At the first meeting we decided to state our personal goals for the project. This resulted in the following list:
\begin{itemize}
\item zfdfsds
\item sdfsfsdf
\end{itemize}


The shift from group to team came as a result of the team building we had in the start. 


\subsection{Team evolution}
%---------------------------------


\subsection{Risk handling}
%----------------------------------
Some of the risks predicted in the planning phase occured to a certain degree during the project.
This section will discuss those risks and how they were handled.

\paragraph{R4. Illness/Absence}
In general, not many team members were absent for longer periods.
One team member was away in northern Norway for a week on vacation, and another was sick for a week and a half. This caused some delay on a few tasks, as the absentees had to get up to date on the state of the project, but it did not majorly hinder the progress of the team. The consequence of this risk was also diminished by the fact that it occured early in the project, and that the team members who were absent did not work on critical tasks at the time.

\paragraph{R6. Conflicts within team}
In the start of the project, the team members were divided on which tools to use, and on which programming language to use. Because of this, the team had to spend some time discussing back and forth. After some constructive discussions the team was able to come to a mutual decision.

Also, some team members felt that it was not realistic that the course should demand 25 hours from each member. As the project progressed, it quickly became apparent that this number of hours were needed to complete the project in a satisfying manner. This led to an overall increase in work effort in the team.

\paragraph{R8. Miscommunication within team}
During sprint 3, the ones responsible for creating test data and the ones writing the test cases did not clearly communicate with each other. This led to having to make small changes in some of the test cases to be able to use the test data.

The test responsible wrote test cases for functionality that the programming team had not thought about. For example, checking that you are not allowed two platforms with the same platform ID.
When the problem was detected, all essential functionality was added.

\paragraph{R10. Lack of experience with Scrum}
As the team had no previous experience with Scrum, the project got off to a slow start.
After evaluating the first sprint, it quickly became apparent that the process was far from perfect.
The second sprint was an improvement of the first, and the planning meeting was longer and more detailed, but in our opinion it was still not good enough. We felt that we did not adhere to proper Scrum, and that we did not properly explain the different tasks in the backlog. In the last two sprints, we felt that we had achieved a better understanding, and this really showed in the process. A more detailed discussion can be found in the Scrum section.

\paragraph{R11. Requirements added or modified late in the project}
At the customer meeting on the first day of sprint 4, the customer suggested several new requirements that they would like is to implement. Some requirements were also modified, as we had not implemented them exactly the way they wanted us to.

As we had to focus on tweaking some functionalities to work on their code, and also had to spend time on writing the report, we had to tell the customer that we would probably not be able to finish all the new requirements. This is because implementing a requirement would also require testing, user documentation and additional report work, which is something we could not afford to allot time for.


%--------------------------------
\section{The Process}
%under construction!
%--------------------------------
\subsection{Agile Methodology}
%--------------------------------

\subsection{bla}
%--------------------------------


%----------------
\section{Advisors}
%----------------

%----------------
\section{Customer}
%----------------

%----------------
\section{Summary}
%----------------




%############
% Appendices
%############
\part{Appendices}
\appendix
\include{./misc/acronyms}
\include{./misc/glossary}
%==================================
\chapter{User and Developer Manual}
%==================================
\begin{standalone}
    % Generated by Sphinx.
\def\sphinxdocclass{report}
\documentclass[A4paper,10pt,english]{sphinxmanual}
\usepackage[utf8]{inputenc}
\DeclareUnicodeCharacter{00A0}{\nobreakspace}
\usepackage[T1]{fontenc}
\usepackage{babel}
\usepackage{times}
\usepackage[Bjarne]{fncychap}
\usepackage{longtable}
\usepackage{sphinx}
\usepackage{multirow}


\title{CSjark Documentation}
\date{November 16, 2011}
\release{0.3.2}
\author{Erik Bergersen \and Jaroslav Fibichr \and Sondre Johan Mannsverk \and Terje Snarby \and Even Wiik Thomassen \and Lars Solvoll Tønder \and Sigurd Wien}
\newcommand{\sphinxlogo}{}
\renewcommand{\releasename}{Release}
\makeindex

\makeatletter
\def\PYG@reset{\let\PYG@it=\relax \let\PYG@bf=\relax%
    \let\PYG@ul=\relax \let\PYG@tc=\relax%
    \let\PYG@bc=\relax \let\PYG@ff=\relax}
\def\PYG@tok#1{\csname PYG@tok@#1\endcsname}
\def\PYG@toks#1+{\ifx\relax#1\empty\else%
    \PYG@tok{#1}\expandafter\PYG@toks\fi}
\def\PYG@do#1{\PYG@bc{\PYG@tc{\PYG@ul{%
    \PYG@it{\PYG@bf{\PYG@ff{#1}}}}}}}
\def\PYG#1#2{\PYG@reset\PYG@toks#1+\relax+\PYG@do{#2}}

\def\PYG@tok@gd{\def\PYG@tc##1{\textcolor[rgb]{0.63,0.00,0.00}{##1}}}
\def\PYG@tok@gu{\let\PYG@bf=\textbf\def\PYG@tc##1{\textcolor[rgb]{0.50,0.00,0.50}{##1}}}
\def\PYG@tok@gt{\def\PYG@tc##1{\textcolor[rgb]{0.00,0.25,0.82}{##1}}}
\def\PYG@tok@gs{\let\PYG@bf=\textbf}
\def\PYG@tok@gr{\def\PYG@tc##1{\textcolor[rgb]{1.00,0.00,0.00}{##1}}}
\def\PYG@tok@cm{\let\PYG@it=\textit\def\PYG@tc##1{\textcolor[rgb]{0.25,0.50,0.56}{##1}}}
\def\PYG@tok@vg{\def\PYG@tc##1{\textcolor[rgb]{0.73,0.38,0.84}{##1}}}
\def\PYG@tok@m{\def\PYG@tc##1{\textcolor[rgb]{0.13,0.50,0.31}{##1}}}
\def\PYG@tok@mh{\def\PYG@tc##1{\textcolor[rgb]{0.13,0.50,0.31}{##1}}}
\def\PYG@tok@cs{\def\PYG@tc##1{\textcolor[rgb]{0.25,0.50,0.56}{##1}}\def\PYG@bc##1{\colorbox[rgb]{1.00,0.94,0.94}{##1}}}
\def\PYG@tok@ge{\let\PYG@it=\textit}
\def\PYG@tok@vc{\def\PYG@tc##1{\textcolor[rgb]{0.73,0.38,0.84}{##1}}}
\def\PYG@tok@il{\def\PYG@tc##1{\textcolor[rgb]{0.13,0.50,0.31}{##1}}}
\def\PYG@tok@go{\def\PYG@tc##1{\textcolor[rgb]{0.19,0.19,0.19}{##1}}}
\def\PYG@tok@cp{\def\PYG@tc##1{\textcolor[rgb]{0.00,0.44,0.13}{##1}}}
\def\PYG@tok@gi{\def\PYG@tc##1{\textcolor[rgb]{0.00,0.63,0.00}{##1}}}
\def\PYG@tok@gh{\let\PYG@bf=\textbf\def\PYG@tc##1{\textcolor[rgb]{0.00,0.00,0.50}{##1}}}
\def\PYG@tok@ni{\let\PYG@bf=\textbf\def\PYG@tc##1{\textcolor[rgb]{0.84,0.33,0.22}{##1}}}
\def\PYG@tok@nl{\let\PYG@bf=\textbf\def\PYG@tc##1{\textcolor[rgb]{0.00,0.13,0.44}{##1}}}
\def\PYG@tok@nn{\let\PYG@bf=\textbf\def\PYG@tc##1{\textcolor[rgb]{0.05,0.52,0.71}{##1}}}
\def\PYG@tok@no{\def\PYG@tc##1{\textcolor[rgb]{0.38,0.68,0.84}{##1}}}
\def\PYG@tok@na{\def\PYG@tc##1{\textcolor[rgb]{0.25,0.44,0.63}{##1}}}
\def\PYG@tok@nb{\def\PYG@tc##1{\textcolor[rgb]{0.00,0.44,0.13}{##1}}}
\def\PYG@tok@nc{\let\PYG@bf=\textbf\def\PYG@tc##1{\textcolor[rgb]{0.05,0.52,0.71}{##1}}}
\def\PYG@tok@nd{\let\PYG@bf=\textbf\def\PYG@tc##1{\textcolor[rgb]{0.33,0.33,0.33}{##1}}}
\def\PYG@tok@ne{\def\PYG@tc##1{\textcolor[rgb]{0.00,0.44,0.13}{##1}}}
\def\PYG@tok@nf{\def\PYG@tc##1{\textcolor[rgb]{0.02,0.16,0.49}{##1}}}
\def\PYG@tok@si{\let\PYG@it=\textit\def\PYG@tc##1{\textcolor[rgb]{0.44,0.63,0.82}{##1}}}
\def\PYG@tok@s2{\def\PYG@tc##1{\textcolor[rgb]{0.25,0.44,0.63}{##1}}}
\def\PYG@tok@vi{\def\PYG@tc##1{\textcolor[rgb]{0.73,0.38,0.84}{##1}}}
\def\PYG@tok@nt{\let\PYG@bf=\textbf\def\PYG@tc##1{\textcolor[rgb]{0.02,0.16,0.45}{##1}}}
\def\PYG@tok@nv{\def\PYG@tc##1{\textcolor[rgb]{0.73,0.38,0.84}{##1}}}
\def\PYG@tok@s1{\def\PYG@tc##1{\textcolor[rgb]{0.25,0.44,0.63}{##1}}}
\def\PYG@tok@gp{\let\PYG@bf=\textbf\def\PYG@tc##1{\textcolor[rgb]{0.78,0.36,0.04}{##1}}}
\def\PYG@tok@sh{\def\PYG@tc##1{\textcolor[rgb]{0.25,0.44,0.63}{##1}}}
\def\PYG@tok@ow{\let\PYG@bf=\textbf\def\PYG@tc##1{\textcolor[rgb]{0.00,0.44,0.13}{##1}}}
\def\PYG@tok@sx{\def\PYG@tc##1{\textcolor[rgb]{0.78,0.36,0.04}{##1}}}
\def\PYG@tok@bp{\def\PYG@tc##1{\textcolor[rgb]{0.00,0.44,0.13}{##1}}}
\def\PYG@tok@c1{\let\PYG@it=\textit\def\PYG@tc##1{\textcolor[rgb]{0.25,0.50,0.56}{##1}}}
\def\PYG@tok@kc{\let\PYG@bf=\textbf\def\PYG@tc##1{\textcolor[rgb]{0.00,0.44,0.13}{##1}}}
\def\PYG@tok@c{\let\PYG@it=\textit\def\PYG@tc##1{\textcolor[rgb]{0.25,0.50,0.56}{##1}}}
\def\PYG@tok@mf{\def\PYG@tc##1{\textcolor[rgb]{0.13,0.50,0.31}{##1}}}
\def\PYG@tok@err{\def\PYG@bc##1{\fcolorbox[rgb]{1.00,0.00,0.00}{1,1,1}{##1}}}
\def\PYG@tok@kd{\let\PYG@bf=\textbf\def\PYG@tc##1{\textcolor[rgb]{0.00,0.44,0.13}{##1}}}
\def\PYG@tok@ss{\def\PYG@tc##1{\textcolor[rgb]{0.32,0.47,0.09}{##1}}}
\def\PYG@tok@sr{\def\PYG@tc##1{\textcolor[rgb]{0.14,0.33,0.53}{##1}}}
\def\PYG@tok@mo{\def\PYG@tc##1{\textcolor[rgb]{0.13,0.50,0.31}{##1}}}
\def\PYG@tok@mi{\def\PYG@tc##1{\textcolor[rgb]{0.13,0.50,0.31}{##1}}}
\def\PYG@tok@kn{\let\PYG@bf=\textbf\def\PYG@tc##1{\textcolor[rgb]{0.00,0.44,0.13}{##1}}}
\def\PYG@tok@o{\def\PYG@tc##1{\textcolor[rgb]{0.40,0.40,0.40}{##1}}}
\def\PYG@tok@kr{\let\PYG@bf=\textbf\def\PYG@tc##1{\textcolor[rgb]{0.00,0.44,0.13}{##1}}}
\def\PYG@tok@s{\def\PYG@tc##1{\textcolor[rgb]{0.25,0.44,0.63}{##1}}}
\def\PYG@tok@kp{\def\PYG@tc##1{\textcolor[rgb]{0.00,0.44,0.13}{##1}}}
\def\PYG@tok@w{\def\PYG@tc##1{\textcolor[rgb]{0.73,0.73,0.73}{##1}}}
\def\PYG@tok@kt{\def\PYG@tc##1{\textcolor[rgb]{0.56,0.13,0.00}{##1}}}
\def\PYG@tok@sc{\def\PYG@tc##1{\textcolor[rgb]{0.25,0.44,0.63}{##1}}}
\def\PYG@tok@sb{\def\PYG@tc##1{\textcolor[rgb]{0.25,0.44,0.63}{##1}}}
\def\PYG@tok@k{\let\PYG@bf=\textbf\def\PYG@tc##1{\textcolor[rgb]{0.00,0.44,0.13}{##1}}}
\def\PYG@tok@se{\let\PYG@bf=\textbf\def\PYG@tc##1{\textcolor[rgb]{0.25,0.44,0.63}{##1}}}
\def\PYG@tok@sd{\let\PYG@it=\textit\def\PYG@tc##1{\textcolor[rgb]{0.25,0.44,0.63}{##1}}}

\def\PYGZbs{\char`\\}
\def\PYGZus{\char`\_}
\def\PYGZob{\char`\{}
\def\PYGZcb{\char`\}}
\def\PYGZca{\char`\^}
\def\PYGZsh{\char`\#}
\def\PYGZpc{\char`\%}
\def\PYGZdl{\char`\$}
\def\PYGZti{\char`\~}
% for compatibility with earlier versions
\def\PYGZat{@}
\def\PYGZlb{[}
\def\PYGZrb{]}
\makeatother

%


\begin{document}

% \maketitle
% \tableofcontents
\phantomsection\label{index::doc}

\newpage    

CSjark is a tool for generating Lua dissectors from C struct definitions to use with Wireshark. Wireshark is a leading tool for capturing and analysing network traffic. The goal with the dissectors is to make Wireshark able to nicely display the values of a struct sent over the network, along with member names and type. This can be a powerful tool for debugging C programs that communicates with strucs over the network.

For more information about Wireshark please visit \href{http://www.wireshark.org}{Wireshark website}.


\section{User Documentation}
\label{index:user-documentation}\label{index:welcome-to-csjark-s-documentation}

\subsection{Installing CSjark}
\label{user/install::doc}\label{user/install:installing-csjark}

\subsubsection{Dependencies}
\label{user/install:dependencies}
CSjark is written in Python 3.2, and therefore needs Python 3.2 (or later) to run. Latest implementation of Python can be downloaded from \href{http://www.python.org/}{Python website}. For installing please follow the instruction found there.

There are 4 third party dependencies to get CSjark working:
\begin{enumerate}
\item {} \begin{description}
\item[{\textbf{PLY} (Python Lex-Yacc)}] \leavevmode
PLY is an implementation of lex and yacc parsing tools for Python. It is required by pycparser. Instructions and further information can be found on the page linked above.

\begin{tabulary}{\linewidth}{|L|L|}
\hline

\textbf{Required version}
 & 
3.4
\\\hline

\textbf{Download location}
 & 
\href{http://www.dabeaz.com/ply/}{http://www.dabeaz.com/ply/}
\\\hline
\end{tabulary}


\end{description}

\item {} \begin{description}
\item[{\textbf{pycpaser}}] \leavevmode
\href{http://code.google.com/p/pycparser/}{Pycparser} is a C parser (and AST generator) implemented in Python. Due to the continuous development, CSjark requires the latest development version (not the release version).

\begin{tabulary}{\linewidth}{|L|L|}
\hline

\textbf{Required version}
 & 
latest development version from pycparser repository
\\\hline

\textbf{Download location}
 & 
pycparser repository: \href{http://code.google.com/p/pycparser/source/checkout}{http://code.google.com/p/pycparser/source/checkout}
\\\hline
\end{tabulary}


\end{description}

\item {} \begin{description}
\item[{\textbf{C  preprocessor}}] \leavevmode
CSjark requires a C-preprocessor. The way how to get one depends on operating system used by the user:

\begin{tabulary}{\linewidth}{|L|L|}
\hline

\textbf{Windows}
 & 
Bundled with CSjark.
\\\hline

\textbf{OS X, Linux, Solaris}
 & 
Needs to be installed separately. For example, as a part of \href{http://gcc.gnu.org/}{GCC}
\\\hline
\end{tabulary}


\end{description}

\item {} 
\textbf{pyYAML}
\begin{quote}

\href{http://pyyaml.org/wiki/PyYAML}{pyYAML} is a YAML parser and emitter for the Python programming language. \href{http://yaml.org/}{YAML} is a standard used to specify configurations to CSjark. The website includes both a way to download the software and also instructions of how to install it.

\begin{tabulary}{\linewidth}{|L|L|}
\hline

\textbf{Required version}
 & 
3.10
\\\hline

\textbf{Download location}
 & 
\href{http://pyyaml.org/wiki/PyYAML}{http://pyyaml.org/wiki/PyYAML}
\\\hline
\end{tabulary}

\end{quote}

\end{enumerate}


\subsubsection{Wireshark}
\label{user/install:ws}\label{user/install:wireshark}
\href{http://www.wireshark.org/}{Wireshark} is an open source protocol analyzer which can use the Lua dissectors generated by CSjark. To get the proper integration of Lua dissectors, the latest development version of Wireshark is required.

\begin{tabulary}{\linewidth}{|L|L|}
\hline

\textbf{Required version}
 & 
1.7 dev (build 39446 or newer)
\\\hline

\textbf{Download location}
 & 
\href{http://www.wireshark.org/download/automated/}{http://www.wireshark.org/download/automated/}, on the page, browse for the required platform version
\\\hline
\end{tabulary}



\subsubsection{CSjark}
\label{user/install:csjark}
CSjark can be obtained at git CSjark repository: \href{https://github.com/eventh/kpro9/}{https://github.com/eventh/kpro9/}.
CSjark itself requires no installation. After the steps described in the dependencies section is completed. It can be ran by opening a terminal, navigating to the directory containing \code{cshark.py} and invoking as described in section {\hyperref[user/use:use]{\emph{Using CSjark}}}.


\subsection{Using CSjark}
\label{user/use:using-csjark}\label{user/use:use}\label{user/use::doc}
CSjark can be invoked by running the \code{csjark.py} script. The arguments must be specified according to:

\begin{Verbatim}[commandchars=\\\{\}]
csjark.py [-h] [-v] [-d] [-s] [-f [header [header ...]]]
          [-c [config [config ...]]] [-x [path [path ...]]]
          [-o [output]] [-p] [-n] [-C [cpp]] [-i [header [header ...]]]
          [-I [directory [directory ...]]]
          [-D [name=definition [name=definition ...]]]
          [-U [name [name ...]]] [-A [argument [argument ...]]]
          [header] [config]
\end{Verbatim}

The arguments here specify the following:
\begin{description}
\item[{\code{header}}] \leavevmode
a c header file to parse.

\item[{\code{config}}] \leavevmode
a configuration file to parse.

\end{description}

\textbf{Optional arguments:}
\begin{quote}

\begin{tabulary}{\textwidth}{|L|L|}
\hline

\code{-h, -{-}help}
 & 
Show a help message and exit.
\\\hline

\code{-v, -{-}verbose}
 & 
Print detailed information.
\\\hline

\code{-d, -{-}debug}
 & 
Print debugging information.
\\\hline

\code{-s, -{-}strict}
 & 
Only generate dissectors for known structs.
\\\hline

\code{-f {[}header {[}header ...{]}{]}, -{-}file {[}header {[}header ...{]}{]}}
 & 
Specifies that CSjark should look for struct definitions in the \code{header} files.
\\\hline

\code{-c {[}config {[}config ...{]}{]}, -{-}config {[}config {[}config ...{]}{]}}
 & 
Specifies that the program should use the \code{config} files as configuration.
\\\hline

\code{-x {[}path {[}path ...{]}{]}, -{-}exclude {[}path {[}path ...{]}{]}}
 & 
File or folders to exclude from parsing
\\\hline

\code{-o {[}output{]}, -{-}output {[}output{]}}
 & 
Writes the output to the specified file \code{output}.
\\\hline

\code{-p, -{-}placeholders}
 & 
Generate placeholder config file for unknown structs
\\\hline

\code{-n, -{-}nocpp}
 & 
Disables the C pre-processor.
\\\hline

\code{-C {[}cpp{]}, -{-}CPP {[}cpp{]}}
 & 
Specifies which preprocessor to use.
\\\hline

\code{-i {[}header {[}header ...{]}{]}, -{-}include {[}header {[}header ...{]}{]}}
 & 
Process file as Cpp \code{\#include "file"} directive
\\\hline

\code{-I {[}directory {[}directory ...{]}{]}, -{-}Includes {[}directory {[}directory ...{]}{]}}
 & 
Directories to be searched for Cpp includes
\\\hline

\code{-D {[}name=definition {[}name=definition ...{]}{]}, -{-}Define {[}name=definition {[}name=definition ...{]}{]}}
 & 
Predefine name as a Cpp macro
\\\hline

\code{-U {[}name {[}name ...{]}{]}, -{-}Undefine {[}name {[}name ...{]}{]}}
 & 
Cancel any previous Cpp definition of name
\\\hline

\code{-A {[}argument {[}argument ...{]}{]}, -{-}Additional {[}argument {[}argument ...{]}{]}}
 & 
Any additional C preprocessor arguments
\\\hline
\end{tabulary}

\end{quote}

\textbf{Example usage:}

\begin{Verbatim}[commandchars=\\\{\}]
python csjark.py -v headerfile.h configfile.yml
\end{Verbatim}

\textbf{Batch mode}

One of the most important features of CSjark is processing multiple C header files in one run. That can be easily achieved by specifying a directory instead of a single file as command line argument (see above):

\begin{Verbatim}[commandchars=\\\{\}]
python csjark.py headers configs
\end{Verbatim}

In batch mode, CSjark only generates dissectors for structs that have a configuration file with an ID (see section {\hyperref[user/config:ids]{\emph{Dissector message ID}}} for information how to specify dissector message ID), and for structs that depend on other structs. This speeds up the generation of dissectors, since it only generates dissectors that Wireshark can use.


\subsection{Using the generated Lua files in Wireshark}
\label{user/use_ws:using-the-generated-lua-files-in-wireshark}\label{user/use_ws::doc}
These are the steps needed to use a Lua dissector generated by CSjark with Wireshark.
\begin{enumerate}
\item {} 
Get the latest version of Wireshark as described in the installation section {\hyperref[user/install:ws]{\emph{Wireshark}}}.

\item {} 
Locate the Personal configuration and the Personal Plugins directories. To do this, start Wireshark and click on \code{Help} in the menubar and then on \code{About Wireshark}. This should bring up the About Wireshark dialog. From there, navigate to the \code{Folders} tab. Locate folders \code{Personal configuration} and \code{Personal Plugins} and note their paths (see below).

\end{enumerate}

{\hfill\includegraphics[width=\textwidth]{img/ws_about_folders.png}\hfill}
\begin{quote}
\begin{itemize}
\item {} 
on Linux/Unix system it may be  \code{\textasciitilde{}/.wireshark/} and  \code{\textasciitilde{}/.wireshark/plugins/}

\item {} \begin{description}
\item[{on Windows it may be \code{C:\textbackslash{}Users\textbackslash{}*YourUserName*\textbackslash{}AppData\textbackslash{}Roaming\textbackslash{}Wireshark\textbackslash{}}}] \leavevmode
and \code{C:\textbackslash{}Users\textbackslash{}*YourUserName*\textbackslash{}AppData\textbackslash{}Roaming\textbackslash{}Wireshark\textbackslash{}plugins\textbackslash{}}

\end{description}

\end{itemize}

If the folders does not exist, create them.
\end{quote}
\begin{enumerate}
\setcounter{enumi}{2}
\item {} 
Copy CSjark generated file \code{luastructs.lua} into the \code{Personal configuration} folder located in step 1.

\item {} 
Copy CSjark generated Lua dissectors into the \code{Personal Plugins} folder located in step 1.

\item {} 
Open the \code{init.lua} in the \code{Personal configuration} folder located in step 1. Insert the following code:

\begin{Verbatim}[commandchars=\\\{\}]
dofile("luastructs.lua")
\end{Verbatim}

\end{enumerate}
\begin{quote}

This ensures that the \code{luastructs.lua} is loaded before all other Lua scripts.
\end{quote}
\begin{enumerate}
\setcounter{enumi}{5}
\item {} \begin{description}
\item[{Restart Wireshark.}] \leavevmode
To check that the scripts are loaded, navigate to \code{Help} -\textgreater{} \code{About} -\textgreater{} \code{Plugins}. The scripts should now appear in the list as ``lua script''.

\end{description}

\end{enumerate}

{\hfill\includegraphics[width=\textwidth]{img/ws_about_plugins.png}\hfill}

To add further dissectors, only step 2, 5 and 6 needs to be repeated.

For further information on the Lua integration in Wireshark, please visit:
\href{http://www.wireshark.org/docs/wsug\_html\_chunked/wsluarm.html}{Lua Support in Wireshark}.


\subsection{Configuration}
\label{user/config:configuration}\label{user/config::doc}
Because there exists distinct requirements for flexibility of generating dissectors, CSjark supports configuration for various parts of the program. First, general parameters for utility running can be set up. This can be for example settings of variable sizes for different platforms or other parameters that could determine generating dissectors regardless actual C header file. Second, each individual C struct can be treated in different way. For example, value of specific struct member can be checked for being within specified limits.
\setbox0\vbox{
\begin{minipage}{0.95\linewidth}
\textbf{Contents}

\medskip

\begin{itemize}
\item {} 
{\hyperref[user/config:configuration]{Configuration}}
\begin{itemize}
\item {} 
{\hyperref[user/config:configuration-file-format-and-structure]{Configuration file format and structure}}

\item {} 
{\hyperref[user/config:struct-configuration]{Struct Configuration}}
\begin{itemize}
\item {} 
{\hyperref[user/config:value-ranges]{Value ranges}}

\item {} 
{\hyperref[user/config:value-explanations]{Value explanations}}
\begin{itemize}
\item {} 
{\hyperref[user/config:enums]{Enums}}

\item {} 
{\hyperref[user/config:bitstrings]{Bitstrings}}

\end{itemize}

\item {} 
{\hyperref[user/config:dissector-message-id]{Dissector message ID}}

\item {} 
{\hyperref[user/config:external-lua-dissectors]{External Lua dissectors}}
\begin{itemize}
\item {} 
{\hyperref[user/config:support-for-offset-and-value-in-lua-files]{Support for Offset and Value in Lua Files}}

\end{itemize}

\item {} 
{\hyperref[user/config:trailers]{Trailers}}

\item {} 
{\hyperref[user/config:custom-handling-of-data-types]{Custom handling of data types}}

\item {} 
{\hyperref[user/config:unknown-structs-handling]{Unknown structs handling}}

\end{itemize}

\item {} 
{\hyperref[user/config:options-configuration]{Options Configuration}}

\item {} 
{\hyperref[user/config:platform-specific-configuration]{Platform specific configuration}}

\end{itemize}

\end{itemize}
\end{minipage}}
\begin{center}\setlength{\fboxsep}{5pt}\shadowbox{\box0}\end{center}


\subsubsection{Configuration file format and structure}
\label{user/config:configuration-file-format-and-structure}
\textbf{Format}

The configuration files are written in \href{http://www.yaml.org/}{YAML} which is a data serialization format designed to be easy to read and write. Detailed specification can be found at \href{http://www.yaml.org/spec/1.2/spec.html}{YAML website}. The configuration must be put in a \code{filename.yml} file and specified when running CSjark as a command line argument (more about CLI in section {\hyperref[user/use:use]{\emph{Using CSjark}}}).

\textbf{Structure}

CSjark configuration files consist of 2 main parts. The first part is used for specifing all the configuration corresponding CSjark processing in general. More about CSjark options in {\hyperref[user/config:options-configuration]{Options Configuration}}. The second part contains configuration for individual C struct definitions. That is described in section {\hyperref[user/config:struct-configuration]{Struct Configuration}}.

The configuration file may have following strucuture:

\begin{Verbatim}[commandchars=\\\{\}]
Options:
  # there will be all your CSjark processing configuration
  use_cpp: True
  ...

Structs:
  # there will be a sequence of Struct definition configurations
  - name: struct1
    id: [10, 12, 14]
    # another struct1 config
  - name: struct2
    id: [11, 13, 15]
    # another struct2 config
\end{Verbatim}

\textbf{Automatic generation of configuration files}

Autogeneration of configuration file is a simple feature, that could save the user of the utility some time, since  the essential part of the configuration file is generated automatically.  The utility will only create a new file, containg the name of the struct and line to specifiy the ID for the dissector.  To generate the configuration file, the utility must be run with \code{-p} or \code{-{-}placeholders} as an option (see {\hyperref[user/use:use]{\emph{Using CSjark}}} for about CSjark CLI.

% \begin{notice}{note}{Note:}
One part of the configuration is held directly in the code. It represents the platform specific setup (file \code{platform.py}) - see {\hyperref[user/config:platform-specific-configuration]{Platform specific configuration}}.
% \end{notice}


\subsubsection{Struct Configuration}
\label{user/config:struct-configuration}
Each individual C struct processed by CSjark can be treated in different way. All the configuration settings must be done in the \code{Structs} section of the configuration file. Every Struct definition is one item of the sequence and may contain these attributes:

\begin{tabulary}{\linewidth}{|L|L|}
\hline
\textbf{
Attribute name
} & \textbf{
Description
}\\\hline

name
 & 
C struct name (required field)
\\\hline

id
 & 
Dissector message id - more in \emph{Dissector message ID}
\\\hline

description
 & 
Struct name displayed in Wireshark
\\\hline

size
 & 
Size of the struct in memory - more in {\hyperref[user/config:unknown-structs-handling]{Unknown structs handling}}
\\\hline

cnf
 & 
Conformance file name - more in {\hyperref[user/config:external-lua-dissectors]{External Lua dissectors}}
\\\hline

ranges
 & 
Value ranges limitations - more in {\hyperref[user/config:value-ranges]{Value ranges}}
\\\hline

enums
 & 
Enumeration definitions - more in {\hyperref[user/config:enums]{Enums}}
\\\hline

bitstrings
 & 
Bitstrings definitions - more in {\hyperref[user/config:bitstrings]{Bitstrings}}
\\\hline

trailers
 & 
Trailers definitions - more in {\hyperref[user/config:trailers]{Trailers}}
\\\hline

customs
 & 
Definitions for custom struct member handling - more in {\hyperref[user/config:custom-handling-of-data-types]{Custom handling of data types}}
\\\hline
\end{tabulary}



\paragraph{Value ranges}
\label{user/config:value-ranges}
Some variables may have a domain that is smaller than its given type. You could for example use an integer to describe percentage, which is a number between 0 and 100. It is possible to specify this to CSjark, so that the resulting dissector will tell Wireshark if the values are in the specified range or not. Value ranges are defined by the following syntax:

\begin{Verbatim}[commandchars=\\\{\}]
Structs:
  - name: "Name of the struct"
    id: 989
    ranges:
        - member | type: "Name of struct member / type"
          min: "Lowest allowed value"
          max: "Highest allowed value"
\end{Verbatim}

When the definition specified as a type, the value range is applid to all the members of that type within the struct.

Example:

\begin{Verbatim}[commandchars=\\\{\}]
Structs:
  - name: example_struct
    id: 90
    ranges:
        - member: percent
          min: 0
          max: 100
        - type: int
          min: -10
          max: 10
\end{Verbatim}


\paragraph{Value explanations}
\label{user/config:value-explanations}
Some variables may actually represent other values than its type. For example, for an enum it could be preferable to get the textual name of the value displayed, instead of the integer value that represent it. Such example can be an enum type or a bitstring.


\subparagraph{Enums}
\label{user/config:enums}
Values of integer variables can be assigned to string values similarly to enumerated values in most programming languages. Thus, instead of integer value, a corresponding value defined in configuration file as a enumeration can be displayed.

The enumeration definition can be of two types. The first one, mapping specified integer by its struct member name, so it gains string value dependent on the actual integer value. And the second, where assigned string values correspond to every struct member of the type defined in the configuration.

The enum definition, as an attribute of the \code{Structs} item of the configuration file, always starts by \code{enums} keyword. It is followed by list of members/types for which we want to define enumerated integer values for. Each list item consists of 2 mandatory and 1 optional values

\begin{Verbatim}[commandchars=\\\{\}]
- member | type: member name | type name
  values: [value1, value2, ...] | { key1: value1, key2: value2, ...}
  strict: True | False
\end{Verbatim}

where
\begin{itemize}
\item {} 
\code{member name}/\code{type name} contains string value of integer variable name for which we want to define enumerated values

\item {} 
\code{{[}value1, value2, ...{]}} is comma-separated list of enumerated values (implicitly numbered, starting from 0)

\item {} 
\code{\{ key1: value1, key2: value2, ...\}} is comma-separated list of key-value pairs, where \code{key} is integer value and \code{value} is it's assigned string value

\item {} 
\code{strict} is boolean value, which disables warning, if integer does not contain a value specified in the enum list (default \code{True})

\end{itemize}

Example of enums in struct definition contains:
- member named \code{weekday} and values defined as a list of key-value pairs.
- definition of enumerated values for \code{int} type. Values are given by simple list, therefore numbering is implicit (starting from 0, i.e. \code{Blue} = 2). Warning in case of invalid integer value \emph{will} be displayed.

\begin{Verbatim}[commandchars=\\\{\}]
Structs:
  - name: enum_example1
    id: 10
    description: Enum config example
    enums:
      - member: weekday
        values: {1: MONDAY, 2: TUESDAY, 3: WEDNESDAY, 
        4: THURSDAY, 5: FRIDAY, 6: SATURDAY, 7: SUNDAY}
      - type: int
        values: [Black, Red, Blue, Green, Yellow, White]
        strict: True # Disable warning if not a valid value
\end{Verbatim}


\subparagraph{Bitstrings}
\label{user/config:bitstrings}
It is possible to configure bitstrings in the utility. This makes it possible to view common data types like integer, short, float, etc. used as a bitstring in the wireshark dissector.

There is two ways to configure bitstrings, the first one is to specify a struct member and define the bit representation. The second option is to specify bits for all struct members of a given type.

These rules specifies the config:
\begin{itemize}
\item {} 
The bits are specified as 0...n, where 0 is the most significant bit

\item {} 
A bit group can be one or more bits.

\item {} 
Bit groups have a name

\item {} 
It is possible to name all possible values in a bit group.

\end{itemize}

Below, there is an example of a configuration for the member named \code{flags} and all the members of \code{short} type belonging to the struct \code{example}.
\begin{itemize}
\item {} 
member \code{flags}: This example has four bits specified, the first bit group is named ``In use'' and represent bit 0. The second group represent bit 1 and is named ``Endian'', and the values are named: 0 = ``Big'', 1 = ``Little''. The last group is ``Platform'' and represent bit 2-3 and have 4 named values.

\item {} 
type \code{short}: Each of the 3 bits represents one colour channel and it can be either ``True'' or ``False''.

\end{itemize}

\begin{Verbatim}[commandchars=\\\{\}]
Structs:
  - name: example
    id: 1000
    description: An example
    bitstrings:
      - member: flags
        0: In use
        1: [Endian, Big, Little]
        2-3: [Platform, Win, Linux, Mac, Solaris]
      - type: short
        0: Red
        1: Green
        2: Blue
\end{Verbatim}


\paragraph{Dissector message ID}
\label{user/config:dissector-message-id}\label{user/config:ids}
Every packet with C struct captured by Wireshark contains a header. One of the fields in the header, the \code{id} field, specifies which dissector should be loaded to dissect the actual struct. The value of this field can be specified in the configuration file.

This is an example of the specification

\begin{Verbatim}[commandchars=\\\{\}]
Structs:
    - name: structname
      id: 10
\end{Verbatim}

More different messages can be dissected by one specific dissector. Therefore, the struct configuration can contain a whole list of dissector message ID's, that can process the struct.

\begin{Verbatim}[commandchars=\\\{\}]
Structs:
    - name: structname
      id: [12, 43, 3498]
\end{Verbatim}

% \begin{notice}{note}{Note:}
The \code{id} must be an integer between 0 and 65535.
% \end{notice}


\paragraph{External Lua dissectors}
\label{user/config:external-lua-dissectors}
In some cases, CSjark will not be able to deliver the desired result from its own analysis, and the configuration options above may be too constraining. In this case, it is possible to write the lua dissector by hand, either for a given member or for an entire struct.

More information how to write Lua code can be found in \href{http://www.lua.org/manual/5.1/}{Lua reference manual}.

A custom Lua code for desired struct must be defined in an external conformance file with extension \code{.cnf}. The conformance file name and relative path then must be defined in the configuration file for the struct for which is the custom code applied for. The attribute name for the custom Lua definition file and path is \code{cnf}, as shown below:

\begin{Verbatim}[commandchars=\\\{\}]
# CSjark configuration file

Structs:
    - name: custom_lua
      cnf: etc/custom_lua.cnf
      id: 1
      description: example of external custom Lua file definition
\end{Verbatim}

Writing the conformance file implies respecting following rules:
\begin{itemize}
\item {} 
The conformance file (as well as CSjark configuration files) follows \href{http://www.yaml.org/}{YAML} syntax specification.

\item {} 
Each section starts with \code{\#.\textless{}SECTION\textgreater{}} for example \code{\#.COMMENT}.

\item {} 
Unknown sections are ignored.

\end{itemize}

The conformance file implementation allows user to place the custom Lua code on various places within the Lua dissector code already generated by CSjark. There is a list of possible places:
\begin{quote}

\setlength{\tymin}{70pt}
\begin{tabulary}{\textwidth}{|L|L|}
\hline

\code{DEF\_HEADER id}
 & 
Lua code added before a Field defintion.
\\\hline

\code{DEF\_BODY id}
 & 
Lua code to replace a Field defintion. Within the definition, the original body can be referenced as \code{\%(DEFAULT\_BODY)s} or \code{\{DEFAULT\_BODY\}}
\\\hline

\code{DEF\_FOOTER id}
 & 
Lua code added after a Field defintion
\\\hline

\code{DEF\_EXTRA}
 & 
Lua code added after the last defintion
\\\hline

\code{FUNC\_HEADER id}
 & 
Lua code added before a Field function code
\\\hline

\code{FUNC\_BODY id}
 & 
Lua code to replace a Field function code
\\\hline

\code{FUNC\_FOOTER id}
 & 
Lua code added after a Field function code
\\\hline

\code{FUNC\_EXTRA}
 & 
Lua code added at end of dissector function
\\\hline

\code{COMMENT}
 & 
A multiline comment section
\\\hline

\code{END}
 & 
End of a section
\\\hline

\code{END\_OF\_CNF}
 & 
End of the conformance file
\\\hline
\end{tabulary}

\end{quote}

Where \code{id} denotes C struct member name (\code{DEF\_*}) or field name (\code{FUNC\_*}).

Example of such conformance file follows:

\begin{Verbatim}[commandchars=\\\{\}]
#.COMMENT
    This is a .cnf file comment section
#.END

#.DEF_HEADER super
-- This code will be added above the 'super' field definition
#.END

#.COMMENT
    DEF_BODY replaces code inside the dissector function.
    Use %(DEFAULT_BODY)s or {DEFAULT_BODY} to use generated code.
#.DEF_BODY hyper
-- This is above 'hyper' definition
%(DEFAULT_BODY)s
-- This is below 'hyper'
#.END

#.DEF_FOOTER name
-- This is below 'name' definition
#.END


#.DEF_EXTRA
-- This was all the Field defintions
#.END


#.FUNC_HEADER precise
    -- This is above 'precise' inside the dissector function.
#.END


#.COMMENT
    FUNC_BODY replaces code inside the dissector function.
    Use %(DEFAULT_BODY)s or {DEFAULT_BODY} to use generated code.
#.FUNC_BODY name
    --[[ This comments out the 'name' code
     {DEFAULT_BODY}
    ]]--
#.END

#.FUNC_FOOTER super
    -- This is below 'super' inside dissector function
#.END

#.FUNC_EXTRA
    -- This is the last line of the dissector function
#.END_OF_CNF
\end{Verbatim}

This conformance file when run with this C header code:

\begin{Verbatim}[commandchars=\\\{\}]
struct custom_lua {
    short normal;
    int super;
    long long hyper;

    char name;
    double precise;

};
\end{Verbatim}

...will produce this Lua dissector:

\begin{Verbatim}[commandchars=\\\{\}]
-- Dissector for win32.custom_lua: custom_lua (Win32)
local proto_custom_lua = Proto("win32.custom_lua", "custom_lua (Win32)")

-- ProtoField defintions for: custom_lua
local f = proto_custom_lua.fields
f.normal = ProtoField.int16("custom_lua.normal", "normal")
-- This code will be added above the 'super' field definition
f.super = ProtoField.int32("custom_lua.super", "super")
-- This is above 'hyper' definition
f.hyper = ProtoField.int64("custom_lua.hyper", "hyper")
-- This is below 'hyper'
f.name = ProtoField.string("custom_lua.name", "name")
-- This is below 'name' definition
f.precise = ProtoField.double("custom_lua.precise", "precise")
-- This was all the field defintions

-- Dissector function for: custom_lua
function proto_custom_lua.dissector(buffer, pinfo, tree)
    local subtree = tree:add_le(proto_custom_lua, buffer())
    if pinfo.private.caller_def_name then
        subtree:set_text(pinfo.private.caller_def_name .. ": " 
        .. proto_custom_lua.description)
        pinfo.private.caller_def_name = nil
    else
        pinfo.cols.info:append(" (" .. proto_custom_lua.description .. ")")
    end

    subtree:add_le(f.normal, buffer(0, 2))
    subtree:add_le(f.super, buffer(4, 4))
    -- This is below 'super' inside dissector function
    subtree:add_le(f.hyper, buffer(8, 8))
    --[[ This comments out the 'name' code
        subtree:add_le(f.name, buffer(16, 1))
    ]]--
    -- This is above 'precise' inside the dissector function.
    subtree:add_le(f.precise, buffer(24, 8))
    -- This is the last line of the dissector function
end

delegator_register_proto(proto_custom_lua, "Win32", "custom_lua", 1)
\end{Verbatim}


\subparagraph{Support for Offset and Value in Lua Files}
\label{user/config:support-for-offset-and-value-in-lua-files}
Via {\hyperref[user/config:external-lua-dissectors]{External Lua dissectors}} CSjark also provides a way to add new proto fields to the dissector in Wireshark, with correct offset value and correct Lua variable.h

To access the fields value and offset, \code{\{OFFSET\}} and \code{\{VALUE\}} strings may be put into the conformance file as shown below:

\begin{Verbatim}[commandchars=\\\{\}]
#.FUNC_FOOTER pointer
    -- Offset: {OFFSET}
    -- Field value stored in lua variable: {VALUE}
#.END
\end{Verbatim}

Adding the offset and variable value is only possible in the parts that change the code of Lua functions, i.e. \code{FUNC\_HEADER}, \code{FUNC\_BODY} and \code{FUNC\_FOOTER}.

Above listed example leads to following Lua code:

\begin{Verbatim}[commandchars=\\\{\}]
local field_value_var = subtree:add(f.pointer, buffer(56,4))
    -- Offset: 56
    -- Field value stored in lua variable: field_value_var
\end{Verbatim}

% \begin{notice}{note}{Note:}
The value of the referenced variable can be used after it is defined.
% \end{notice}


\paragraph{Trailers}
\label{user/config:trailers}
CSjark only creates dissectors from C structs defined as its input. To be able to use built-in dissectors in Wireshark, it is necessary to configure it. Wireshark has more than 1000 built-in dissectors. Several trailers can be configured for a packet.

The following parameters are allowed in trailers:
\begin{quote}

\begin{tabulary}{\linewidth}{|L|L|}
\hline

name
 & 
Protocol name for the built-in dissector
\\\hline

count
 & 
The number of trailers
\\\hline

member
 & 
Struct member, that contain the amount of trailers
\\\hline

size
 & 
Size of the buffer to feed to the protocol
\\\hline
\end{tabulary}

\end{quote}

There are two ways to configure the trailers - specify the total number of trailers or give a variable in the struct, which contains the amount of trailers. Both ways to configure trailers are shown below. In case the variable \code{trailer\_count} equals 2, the definitions has the same effect.

\begin{Verbatim}[commandchars=\\\{\}]
trailers:
  - name: proto1
    member: trailer_count
    size: 32

trailers:
  - name: proto1
    count: 2
    size: 32
\end{Verbatim}

Example:
The example below shows an example with BER \footnote{
Basic Encoding Rules
}, which has 4 trailers with a size of 6 bytes.

\begin{Verbatim}[commandchars=\\\{\}]
trailers:
  - name: ber
    count: 4
    size: 6
\end{Verbatim}


\paragraph{Custom handling of data types}
\label{user/config:custom-handling-of-data-types}
The utility supports custom handling of specified data types. Some variables in input C header may actually represent other values than its own type. This CSjark feature allows user to map types defined in C header to Wireshark field types. Also, it provides a method to change how the input field is displayed in Wireshark. The custom handling must be done through a configuration file.

For example, this functionality can cause Wireshark to display \code{time\_t} data type as \code{absolute\_time}. The displayed type is given by generated Lua dissector and functions of \code{ProtoField} class.

List of available output types follows:
\begin{description}
\item[{\code{Integer types}}] \leavevmode
uint8, uint16, uint24, uint32, uint64, int8, int16, int24, int32, int64, framenum

\item[{\code{Other types}}] \leavevmode
float, double, string, stringz, bytes, bool, ipv4, ipv6, ether, oid, guid, absolute\_time, relative\_time

\end{description}

For \code{Integer} types, there are some specific attributes that can be defined (see {\hyperref[user/config:below]{below}}). More about each individual type can be found in \href{http://www.wireshark.org/docs/wsug\_html\_chunked/lua\_module\_Proto.html\#lua\_class\_ProtoField}{Wireshark reference}.

The section name in configuration file for custom data type handling is called \code{customs}. This section can contain following attributes:
\begin{itemize}
\item {} 
Required attributes
\begin{quote}

\begin{tabulary}{\linewidth}{|L|L|}
\hline
\textbf{
Attribute name
} & \textbf{
Value
}\\\hline

\code{member} \textbar{} \code{type}
 & 
Name of member or type for which is the configuration applied
\\\hline

\code{field}
 & 
Displayed type (see above)
\\\hline
\end{tabulary}

\end{quote}

\item {} 
Optional attributes - all types
\begin{quote}

\begin{tabulary}{\linewidth}{|L|L|}
\hline
\textbf{
Attribute name
} & \textbf{
Value
}\\\hline

\code{abbr}
 & 
Filter name of the field (the string that is used in filters)
\\\hline

\code{name}
 & 
Actual name of the field
\\\hline

\code{desc}
 & 
The description of the field (displayed on Wireshark statusbar)
\\\hline
\end{tabulary}

\end{quote}

\end{itemize}
\phantomsection\label{user/config:below}\begin{itemize}
\item {} 
Optional attributes - Integer types only:
\begin{quote}

\begin{tabulary}{\linewidth}{|L|L|}
\hline
\textbf{
Attribute name
} & \textbf{
Value
}\\\hline

\code{base}
 & 
Displayed representation - can be one of \code{base.DEC}, \code{base.HEX} or \code{base.OCT}
\\\hline

\code{values}
 & 
List of \code{key:value} pairs representing the Integer value - e.g. \code{\{0: Monday, 1: Tuesday\}}
\\\hline

\code{mask}
 & 
Integer mask of this field
\\\hline
\end{tabulary}

\end{quote}

\end{itemize}

Example of such a configuration file follows:

\begin{Verbatim}[commandchars=\\\{\}]
Structs:
  - name: custom_type_handling
    id: 1
    customs:
      - type: time_t
        field: absolute_time
      - member: day
        field: uint32
        abbr: day.name
        name: Weekday name
        base: base.DEC
        values: { 0: Monday, 1: Tuesday, 2: Wednesday, 3: Thursday, 4: Friday}
        mask: nil
        desc: This day you will work a lot!!
\end{Verbatim}

and applies for example for this C header file:

\begin{Verbatim}[commandchars=\\\{\}]
#include <time.h>

struct custom_type_handling {
    time_t abs;
    int day;
};
\end{Verbatim}

Both struct members are redefined. First will be displayed as \code{absolute\_type} according to its type (\code{time\_t}), second one is changed because of the struct member name (\code{day}).


\paragraph{Unknown structs handling}
\label{user/config:unknown-structs-handling}
The header files that the utility parses, may have nested struct that is not defined in any other header file. To make  it possible to generate a dissector for this case, the user must be able to specify the size of the struct in a configuration file. When the sizes are specified it will be possible to generate a struct that can display the defined members of the struct correctly in the utility, for the parts that are not defined only the hex value will be displayed. This feature is added as a possible way to solve include dependencies that our utility is not able to solve. The user of the utility will get an error message when the utility is not able to find include dependencies, and the user may add the size of struct to be able to generate a dissector for the struct.

The size of unknown struct may be defined directly in the struct configuration as \code{size} attribute, similar to the example below:

\begin{Verbatim}[commandchars=\\\{\}]
Structs:
    - name: unknown struct
      id: 111
      size: 78
\end{Verbatim}

% \begin{notice}{note}{Note:}
Size must be defined as a positive integer (or 0).
% \end{notice}


\subsubsection{Options Configuration}
\label{user/config:options-configuration}
CSjark processing behaviour can be set up in various ways. Besides letting the user to specify how the CSjark should work by the command line arguments (see section {\hyperref[user/use:use]{\emph{Using CSjark}}}), it is also possible to define the options as a part of the configuration file(s).

\begin{tabulary}{\linewidth}{|L|L|L|L|}
\hline
\textbf{
Configuration file field
} & \textbf{
CLI equivalent
} & \textbf{
Value
} & \textbf{
Description
}\\\hline

\code{verbose}
 & 
\code{-v}
 & 
\code{True}/\code{False}
 & 
Print detailed information
\\\hline

\code{debug}
 & 
\code{-d}
 & 
\code{True}/\code{False}
 & 
Print debugging information
\\\hline

\code{strict}
 & 
\code{-s}
 & 
\code{True}/\code{False}
 & 
Only generate dissectors for known structs
\\\hline

\code{output\_dir}
 & 
\code{-o}
 & 
\code{None} or path
 & 
Definition of output destination
\\\hline

\code{output\_file}
 & 
\code{-o}
 & 
\code{None} or file name
 & 
Writes the output to the specified file
\\\hline

\code{generate\_placeholders}
 & 
\code{-p}
 & 
\code{True}/\code{False}
 & 
Generate placeholder config file for unknown structs
\\\hline

\code{use\_cpp}
 & 
\code{-n}
 & 
\code{True}/\code{False}
 & 
Enables/disables the C pre-processor
\\\hline

\code{cpp\_path}
 & 
\code{-C}
 & 
\code{None} or file name
 & 
Specifies which preprocessor to use
\\\hline

\code{excludes}
 & 
\code{-x}
 & 
List of excluded paths
 & 
File or folders to exclude from parsing
\\\hline

\code{platforms}
 &  & 
List of platform names
 & 
Set of platforms to support in dissectors
\\\hline

\code{include\_dirs}
 & 
\code{-I}
 & 
List of directories
 & 
Directories to be searched for Cpp includes
\\\hline

\code{includes}
 & 
\code{-i}
 & 
List of includes
 & 
Process file as Cpp \#include ``file'' directive
\\\hline

\code{defines}
 & 
\code{-D}
 & 
List of defines
 & 
Predefine name as a Cpp macro
\\\hline

\code{undefines}
 & 
\code{-U}
 & 
List of undefines
 & 
Cancel any previous Cpp definition of name
\\\hline

\code{arguments}
 & 
\code{-A}
 & 
List of additional arguments
 & 
Any additional C preprocessor arguments
\\\hline
\end{tabulary}


The last 5 options can be also specified separately for each individual input C header file. This can be achieved by adding sequence \code{files} with mandatory attribute \code{name}.

Below you can see an example of such \code{Options} section:

\begin{Verbatim}[commandchars=\\\{\}]
Options:
    verbose: True
    debug: False
    strict: False
    output_dir: ../out
    output_file: output.log
    generate_placeholders: False
    use_cpp: True
    cpp_path: ../utils/cpp.exe
    excludes: [examples, test]
    platforms: [default, Win32, Win64, Solaris-sparc, Linux-x86]
    include_dirs: [../more_includes]
    includes: [foo.h, bar.h]
    defines: [CONFIG_DEFINED=3, REMOVE=1]
    undefines: [REMOVE]
    arguments: [-D ARR=2]
    files:
      - name: a.h
        includes: [b.h, c.h]
        define: [MY_DEFINE]
\end{Verbatim}

% \begin{notice}{note}{Note:}
If you give CSjark multiple configuration files with the same values defined, it takes:
\begin{itemize}
\item {} 
for attributes with single value: a value from \emph{last processed config file} is valid

\item {} 
for attributes with list values: lists are \emph{merged}

\end{itemize}
% \end{notice}


\subsubsection{Platform specific configuration}
\label{user/config:platform-specific-configuration}
To ensure that CSjark is usable as much as possible, platform specific

Entire platform setup is done via Python code, specifically \code{platform.py}. This file contains following sections:
\begin{enumerate}
\item {} 
Platform class definition including it's methods

\item {} 
Default mapping of C type and their wireshark field type

\item {} 
Default C type size in bytes

\item {} 
Default alignment size in bytes

\item {} 
Custom C type sizes for every platform which differ from default

\item {} 
Custom alignment sizes for every platform which differ from default

\item {} 
Platform-specific C preprocessor macros

\item {} 
Platform registration method and calling for each platform

\end{enumerate}

When defining new platform, following steps should be done. Referenced sections apply to \code{platform.py} sections listed above. All the new dictionary variables should have proper syntax of \href{http://docs.python.org/release/3.1.3/tutorial/datastructures.html\#dictionaries}{Python dictionary}:
\begin{description}
\item[{\textbf{Field sizes}}] \leavevmode
Define custom C type sizes in section 5. Create new dictionary with name in capital letters. Only those different from default (section 3) must be defined.

\begin{Verbatim}[commandchars=\\\{\}]
NEW_PLATFORM_C_SIZE_MAP = {
    'unsigned long': 8,
    'unsigned long int': 8,
    'long double': 16
}
\end{Verbatim}

\item[{\textbf{Memory alignment}}] \leavevmode
Define custom memory alignment sizes in section 6. Create new dictionary with name in capital letters. Only those different from default (section 4) must be defined.

\begin{Verbatim}[commandchars=\\\{\}]
NEW_PLATFORM_C_ALIGNMENT_MAP = {
    'unsigned long': 8,
    'unsigned long int': 8,
    'long double': 16
}
\end{Verbatim}

\item[{\textbf{Macros}}] \leavevmode
Define dictionary of platform specific macros in section 7. These macros then can be used within C header files to define platform specific struct members etc. E.g.:

\begin{Verbatim}[commandchars=\\\{\}]
#if _WIN32
    float num;
#elif __sparc
    long double num;
#else
    double num;
\end{Verbatim}


Example of such macros:

\begin{Verbatim}[commandchars=\\\{\}]
NEW_PLATFORM_MACROS = {
    '__new_platform__': 1, '__new_platform': 1
}
\end{Verbatim}

\item[{\textbf{Register platform}}] \leavevmode
In last section (8), the new platform must be registered. Basically, it means calling the constructor of Platform class. That has following parameters:

\begin{Verbatim}[commandchars=\\\{\}]
Platform(name, flag, endian, macros=None, sizes=None, alignment=None)
\end{Verbatim}

where

\begin{tabulary}{\linewidth}{|L|L|}
\hline

\code{name}
 & 
name of the platform
\\\hline

\code{flag}
 & 
unique integer value representing this platform
\\\hline

\code{endian}
 & 
either \code{Platform.big} or \code{Platform.little}
\\\hline

\code{macros}
 & 
C preprocessor platform-specific macros like \_WIN32
\\\hline

\code{sizes}
 & 
dictionary which maps C types to their size in bytes
\\\hline
\end{tabulary}


Registering of the platform then might look as follows:

\begin{Verbatim}[commandchars=\\\{\}]
# New platform
Platform('New-platform', 8, Platform.little,
         macros=NEW_PLATFORM_MACROS,
         sizes=NEW_PLATFORM_C_SIZE_MAP,
         alignment=NEW_PLATFORM_C_ALIGNMENT_MAP)
\end{Verbatim}

\end{description}


\section{Other Information}
\label{index:yaml}\label{index:other-information}

\subsection{Copyright}
\label{other/license::doc}\label{other/license:copyright}
Copyright (c) 2011, Erik Bergersen, Jaroslav Fibichr, Sondre Johan Mannsverk, Terje Snarby, Even Wiik Thomassen, Lars Solvoll Tonder, Sigurd Wien. All rights reserved.


\subsection{License \& Warranty}

\label{other/license:license-warranty}\label{other/license:license}

\textbf{License}\\

Redistribution and use in source and binary forms, with or without
modification, are permitted provided that the following conditions are met:
Redistributions of source code must retain the above copyright notice, this list of conditions and the following disclaimer.

Redistributions in binary form must reproduce the above copyright notice, this list of conditions and the following disclaimer in the documentation and/or other materials provided with the distribution.
\\

\textbf{Warranty}\\

THIS SOFTWARE IS PROVIDED BY THE COPYRIGHT HOLDERS AND CONTRIBUTORS ``AS IS'' AND ANY EXPRESS OR IMPLIED WARRANTIES, INCLUDING, BUT NOT LIMITED TO, THE IMPLIED WARRANTIES OF MERCHANTABILITY AND FITNESS FOR A PARTICULAR PURPOSE ARE DISCLAIMED. IN NO EVENT SHALL THE COPYRIGHT HOLDER OR CONTRIBUTORS BE LIABLE FOR ANY DIRECT, INDIRECT, INCIDENTAL, SPECIAL, EXEMPLARY, OR CONSEQUENTIAL DAMAGES (INCLUDING, BUT NOT LIMITED TO, PROCUREMENT OF SUBSTITUTE GOODS OR SERVICES; LOSS OF USE, DATA, OR PROFITS; OR BUSINESS INTERRUPTION) HOWEVER CAUSED AND ON ANY THEORY OF LIABILITY, WHETHER IN CONTRACT, STRICT LIABILITY, OR TORT (INCLUDING NEGLIGENCE OR OTHERWISE) ARISING IN ANY WAY OUT OF THE USE
OF THIS SOFTWARE, EVEN IF ADVISED OF THE POSSIBILITY OF SUCH DAMAGE.


\subsection{About these documents}
\label{other/about::doc}\label{other/about:about-these-documents}
These documents are generated from \href{http://docutils.sf.net/rst.html}{reStructuredText} sources by \href{http://sphinx.pocoo.or}{Sphinx}, a
document processor specifically written for the Python documentation.

These documents are hosted at \href{http://www.readthedocs.org/}{ReadTheDocs}:
\href{http://csjark.readthedocs.org/}{http://csjark.readthedocs.org/}.


\bigskip\hrule{}\bigskip


\textbf{Currently supported platforms}
\begin{itemize}
\item {} 
Windows 32-bit

\item {} 
Windows 64-bit

\item {} 
Solaris 32-bit

\item {} 
Solaris 64-bit

\item {} 
Solaris SPARC 64-bit

\item {} 
MacOS

\item {} 
Linux 32 bit

\end{itemize}

(additional platforms can be added by configuration)

\renewcommand{\indexname}{Index}
% \printindex
\end{document}

\end{standalone}

\include{./misc/feedback}


\end{document}

