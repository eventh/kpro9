%\chapter{Use Cases}
%\section{Actors}
%\section{High-Level Use Case Diagram}
%\section{Low-Level Use Case Diagrams}

\chapter{Overview}

\section{Introduction}
This document contains requirements for an utility that allows Wireshark to
interpret the binary representations of C-language structs. While C structs
seldom are exchanged across networks, they are sometimes used in inter-process
communication. The purpose of the utility described here is to provide
Wireshark with the capability of automatically dissecting the binary
representation of a C struct, as long as its definition is known.

The expected work flow for the utility is to read one or more C header files,
which contain struct definitions, and output Wireshark dissectors, implemented
in Lua scripts. A configuration file or source scode annotations in the header
files may be used when additional configuration is required.

\section{Requirements}
An overview over both functional and non-functional requirements.

TODO: create less detailed lists of those found futher down.

\chapter{Prioritization}

\chapter{Requirements}
\autoref{tab:funcreq} lists the functional requirements and their priority,
while \autoref{tab:nonfuncreq} lists the non-functional requirements.

\begin{table}[h] \center
\caption{Functional Requirements\label{tab:funcreq}}
\begin{tabular}{| c | p{9cm} | c |}
	\hline ID & Description & Priority \\
    \hline FR01 & The utility shall be able to read basic C language struct
		definitions, and generate a Wireshark dissector for the binary
		representation of the structs. & TODO \\
	\hline FR02 & The utility shall support structs with the following basic
		data types: int, float, char, boolean, structs, unions, array and
		enums. & TODO \\
	\hline FR03 & The utility must support the following C preprocessor
		directives and macros: \#include, WIN32, TODO!! & TODO \\
	\hline FR04 & The dissector shall be able to recognize invalid values for
		a struct member. Allowed ranges should be specified by configuration.
		& TODO \\
	\hline FR05 & A struct may have a header and/or trailer (other registered
		protocol), which must be configurable. & TODO \\
	\hline FR06 & The dissector shall be able to display each struct member.
		Structs within structs shall also be dissected and displayed. & TODO \\
	\hline FR07 & It shall be possible to configure special handling of
		specific data types. E.g. a 'time\_t' may be interpreted to contain a
		unixtime value, and be displayed as a date. & TODO \\
	\hline FR08 & An integer member may indicate that a variable number of
		other structs (array of structs) are following the current struct.
		& TODO \\
	\hline FR09 & Integers may be an enumerated named value or a bit string.
		TODO!!!
		& TODO \\
	\hline FR10 & The dissectors must be able to handle binary input which size
		and endian depends on originating platform. & TODO \\
	\hline
\end{tabular}
\end{table}

\begin{table}[h] \center
\caption{Non-Functional Requirements\label{tab:nonfuncreq}}
\begin{tabular}{| c | p{9cm} | c |}
	\hline ID & Description & Priority \\
	\hline NR01 & The utility shall be able to run on Windows 32bit \& 64bit,
		Solaris 64bit and Sparc. & TODO \\
	\hline NR02 & The utility shall be able to run in batch mode. & TODO \\
	\hline NR03 & The utility needs to have flexible configuration. & TODO \\
	\hline NR04 & The configuration needs to be well documentet. & TODO \\
	\hline NR05 & The utility and its code should have some documentation
		& TODO \\
	\hline
\end{tabular}
\end{table}

\chapter{Test plan?}

