%\chapter{Use Cases}
%\section{Actors}
%\section{High-Level Use Case Diagram}
%\section{Low-Level Use Case Diagrams}

\chapter{Overview}

\section{Introduction}
This document contains requirements for an utility that allows Wireshark to
interpret the binary representations of C-language structs. While C structs
seldom are exchanged across networks, they are sometimes used in inter-process
communication. The purpose of the utility described here is to provide
Wireshark with the capability of automatically dissecting the binary
representation of a C struct, as long as its definition is known.

The expected work flow for the utility is to read one or more C header files,
which contain struct definitions, and output Wireshark dissectors, implemented
in Lua scripts. A configuration file or source scode annotations in the header
files may be used when additional configuration is required.

\section{Requirements}
An overview over both functional and non-functional requirements.

TODO: create less detailed lists of those found futher down.

\chapter{Prioritization}

\chapter{Functional Requirements}
\begin{description}
	\item[F01] The utility shall be able to read basic C language struct
		definitions, and generate a Wireshark dissector for the binary
		representation of the structs.
	\item[F02] The utility shall support structs with any of the basic data
		types (e.g. int, boolean, float, char) and structs.
	\item[F03] The utility shall be able to follow \#include <...> statements.
		This allows parsing structs that depend on structs or defines from
		other header files.
	\item[F04] The dissector shall be able to recognize invalid values for a
		struct member. Allowed ranges should be specified by configuration. An
		example is an integer that indictates a percentage between 0 and 100.
	\item[F05] A struct may have a header and/or trailer (other registered
		protocol). This must be configurable.
	\item[F06] The dissector shall be able to display each struct member.
		Structs within structs shall also be dissected and displayed.
	\item[F07] It shall be possible to configure special handling of specific
		data types. E.g. a 'time\_t' may be interpreted to contain a unixtime
		value, and be displayed as a date.
	\item[F08] An integer member may indicate that a variable number of other
		structs (array of structs) are following the current struct.
	\item[F09] Integers may be an enumerated named value or a bit string.
	\item[F10] The dissectors produced shall be able to handle binary input
		from at least Windows 32bit and 64bit, Solaris 64bit and Sparc.
		Example: BOOL is 1 byte on Solaris and 4 bytes on Win32. Endian and
		alignment also differs between the architectures.
\end{description}

\chapter{Non-Functional Requirements}

\chapter{Test plan?}

