%=====================
\chapter{Requirements}
%=====================
\label{chap:requirements}
This chapter describes an utility that creates Wireshark dissectors from C
header files. The dissectors must interpret binary representations of C
structs. In \autoref{sec:reqsoverview} we give a high level overview of the
utility and its requirements, \autoref{sec:usecases} provides use cases for the
utility, \autoref{sec:reqspriority} explains how we prioritizes the
requirements and \autoref{sec:reqslist} lists all the functional and
non-function requirements and their priority. Since this project follows Scrum,
each sprint chapter contains its own backlog.

\section{Overview}
%-----------------
\label{sec:reqsoverview}
This chapter specifies requirements for an utility that allows Wireshark to
interpret the binary representations of C-language structs. While C structs
seldom are exchanged across networks, they are sometimes used in inter-process
communication. The purpose of the utility described here is to provide
Wireshark with the capability of automatically dissecting the binary
representation of a C struct, as long as its definition is known.

The expected work flow for the utility is to read one or more C header files,
which contain struct definitions, and output Wireshark dissectors, implemented
in Lua scripts. A configuration file or source code annotations in the header
files may be used when additional configuration is required.

\begin{description}
    \item[FR01] Read C structs and generate Wireshark dissectors
	\item[FR02] Support structs with basic C data types
	\item[FR03] Support C preprocessor \#include \#define etc.
	\item[FR04] Recognize invalid values for struct members
	\item[FR05] A struct may have a header and/or trailer
	\item[FR06] Support structs within structs
	\item[FR07] Support custom handling of specific data types
	\item[FR08] Support integers which indicate array of structs are following
	\item[FR09] Configuration for enumerated named value or a bit strings
	\item[FR10] Handle platform dependent endian and type sizes
	\item[NR01] Run on Windows 32bit \& 64bit, Solaris 64bit and Sparc
	\item[NR02] Be able to run in batch mode
	\item[NR03] Need to have flexible configuration
	\item[NR04] The configuration needs to be well documented
\end{description}

See \autoref{sec:reqslist} for more detailed description of the
requirements.

\section{Use Cases}
%------------------
\label{sec:usecases}

\subsection{Actors}
% * Configurer
% * User of our solution
% * User of wireshark dissectors

\subsection{High-Level Use Case Diagram}

\subsection{Low-Level Use Case Diagrams}


\section{Prioritization}
%-----------------------
\label{sec:reqspriority}

\section{List of requirements}
%-----------------------------
\label{sec:reqslist}
\autoref{tab:funcreq} lists the functional requirements and their priority,
while \autoref{tab:nonfuncreq} lists the non-functional requirements.

\begin{table}[ht] \center
\caption{Functional Requirements\label{tab:funcreq}}
\begin{tabular}{| c | p{9cm} | c |}
	\hline ID & Description & Priority \\
    \hline FR01 & The utility shall be able to read basic C language struct
		definitions, and generate a Wireshark dissector for the binary
		representation of the structs. & High \\
	\hline FR02 & The utility shall support structs with the following basic
		data types: int, float, char, boolean, structs, unions, array and
		enums. & High \\
	\hline FR03 & The utility must support the following C preprocessor
		directives and macros: \verb+#include+, \verb+#define+, \verb+#if+,
		\verb+WIN32+, \verb+_WIN32+, \verb+_WIN64+, \verb+__sparc__+,
		\verb+__sparc+ and \verb+sun+ & Medium \\
	\hline FR04 & The dissector shall be able to recognize invalid values for
		a struct member. Allowed ranges should be specified by configuration.
		& Low \\
	\hline FR05 & A struct may have a header and/or trailer (other registered
		protocol), which must be configurable. & Low \\
	\hline FR06 & The dissector shall display each struct member, and support
		structs within structs. & Medium \\
	\hline FR07 & Configuration must support custom handling of
		specific data types. E.g. a 'time\_t' may be interpreted to contain a
		unixtime value, and be displayed as a date. & Low \\
	\hline FR08 & Configuration must support integer members which indicate
		that a variable number of other structs (array of structs) are
		following the current struct. & High \\
	\hline FR09 & Configuration must support integer members which represent
		enumerated named value or a bit string. & Medium \\
	\hline FR10 & The dissectors must be able to handle binary input which size
		and endian depends on originating platform. Flags within message
		headers should signal the platform. & High \\
	\hline
\end{tabular}
\end{table}

\begin{table}[ht] \center
\caption{Non-Functional Requirements\label{tab:nonfuncreq}}
\begin{tabular}{| c | p{9cm} | c |}
	\hline ID & Description & Priority \\
	\hline NR01 & The utility shall be able to run on Windows and Solaris,
		32bit and 64bit, Intel and Sparc platform. & High \\
	\hline NR02 & The utility shall be able to run in batch mode. & High \\
	\hline NR03 & The utility needs to have flexible configuration. & Medium \\
	\hline NR04 & The configuration needs to be well documented. & Low \\
	\hline
\end{tabular}
\end{table}

