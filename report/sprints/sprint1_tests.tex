\section{Sprint 1 Tests}
%-----------------------

\begin{table}[ht] \footnotesize \center
\caption{Test case TID01}
\begin{tabularx}{\textwidth}{l X}
	\toprule
	Portion & Description \\
	\midrule
	Description & Supporting parameters for c-header file \\
	Tester & Team member responsible for the test \\
	Prerequisites & The program has to have been compiled on the system \\
	Feature & Test that we are able to feed the solution with a c-header file and have it get dissected \\
	Execution & \begin{packed_enum}
		\item Start the program
		\item Write the name of the c-header file in the command line
		\item Read the output given by the program \end{packed_enum} \\
	Expected result & \begin{packed_enum}
		\item The program should start up without any errors and present the user with a command line interface
		\setcounter{enumi}{2}
		\item The user should be presented with some text expressing the success of the LUA-file generation \end{packed_enum} \\
	\bottomrule
\end{tabularx}
\end{table}

\begin{table}[ht] \footnotesize \center
\caption{Test case TID02}
\begin{tabularx}{\textwidth}{l X}
	\toprule
	Portion & Description \\
	\midrule
	Description & Supporting basic data types \\
	Tester & Team member responsible for the test \\
	Prerequisites & The program has to have been started \\
	Feature & Test that our utility will be able to make a dissectors for C-header files including the following basic data types: int, float, char and boolean \\
	Execution & \begin{packed_enum}
		\item Write the name of a c-header file which includes the aforementioned basic data types
		\item Read the output given by the program \end{packed_enum} \\ 
	Expected result & \begin{packed_enum} \setcounter{enumi}{1}
		\item The program should provide the user with some text expressing the success of the LUA-file generation \end{packed_enum} \\
	\bottomrule
\end{tabularx}
\end{table}

\begin{table}[ht] \footnotesize \center
\caption{Test case TID03}
\begin{tabularx}{\textwidth}{l X}
	\toprule
	Portion & Description \\
	\midrule
	Description & Displaying simple structs \\
	Tester & Team member responsible for the test \\
	Prerequisites & The utility has already made a dissector \\
	Feature & Test that our utility is able to generate dissectors that displays simple structs \\
	Execution & \begin{packed_enum}
		\item Open wireshark and run the dissector
		\item Run the dissector on some captured data of simple structs
		\item Read the output \end{packed_enum} \\
	Expected result & \begin{packed_enum}
		\item Wireshark should be able to load the dissector without any errors
		\setcounter{enumi}{2}
		\item Wireshark should display the data inside the structs sent in the capture data \end{packed_enum} \\
	\bottomrule
\end{tabularx}
\end{table}

\begin{table}[ht] \footnotesize \center
\caption{Test case TID04}
\begin{tabularx}{\textwidth}{l X}
	\toprule
	Portion & Description \\
	\midrule
	Description & Supporting \#include \\
	Tester & Team member responsible for the test \\
	Prerequisites & The utility has to be up and running \\
	Feature & Test that our utility supports c-header files with the \#include directive \\
	Execution & \begin{packed_enum}
		\item Write the name of a C-header file with an \#include directive
		\item Read the output \end{packed_enum} \\
	Expected result & \begin{packed_enum}
		\setcounter{enumi}{1}
		\item The program should provide the user with some text expressing the success of the LUA-file generation \end{packed_enum} \\
	\bottomrule
\end{tabularx}
\end{table}

\begin{table}[ht] \footnotesize \center
\caption{Test case TID05}
\begin{tabularx}{\textwidth}{l X}
	\toprule
	Portion & Description \\
	\midrule
	Description & Supporting \#define and \#if \\
	Tester & Team member responsible for the test \\
	Prerequisites & The utility has to be up and running \\
	Feature & Test that our utility supports c-header files with \#define and \#if directives \\
	Execution & \begin{packed_enum}
		\item Write the name of a C-header file with a \#define and \#if directives
		\item Read the output \end{packed_enum} \\
	Expected result & \begin{packed_enum}
		\setcounter{enumi}{1}
		\item The program should provide the user with some text expressing the success of the LUA-file generation \end{packed_enum} \\
	\bottomrule
\end{tabularx}
\end{table}

\begin{table}[ht] \footnotesize \center
\caption{Test case TID06}
\begin{tabularx}{\textwidth}{l X}
	\toprule
	Portion & Description \\
	\midrule
	Description & Supporting configuration files \\
	Tester & Team member responsible for the test \\
	Prerequisites & The utility has to be up and running \\
	Feature & Test that our utility supports reading data from a configuration file \\
	Execution & PLACEHOLDER FOR NOW \\
	Expected result & PLACEHOLDER FOR NOW \\
	\bottomrule
\end{tabularx}
\end{table}

